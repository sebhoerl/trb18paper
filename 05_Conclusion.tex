\section{Discussion \& Conclusion}
\label{sec:Conclusion}

The study shows that the right choice of dispatching algorithm for an AMoD system
does not only have strong impact on the performance in terms of waiting times for
the customer, but also that it bears a significant economic advantage for the
operator. He is able to attract more customers through quicker pickups and
lower prices than a competitor with only little investment.

In order to assess the significance for real fleets of (not neccessarily
automated) taxis it needs to be noted that all of the presented algorithms are
able to process dispatching and rebalancing tasks for fleets of thousands of
vehicles within minutes. It is perfectly feasible to control 100k vehicles in
five minute updates using a standard laptop for the computational tasks.

For the presented simulations, this still poses a burden, though, because there
a speedup compared to reality of around one thousand times is desired to be able
to run large numbers of simulations with different parameters. Hence, the algorithms
could only be tested on a subsample of 1\% of the agent population that is available.
In future studies effort will be put into overcoming these restriction, either
by finding approximate formulations for the presented algorithms or pursuing research
on completely new algorithms.

Throughout the paper, a ``100\%'' demand scenario has been used, in which all
trips that possibly could be undertaken by AV were converted to the automated
mode. The MATSim framework, however, offers the possibility to explicitly
simulate attitudes toward new elements in the traffic system by defining utilities
for using specific modes with distinct valuation of travel costs, travel times and
distances. This way, by integrating the presented algorithms into the full
MATSim loop as shown in \cite{horl_abmtrans17} the actual attractiveness of an
AV service could be analyszed including the tradeoff that people make between
paying for the service, spending time in the vehicle and having to wait for it.
Naturally, not 100\% of possible trips would actually be performed by AV, but only
a fraction. In such a scenario, also if maybe more remote areas would be included,
completely different properties of an AV fleet control algorithm would be of
interest, e.g. how well it is able to attract new customer groups in new regions
by offering unproportinally low waiting times and make them stick to the service.



[ TODO: OTHER LIMITATIONS ]
