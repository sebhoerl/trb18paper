%%%%%%%%%%%%%%%%%%%%%%%%%%%%%%%%%%%%%%%%%%%%%%%%%%%%%%%%%%%%%%%%%%%%%%
%%%%%%%%%%%%%%%%%%%%%%%%%%%%%%%%%%%%%%%%%%%%%%%%%%%%%%%%%%%%%%%%%%%%%%
%%
%% IVT LaTeX template
%%   Kirill Müller
%%   kirill.mueller@ivt.baug.ethz.ch
%%
%%%%%%%%%%%%%%%%%%%%%%%%%%%%%%%%%%%%%%%%%%%%%%%%%%%%%%%%%%%%%%%%%%%%%%
%%%%%%%%%%%%%%%%%%%%%%%%%%%%%%%%%%%%%%%%%%%%%%%%%%%%%%%%%%%%%%%%%%%%%%

%%%%%%%%%%%%%%%%%%%%%%%%%%%%%%%%%%%%%%%%%%%%%%%%%%%%%%%%%%%%%%%%%%%%%%
%%%%%%%%%%%%%%%%%%%%%%%%%%%%%%%%%%%%%%%%%%%%%%%%%%%%%%%%%%%%%%%%%%%%%%
%%
%% Encoding check:
%%   ä  ö  ü  Ä  Ö  Ü  ß  á  é  í  ó  ú  à  è  ì  ò  ù  â  ê  î  ô  û
%%
%%   If the above contains rubbish, please re-open this file using
%%   one of the following encodings: Latin1, ISO-8859-1, Windows-1252.
%%   DO NOT save the file in this case!
%%
%%%%%%%%%%%%%%%%%%%%%%%%%%%%%%%%%%%%%%%%%%%%%%%%%%%%%%%%%%%%%%%%%%%%%%
%%%%%%%%%%%%%%%%%%%%%%%%%%%%%%%%%%%%%%%%%%%%%%%%%%%%%%%%%%%%%%%%%%%%%%

%%%%%%%%%%%%%%%%%%%%%%%%%%%%%%%%%%%%%%%%%%%%%%%%%%%%%%%%%%%%%%%%%%%%%%
%%%%%%%%%%%%%%%%%%%%%%%%%%%%%%%%%%%%%%%%%%%%%%%%%%%%%%%%%%%%%%%%%%%%%%
%%
%% This is an example for writing a paper at the IVT.
%% It supports English and German language.
%% By using it, it is possible to switch paper layouts easily
%% (e.g., working paper style into TRB style)
%%
%% The easiest way to write you own paper is to create an
%% appropriate directory stucture in the papers subdirectory
%% i.e. papers/strc/2007/mypaper,
%% copy this template and modify it there.
%%
%% See the note below on renaming the main file.
%%
%% Each keyword is documented. Just follow the instructions...
%% Enjoy!
%%
%%%%%%%%%%%%%%%%%%%%%%%%%%%%%%%%%%%%%%%%%%%%%%%%%%%%%%%%%%%%%%%%%%%%%%
%%%%%%%%%%%%%%%%%%%%%%%%%%%%%%%%%%%%%%%%%%%%%%%%%%%%%%%%%%%%%%%%%%%%%%

%%%%%%%%%%%%%%%%%%%%%%%%%%%%%%%%%%%%%%%%%%%%%%%%%%%%%%%%%%%%%%%%%%%%%%
%%%%%%%%%%%%%%%%%%%%%%%%%%%%%%%%%%%%%%%%%%%%%%%%%%%%%%%%%%%%%%%%%%%%%%
%%
%% IMPORTANT NOTE ON FILE RENAMES:
%%   If you rename this file, look for the text "Template"
%%   in all files and change this to the new filename, too.
%%   There is a script that will do this for you in
%%   Linux/Windows+cygwin or Mac OS X: Simply execute
%%
%%       _latexfiles/tools/rename.sh Template NewName
%%
%%   Substitute NewName with a name of your choice.
%%
%%   In Windows, use your favorite find-in-text-files tool
%%
%%%%%%%%%%%%%%%%%%%%%%%%%%%%%%%%%%%%%%%%%%%%%%%%%%%%%%%%%%%%%%%%%%%%%%
%%%%%%%%%%%%%%%%%%%%%%%%%%%%%%%%%%%%%%%%%%%%%%%%%%%%%%%%%%%%%%%%%%%%%%

%%%%%%%%%%%%%%%%%%%%%%%%%%%%%%%%%%%%%%%%%%%%%%%%%%%%%%%%%%%%%%%%%%%%%%
%% Location of the common files
%%   If unsure, please leave this as it is
\newcommand{\mypath}{_latexfiles/}
%\newcommand{\mypath}{../../../}
%%%%%%%%%%%%%%%%%%%%%%%%%%%%%%%%%%%%%%%%%%%%%%%%%%%%%%%%%%%%%%%%%%%%%%

%%%%%%%%%%%%%%%%%%%%%%%%%%%%%%%%%%%%%%%%%%%%%%%%%%%%%%%%%%%%%%%%%%%%%%
%% language specification:
%%   Here you define in which langugage your paper will be written.
%%   There are ALWAYS 2 languages to define (even you do not need it.)
%%   Since we are writing only in German or English, other languages
%%   are not supported.
%%   Choose either 'german' or 'english' as your first language.
\newcommand{\myfirstlang}{english}
%%%%%%%%%%%%%%%%%%%%%%%%%%%%%%%%%%%%%%%%%%%%%%%%%%%%%%%%%%%%%%%%%%%%%%

%%%%%%%%%%%%%%%%%%%%%%%%%%%%%%%%%%%%%%%%%%%%%%%%%%%%%%%%%%%%%%%%%%%%%%
%% Include Paper-Layout:
%%   Here the ivt working paper layout is chosen
%%   For changing this paper into i.e. TRB layout just change "ivt-wp"
%%   to "trb"
% !TeX encoding = usascii
% Prepare to get rid of \mypath one day
\providecommand\mypath[3]{}
%%%%%%%%%%%%%%%%%%%%%%%%%%%%%%%%%%%%%%%%%%%%%%%%%%%%%%%%%%%%%%%%%%%%%%
%% $Id: trb-lineNumbered.tex 12326 2013-10-11 10:04:04Z muelleki $
%%%%%%%%%%%%%%%%%%%%%%%%%%%%%%%%%%%%%%%%%%%%%%%%%%%%%%%%%%%%%%%%%%%%%%

%%%%%%%%%%%%%%%%%%%%%%%%%%%%%%%%%%%%%%%%%%%%%%%%%%%%%%%%%%%%%%%%%%%%%%
%%
%% TRB PAPER LAYOUT
%% Date: 2007-06-28
%% author:
%%   Michael Balmer, balmer@ivt.baug.ethz.ch
%%
%%%%%%%%%%%%%%%%%%%%%%%%%%%%%%%%%%%%%%%%%%%%%%%%%%%%%%%%%%%%%%%%%%%%%%

\input{\mypath../_latexfiles/_layouts/trb}
\input{\mypath../_latexfiles/_layouts/_lineno-pagewise}

%%%%%%%%%%%%%%%%%%%%%%%%%%%%%%%%%%%%%%%%%%%%%%%%%%%%%%%%%%%%%%%%%%%%%%

%%%%%%%%%%%%%%%%%%%%%%%%%%%%%%%%%%%%%%%%%%%%%%%%%%%%%%%%%%%%%%%%%%%%%%
%%%%%%%%%%%%%%%%%%%%%%%%%%%%%%%%%%%%%%%%%%%%%%%%%%%%%%%%%%%%%%%%%%%%%%
%%
%% START DOCUMENT KEYWORDS
%%   In this part you can define Authors, Date, Titlefigure, etc...
%%
%%%%%%%%%%%%%%%%%%%%%%%%%%%%%%%%%%%%%%%%%%%%%%%%%%%%%%%%%%%%%%%%%%%%%%
%%%%%%%%%%%%%%%%%%%%%%%%%%%%%%%%%%%%%%%%%%%%%%%%%%%%%%%%%%%%%%%%%%%%%%
\usepackage{caption}

\usepackage{epstopdf}
\usepackage[english]{babel}
%\usepackage[latin1]{inputenc}
%\usepackage{bibentry}
%\usepackage{tikz}
%\usepackage{amsmath,amsfonts,amssymb}
\usepackage{tikz}
\usepackage{amsmath,amsfonts,amssymb,nccmath}

%\usetheme{Singapore}
%\usecolortheme{default}

 \usetikzlibrary{arrows}
 \usetikzlibrary{patterns,decorations}
\usetikzlibrary{shapes,snakes,shadows}

\usepackage{rotating}
\usepackage{adjustbox}
\usepackage{setspace}
\usetikzlibrary{positioning}
\usetikzlibrary{chains}
%%%%%%%%%%%%%%%%%%%%%%%%%%%%%%%%%%%%%%%%%%%%%%%%%%%%%%%%%%%%%%%%%%%%%%
%% The figure to include in the title page:
%% - {path/to/the/figure}: includes this figure in the title
%% - {}: no figure will be included
%% Note:
%%   do not write the ending of your figure ("MATSimLoop" instead
%%   of "MATSimLoop.pdf")
\newcommand{\mytextwordcount}{6843}%

\newcommand{\mytitlefigure}{}
%%%%%%%%%%%%%%%%%%%%%%%%%%%%%%%%%%%%%%%%%%%%%%%%%%%%%%%%%%%%%%%%%%%%%%

%%%%%%%%%%%%%%%%%%%%%%%%%%%%%%%%%%%%%%%%%%%%%%%%%%%%%%%%%%%%%%%%%%%%%%
%% The title of the paper
%\newcommand{\mytitle}{Simulation-based assessment of fleet control algorithms for autonomous mobility on demand systems for the Zurich case}
%Axhausen title:
\newcommand{\mytitle}{Fleet control algorithms for autonomous mobility: A simulaton assessment for Zurich}
%%%%%%%%%%%%%%%%%%%%%%%%%%%%%%%%%%%%%%%%%%%%%%%%%%%%%%%%%%%%%%%%%%%%%%

%%%%%%%%%%%%%%%%%%%%%%%%%%%%%%%%%%%%%%%%%%%%%%%%%%%%%%%%%%%%%%%%%%%%%%
%% The institution (group) for which the paper is written
\newcommand{\myinstitutionEN}{}
%%%%%%%%%%%%%%%%%%%%%%%%%%%%%%%%%%%%%%%%%%%%%%%%%%%%%%%%%%%%%%%%%%%%%%

%%%%%%%%%%%%%%%%%%%%%%%%%%%%%%%%%%%%%%%%%%%%%%%%%%%%%%%%%%%%%%%%%%%%%%
%% The number of the paper
\newcommand{\mynumber}{}
%%%%%%%%%%%%%%%%%%%%%%%%%%%%%%%%%%%%%%%%%%%%%%%%%%%%%%%%%%%%%%%%%%%%%%

%%%%%%%%%%%%%%%%%%%%%%%%%%%%%%%%%%%%%%%%%%%%%%%%%%%%%%%%%%%%%%%%%%%%%%
%% The year of the paper
\newcommand{\myyear}{2017}
%%%%%%%%%%%%%%%%%%%%%%%%%%%%%%%%%%%%%%%%%%%%%%%%%%%%%%%%%%%%%%%%%%%%%%

%%%%%%%%%%%%%%%%%%%%%%%%%%%%%%%%%%%%%%%%%%%%%%%%%%%%%%%%%%%%%%%%%%%%%%
%% The month of the paper
%% - include only: jan,feb,mar,apr,may,jun,jul,aug,sep,oct,nov,dec
\newcommand{\mymonth}{aug}
%%%%%%%%%%%%%%%%%%%%%%%%%%%%%%%%%%%%%%%%%%%%%%%%%%%%%%%%%%%%%%%%%%%%%%

%%%%%%%%%%%%%%%%%%%%%%%%%%%%%%%%%%%%%%%%%%%%%%%%%%%%%%%%%%%%%%%%%%%%%%
%% The day of the paper
%% - include day in number 1,..., 12,..., 31
\newcommand{\myday}{01}
%%%%%%%%%%%%%%%%%%%%%%%%%%%%%%%%%%%%%%%%%%%%%%%%%%%%%%%%%%%%%%%%%%%%%%

%%%%%%%%%%%%%%%%%%%%%%%%%%%%%%%%%%%%%%%%%%%%%%%%%%%%%%%%%%%%%%%%%%%%%%
%% The keywords (English)
\newcommand{\mykeywordsEN}{activity scoring, MATSim}
%%%%%%%%%%%%%%%%%%%%%%%%%%%%%%%%%%%%%%%%%%%%%%%%%%%%%%%%%%%%%%%%%%%%%%

%%%%%%%%%%%%%%%%%%%%%%%%%%%%%%%%%%%%%%%%%%%%%%%%%%%%%%%%%%%%%%%%%%%%%%
%% The keywords (German)
%\newcommand{\mykeywordsDE}{Schlüsselwörter, auf Deutsch, Sprache}
%%%%%%%%%%%%%%%%%%%%%%%%%%%%%%%%%%%%%%%%%%%%%%%%%%%%%%%%%%%%%%%%%%%%%%

%%%%%%%%%%%%%%%%%%%%%%%%%%%%%%%%%%%%%%%%%%%%%%%%%%%%%%%%%%%%%%%%%%%%%%
%%
%% Define the authors
%% - if author name is left empty
%%   \newcommand{\mysecondauthor}{}
%%   it is not shown
%% - keep the right order (first, second, etc...) and keep the
%%   remaining entries empty!
%% - also add the way the authors appear in the reference style
%%
%%%%%%%%%%%%%%%%%%%%%%%%%%%%%%%%%%%%%%%%%%%%%%%%%%%%%%%%%%%%%%%%%%%%%%

%%%%%%%%%%%%%%%%%%%%%%%%%%%%%%%%%%%%%%%%%%%%%%%%%%%%%%%%%%%%%%%%%%%%%%
%% The author 1
\newcommand{\myfirstauthor}{Sebastian Hörl (joint first co-author)}
\newcommand{\myfirstauthorREF}{Hörl, S.}
%%%%%%%%%%%%%%%%%%%%%%%%%%%%%%%%%%%%%%%%%%%%%%%%%%%%%%%%%%%%%%%%%%%%%%
%% The author 2
\newcommand{\mysecondauthor}{Claudio Ruch (joint first co-author)}
\newcommand{\mysecondauthorREF}{Ruch, C.}
%%%%%%%%%%%%%%%%%%%%%%%%%%%%%%%%%%%%%%%%%%%%%%%%%%%%%%%%%%%%%%%%%%%%%%
%% The author 3
\newcommand{\mythirdauthor}{Felix Becker}
\newcommand{\mythirdauthorREF}{Becker, F.}
%%%%%%%%%%%%%%%%%%%%%%%%%%%%%%%%%%%%%%%%%%%%%%%%%%%%%%%%%%%%%%%%%%%%%%
%% The author 4
\newcommand{\myfourthauthor}{Kay W. Axhausen}
\newcommand{\myfourthauthorREF}{Axhausen, K. W.}
%%%%%%%%%%%%%%%%%%%%%%%%%%%%%%%%%%%%%%%%%%%%%%%%%%%%%%%%%%%%%%%%%%%%%%
%% The author 5
\newcommand{\myfifthauthor}{Emilio Frazzoli}
\newcommand{\myfifthauthorREF}{Frazzoli, E.}
%%%%%%%%%%%%%%%%%%%%%%%%%%%%%%%%%%%%%%%%%%%%%%%%%%%%%%%%%%%%%%%%%%%%%%
%% The author 6
\newcommand{\mysixthauthor}{}
\newcommand{\mysixthauthorREF}{}
%% The author 7
\newcommand{\myseventhauthor}{}
\newcommand{\mysevenauthorREF}{}
%%%%%%%%%%%%%%%%%%%%%%%%%%%%%%%%%%%%%%%%%%%%%%%%%%%%%%%%%%%%%%%%%%%%%%
%% .. up to 12 authors supported by the templates
%%%%%%%%%%%%%%%%%%%%%%%%%%%%%%%%%%%%%%%%%%%%%%%%%%%%%%%%%%%%%%%%%%%%%%

%%%%%%%%%%%%%%%%%%%%%%%%%%%%%%%%%%%%%%%%%%%%%%%%%%%%%%%%%%%%%%%%%%%%%%
%%
%% Define the addresses
%% - if author name is left empty
%%   \newcommand{\mysecondaddress}{}
%%   it is not shown
%% - if two authors share the same affiliation, they should also use
%%   the same address
%% - keep the right order (first, second, etc...) and keep the
%%   remaining entries empty!
%% - also add the way the authors appear in the reference style
%%
%%%%%%%%%%%%%%%%%%%%%%%%%%%%%%%%%%%%%%%%%%%%%%%%%%%%%%%%%%%%%%%%%%%%%%

%%%%%%%%%%%%%%%%%%%%%%%%%%%%%%%%%%%%%%%%%%%%%%%%%%%%%%%%%%%%%%%%%%%%%%
%% Increase this if you have more than two addresses
\newcommand{\mynumaddresscolumns}{4}
%%%%%%%%%%%%%%%%%%%%%%%%%%%%%%%%%%%%%%%%%%%%%%%%%%%%%%%%%%%%%%%%%%%%%%
%% The first address
\newcommand{\myfirstaddress}{
  \createcontact{\myfirstauthor}%
  {IVT}
  {ETH Zürich}
  {8093 Zürich, Switzerland}
  {+41-44-633-38-01}
  {}
  {sebastian.hoerl@ivt.baug.ethz.ch}
}
%%%%%%%%%%%%%%%%%%%%%%%%%%%%%%%%%%%%%%%%%%%%%%%%%%%%%%%%%%%%%%%%%%%%%%
%% The second address
\newcommand{\mysecondaddress}{%
  \createcontact{\mysecondauthor}%
  {IDSC}
  {ETH Zürich}
  {8092 Zürich, Switzerland}
  {}
  {}
  {clruch@idsc.mavt.ethz.ch}
}
%%%%%%%%%%%%%%%%%%%%%%%%%%%%%%%%%%%%%%%%%%%%%%%%%%%%%%%%%%%%%%%%%%%%%%
%% The third address
\newcommand{\mythirdaddress}{%
	\createcontact{\mythirdauthor}%
	{IVT}
	{ETH Zürich}
	{8093 Zürich, Switzerland}
	{+41-44-633-65-29}
	{}
	{felix.becker@ivt.baug.ethz.ch}
}
%%%%%%%%%%%%%%%%%%%%%%%%%%%%%%%%%%%%%%%%%%%%%%%%%%%%%%%%%%%%%%%%%%%%%%
%% The fourth address
\newcommand{\myfourthaddress}{%
    \createcontact{\myfourthauthor}%
    {IVT}
    {ETH Zürich}
    {8093 Zürich, Switzerland}
    {+41-44-633-39-43}
    {}
    {axhausen@ivt.baug.ethz.ch}
}
%%%%%%%%%%%%%%%%%%%%%%%%%%%%%%%%%%%%%%%%%%%%%%%%%%%%%%%%%%%%%%%%%%%%%%
%% The fifth address
\newcommand{\myfifthaddress}{%
    \createcontact{\myfifthauthor}%
    {IDSC}
    {ETH Zürich}
    {8092 Zürich, Switzerland}
    {+41-44-632-79-28}
    {}
    {emilio.frazzoli@idsc.mavt.ethz.ch}
}
%%%%%%%%%%%%%%%%%%%%%%%%%%%%%%%%%%%%%%%%%%%%%%%%%%%%%%%%%%%%%%%%%%%%%%
%% The sixth address
\newcommand{\mysixthaddress}{%
}

%%%%%%%%%%%%%%%%%%%%%%%%%%%%%%%%%%%%%%%%%%%%%%%%%%%%%%%%%%%%%%%%%%%%%%
%%%%%%%%%%%%%%%%%%%%%%%%%%%%%%%%%%%%%%%%%%%%%%%%%%%%%%%%%%%%%%%%%%%%%%
%%
%% END DOCUMENT KEYWORDS
%%
%%%%%%%%%%%%%%%%%%%%%%%%%%%%%%%%%%%%%%%%%%%%%%%%%%%%%%%%%%%%%%%%%%%%%%
%%%%%%%%%%%%%%%%%%%%%%%%%%%%%%%%%%%%%%%%%%%%%%%%%%%%%%%%%%%%%%%%%%%%%%


%%%%%%%%%%%%%%%%%%%%%%%%%%%%%%%%%%%%%%%%%%%%%%%%%%%%%%%%%%%%%%%%%%%%%%
%%%%%%%%%%%%%%%%%%%%%%%%%%%%%%%%%%%%%%%%%%%%%%%%%%%%%%%%%%%%%%%%%%%%%%
%%
%% START USER DEFINED COMMANDS
%%   Sometimes latex does not do hyphenations. This happens if it
%%   does not recognize a specific word. with "\hyphenation"
%%   you can add rules for those words
%% for advanced users:
%%   Advanced users can add additional commands here
%%
%%%%%%%%%%%%%%%%%%%%%%%%%%%%%%%%%%%%%%%%%%%%%%%%%%%%%%%%%%%%%%%%%%%%%%
%%%%%%%%%%%%%%%%%%%%%%%%%%%%%%%%%%%%%%%%%%%%%%%%%%%%%%%%%%%%%%%%%%%%%%

%%%%%%%%%%%%%%%%%%%%%%%%%%%%%%%%%%%%%%%%%%%%%%%%%%%%%%%%%%%%%%%%%%%%%%
%% Word split: unknown words for latex. show how to split them.
\hyphenation{Trenn-re-geln}
%%%%%%%%%%%%%%%%%%%%%%%%%%%%%%%%%%%%%%%%%%%%%%%%%%%%%%%%%%%%%%%%%%%%%%


%%%%%%%%%%%%%%%%%%%%%%%%%%%%%%%%%%%%%%%%%%%%%%%%%%%%%%%%%%%%%%%%%%%%%%
%%%%%%%%%%%%%%%%%%%%%%%%%%%%%%%%%%%%%%%%%%%%%%%%%%%%%%%%%%%%%%%%%%%%%%
%%
%% END USER DEFINED COMMANDS
%%
%%%%%%%%%%%%%%%%%%%%%%%%%%%%%%%%%%%%%%%%%%%%%%%%%%%%%%%%%%%%%%%%%%%%%%
%%%%%%%%%%%%%%%%%%%%%%%%%%%%%%%%%%%%%%%%%%%%%%%%%%%%%%%%%%%%%%%%%%%%%%

%%%%%%%%%%%%%%%%%%%%%%%%%%%%%%%%%%%%%%%%%%%%%%%%%%%%%%%%%%%%%%%%%%%%%%
%%%%%%%%%%%%%%%%%%%%%%%%%%%%%%%%%%%%%%%%%%%%%%%%%%%%%%%%%%%%%%%%%%%%%%
%%
%% To speed up compilation, you can split this file and move the
%%   contents of the upper part (just before the %&Template line)
%%   to a new file named Template.ltx.  This requires a working
%%   installation of GNU Make (included in Linux/Mac OS X,
%%   in Windows: through cygwin or GnuWin32).
%%
%% The commands between \iffalse and \fi will do this for you
%% (Linux/Mac OS X):
\iffalse
mv Template.tex Template.tmp
grep -B 10000 '^%&Template' Template.tmp > Template.ltx
grep -A 10000 '^%&Template' Template.tmp > Template.tex
rm Template.tmp
\fi
%&Template
%\input{Template.ltx}

%%%%%%%%%%%%%%%%%%%%%%%%%%%%%%%%%%%%%%%%%%%%%%%%%%%%%%%%%%%%%%%%%%%%%%
%%%%%%%%%%%%%%%%%%%%%%%%%%%%%%%%%%%%%%%%%%%%%%%%%%%%%%%%%%%%%%%%%%%%%%
%%
%% START OF DOCUMENT
%%   Here actually begins your document. The things below
%%   are a typical order in which a paper should be organized.
%%   Sometimes, editors have other suggenstions. In this case
%%   just change the order in which each part should appear.
%%
%%%%%%%%%%%%%%%%%%%%%%%%%%%%%%%%%%%%%%%%%%%%%%%%%%%%%%%%%%%%%%%%%%%%%%
%%%%%%%%%%%%%%%%%%%%%%%%%%%%%%%%%%%%%%%%%%%%%%%%%%%%%%%%%%%%%%%%%%%%%%
\usepackage{amsmath}
\begin{document}


%%%%%%%%%%%%%%%%%%%%%%%%%%%%%%%%%%%%%%%%%%%%%%%%%%%%%%%%%%%%%%%%%%%%%%
%% Include the title page
\createtitlepage
%%%%%%%%%%%%%%%%%%%%%%%%%%%%%%%%%%%%%%%%%%%%%%%%%%%%%%%%%%%%%%%%%%%%%%

%%%%%%%%%%%%%%%%%%%%%%%%%%%%%%%%%%%%%%%%%%%%%%%%%%%%%%%%%%%%%%%%%%%%%%
%% Page numbering is taken care of by the templates
%%%%%%%%%%%%%%%%%%%%%%%%%%%%%%%%%%%%%%%%%%%%%%%%%%%%%%%%%%%%%%%%%%%%%%

%%%%%%%%%%%%%%%%%%%%%%%%%%%%%%%%%%%%%%%%%%%%%%%%%%%%%%%%%%%%%%%%%%%%%%
%% Table of contents
%\tableofcontents
%%%%%%%%%%%%%%%%%%%%%%%%%%%%%%%%%%%%%%%%%%%%%%%%%%%%%%%%%%%%%%%%%%%%%%

%%%%%%%%%%%%%%%%%%%%%%%%%%%%%%%%%%%%%%%%%%%%%%%%%%%%%%%%%%%%%%%%%%%%%%
%% List of figures and tables
%\listoffigures
%%%%%%%%%%%%%%%%%%%%%%%%%%%%%%%%%%%%%%%%%%%%%%%%%%%%%%%%%%%%%%%%%%%%%%

%%%%%%%%%%%%%%%%%%%%%%%%%%%%%%%%%%%%%%%%%%%%%%%%%%%%%%%%%%%%%%%%%%%%%%
%% List of tables
%\listoftables
%%%%%%%%%%%%%%%%%%%%%%%%%%%%%%%%%%%%%%%%%%%%%%%%%%%%%%%%%%%%%%%%%%%%%%

%%%%%%%%%%%%%%%%%%%%%%%%%%%%%%%%%%%%%%%%%%%%%%%%%%%%%%%%%%%%%%%%%%%%%%
%% If the next part should start at a new page insert \clearpage
%% command
\clearpage
%%%%%%%%%%%%%%%%%%%%%%%%%%%%%%%%%%%%%%%%%%%%%%%%%%%%%%%%%%%%%%%%%%%%%%

%%%%%%%%%%%%%%%%%%%%%%%%%%%%%%%%%%%%%%%%%%%%%%%%%%%%%%%%%%%%%%%%%%%%%%
%% Page numbering is taken care of by the templates
%%%%%%%%%%%%%%%%%%%%%%%%%%%%%%%%%%%%%%%%%%%%%%%%%%%%%%%%%%%%%%%%%%%%%%

%%%%%%%%%%%%%%%%%%%%%%%%%%%%%%%%%%%%%%%%%%%%%%%%%%%%%%%%%%%%%%%%%%%%%%

%%%%%%%%%%%%%%%%%%%%%%%%%%%%%%%%%%%%%%%%%%%%%%%%%%%%%%%%%%%%%%%%%%%%%%
%% ABSTRACT (first language)
%%   You can write it just here, if you want. You can also include
%%   an external .tex file (for better organization of you paper)
%%   The abstract MUST always be embedded into a command:
%%     \createabstract{
%%     This is an
%%     example abstract.
%%     }
%%   Be sure that your abstract is embedded into the curly brackets.
%%
%%   You can also put the abstract in a seperate .tex file for better
%%   organization. The command then is:
%%     \input{abstract-first}
%%   see also below (include sections)
\createabstract{The performance of four different dispatching and rebalancing algorithms for the
<<<<<<< HEAD
control of an autonomous mobility-on-demand system is evaluated in simulation.
=======
control of an Automated Mobility On-Demand system is evaluated in simulation.
>>>>>>> master
The case study conducted on an agent-based simulation scenario of the city of Zurich
shows that the right choice of control algorithm not only minimzes customer waiting
times, but also offers large economic benefits to the operator. For an average
waiting time at peak hours of five minutes the most performant algorithm would allow
<<<<<<< HEAD
the operator to offer his service for around 0.45 CHF per km, which is more expensive
than using a private car today. Yet it is significantly cheaper than a conventional
taxi. The results show that autonomous mobility-on-demand service can be offered
=======
the operator to offer his service for around 0.45 CHF per km, which is cheaper than
the average full costs of a private car and substantially cheaper than a conventional
taxi. The results show that such a automated mobility on demand services can be offered
>>>>>>> master
while maintaining a higher fleet occupancy than with private cars today. Simulation
also confirms that the application of intelligent rebalancing algorithms decreases
the average wait time in the system.

[TODO: REWORK FINANCIAL ANALYSIS AND ABSTRACT!!! -> Felix]
}
%%\input{abstract-first}
%%%%%%%%%%%%%%%%%%%%%%%%%%%%%%%%%%%%%%%%%%%%%%%%%%%%%%%%%%%%%%%%%%%%%%

%%%%%%%%%%%%%%%%%%%%%%%%%%%%%%%%%%%%%%%%%%%%%%%%%%%%%%%%%%%%%%%%%%%%%%
%% The following is necessary only if you provide an abstract in the
%%   other language (e.g. a German abstract for a paper written
%%   in English)
%%%%%%%%%%%%%%%%%%%%%%%%%%%%%%%%%%%%%%%%%%%%%%%%%%%%%%%%%%%%%%%%%%%%%%

%%%%%%%%%%%%%%%%%%%%%%%%%%%%%%%%%%%%%%%%%%%%%%%%%%%%%%%%%%%%%%%%%%%%%%
%% ABSTRACT (second language)
%% You need to define the language for the following abstract...
%\switchlanguage
%%   Same as above
%\createabstract{Zusammenfassung in der Zweitsprache}
%\input{abstract-second}
%% reset to the default language
%\switchlanguage
%%%%%%%%%%%%%%%%%%%%%%%%%%%%%%%%%%%%%%%%%%%%%%%%%%%%%%%%%%%%%%%%%%%%%%

%%%%%%%%%%%%%%%%%%%%%%%%%%%%%%%%%%%%%%%%%%%%%%%%%%%%%%%%%%%%%%%%%%%%%%
%% Sections
%%   Here you can start with your paper.
%%   The following just split up the sections into external .tex
%%   files (Section1.tex, ... Section4.tex and Ack.tex),
%%   for better organization.
%%   It is not necessary to do that, but convenient.
%%%%%%%%%%%%%%%%%%%%%%%%%%%%%%%%%%%%%%%%%%%%%%%%%%%%%%%%%%%%%%%%%%%%%%

\section{Introduction}

With growing populations in cities and increasing urbanization, many of today's transportation systems approach their performance limits. In Switzerland the distance travelled per year in private motorized vehicles rose from $82,014 ~ \textnormal{Mkm}$ in 2005 to $96'467 ~\textnormal{Mkm}$ in 2015. Also the distance traveled in trams, trolley buses and buses rose from $3,858 ~ \textnormal{Mkm}$ to $4,397 ~ \textnormal{Mkm}$ in the same time span \cite{BFSVerkehrsvolumen}. The external costs induced by road and rail traffic are rising as well, for example did the costs by noise increase by almost $20 \%$ from 2005 to 2009 in Switzerland \cite{AREexterneKosten}.

Many technologies and policies are explored in research and tested in practice to mitigate the negative effects of the increasing traffic demand, e.g. road-pricing, incentives for increased bicycle usage, improved combustion engines, dynamic traffic assignment. In this article we focus on one particularly promising approach, that promises to combine the advantages of personal mobility (i.e. flexiblity, availability, comfort and speed) with the advantages of public transport (low external and environmental costs, ability to transport large passenger volumes).

In shared-vehicle mobility on demand systems, passengers share a fleet of shared vehicles, e.g. cars or bicycles which are available on demand. In two-way mobility on demand systems vehicles are taken from the same location as they are returned to. Examples of two-way mobility-on-demand systems are ``Mobility'' in Switzerland or ``zipcar'' in the United States. While these systems are very popular, they can't be used if the origin and destination of a trip are different or rest idle for an extended amount of time between the trip to and from a destination. In two-way mobility on demand systems, origin and destination of a trip do not need to be equal. For instance in bicycle sharing schemes, users can pickup a bicycle at a set of stations distributed in the city, use it for their trip and drop it at another station. However, the asymmetric spatio-temporal behavior of travel demand causes these systems to get imbalanced, i.e. certain areas or stations get depleted of vehicles while other stations do not have enough empty spots to allow clients to return their vehicles. Bicycle sharing schemes use redistribution trucks to transport bicycles from full to empty stations and there exists research on how to perform this redistribution of bicycles in an effective manner or using price incentives, e.g. \cite{ruch2014rule}. However, the main limitation of mobility on demand schemes is their tendence to get imbalanced quickly resulting in largely decreased service level and increased cost.



\subsection{Autonomous Mobility on Demand Systems}

Autonomous two-way mobility on demand sytems (AMoD) are mobility on demand systems with vehicles that can drive autonomously, i.e. without a driver. This property allows to pickup customers at arbitrary locations in the city and imbalance the system automatically. It can be seen as the key feature to allow sharing.

AMoD systems have been studied in literature both from an operational as well as a transportation research perspective. There are studies on the effects of AMoD systems when introduced at large scale in cities, on the fleet sizing for given traffic scenarios and on the fleet and operational management of the fleets.

In this work we present four main contributions. Firstly we provide novel simulation results of unmatched accuracy for the city of Zurich, our simulations are based on the agent-based traffic simulation MATSim and include ..... XYZ Sebastian please include. Secondly we compare the performance of the fleet under different operational principles and fleet management algorithms published in literature and show that the algorithms used for dispatching and rebalancing of vehicles are key factors of performance. We further introduce metrics that allow to assess the performance of an AMoD systems in terms both service level and operational cost. Lastly we verify theoretical results on fleet sizing introduced in \cite{spieser2014toward} for the city of Zurich. In \ref{subs:literatureResearch} we present related research, then we introduce performance metrics and control strategies for AMoD systems in \ref{sec:background}. We present simulation results for the city of Zurich in \ref{sec:staticSimulations} and conclude our findings in \ref{sec:Conclusion}.


\subsection{Literature Research}
\label{subs:literatureResearch}

The work in \cite{spieser2014toward} presents a systematic approach to the design of an autonomous mobility on demand system in Singapore. The authors consider the case where the entire travel demand of the city of Singapore would be covered by autonomous shared vehicles. Analytic results are used to compute both the minimal number of vehicles needed to stabilize the number of open requests as well as the amount of vehicles that is needed to provide an acceptable level of service. The authors conclude that the travel demands of the entire population of Singapore could be served with a fleet of $300,000$ vehicles and maximum mean wait times of approximately $15~ \textnormal{mins}$ which corresponds to a sharing factor of $\approx 4$. The results of simulations confirm the theoretical approximations but the scope and granularity of the simulations is not commented and it is not clear whether traffic is adequately taken into account. In addition to the analysis on wait times the authors present an an analysis of the total cost of mobility of an AMoD system compared to the cost of owning a private vehicle. They conclude that the cost reduction per $km$ gained when switching from private car ownership to using an AMoD system is $47~ \%$, $48~ \%$ in Singapore, the United States respectively. The study does not compare different fleet operation algorithms.

In \cite{marczuk2015autonomous} a case study is presented where no private cars can enter the central business district of Singapore. A total of $25,525$ trips is served by autonomous taxis which operate either in station-based or free-floating scheme. In the station-based scheme a set of fixed stations exists where the cars return to after completion of a trip. In the free-floating scheme the cars remain parked at the destination of trips. The simulations are based on the SimMobility agent based simulation platform and include twelve different fleet sizes from $2,000$ to $7,500$ vehicles. The authors conclude that the free-floating scheme can serve $90\%$ of the demand at the maximum fleet size whereas the station-based model can only serve $68 \% $ of the demand. Furthermore they observe mean customer wait times that saturate at approximately $2.5 ~ \textnormal{mins}$ and $6,000$ vehicles. While the station-based and free-floating concepts are compared, a detailed comparison of dispatching and rebalancing operation strategies is not presented. Furthermore results on the fleet efficiency and performance are not shown as well as a more detailed analysis of the wait times (e.g. different wait time quantiles). As the CBD of Singapore attracts much more than $25,525$ trips during a day it would also be interesting to see the results with the full number of trips taking into account routing policies and congestion levels in the city.

\cite{fagnant2015dynamic} presents a case study for Austin, Texas which focuses on the use of shared autonomous vehicles with ride-sharing capabilities, i.e. vehicles that can transport more than one customer under some circumstances. The scenario presented on vehicles with unit capacity includes a fleet of $1,715$ vehicles that serve $56,324$ person-trips. Assuming $3.02$ trips per person and day this results in a sharing factor of $10.87$ at average wait times of $1.18 ~ \textnormal{mins}$ and $4.49 ~ \textnormal{mins}$ during peak hour. The authors present also a scaled version of the case including ride-sharing where $11.1\%$ of the trips within the central region of Austin (the ``Geofence'') are served by $9,037$ vehicles suggesting that a fleet of $81,414$ vehicles could serve the entire population of that area. While the authors do not compare different algorithms for fleet operation and base their fleet size estimation on simulation results, they provide a financial analysis of the entrepreneurial viability for a potential shared autonomous taxi operator.

\cite{zachariah2014uncongested} present a case study for New Jersey also focused on the potential for ride-sharing. The trips generated by a population of $8,791,894$ individuals in New Jersey are covered by walking and biking if the distance is less than a mile. All other trips are either served by the New Jersey train system, by autonomous taxis or both. The study concludes that the ride-sharing potential is large, especially during rush-hour and autonomous vehicles could significantly reduce congestion levels in the city. The required fleet size is not commented as well as the influence of the rebalancing and dispatching strategy for the fleet.

In \cite{martinez2017assessing} the authors present a study on the effects of introducting autonomous taxis and autonomous shared taxis to the city of Lisbon, Portugal. The agent-based simulation includes 1.2 million trips and three scenarios: a baseline scenario showing the current situation and two scenarios where private car, taxi and bus trips are replaced by autonomous taxis and autonomous taxis and shared taxis respectively. The fleet size of autonomous (shared) taxis is set at $4.8\%$ of the baseline vehicle fleet. In these scenarios about 50-70 \% of trips are serviced by the autonomous (shared) taxis which increases the vehicle occupancy from $50 ~ \textnormal{mins}$ to $12.87 ~ h$ on average per day. The authors conclude a decrease in cost by $55 \%$ per kilometer, highly increased transportation accessibility in the city and carbon emission reductions of almost $40\%$. The simulation does not consider the changes on traffic density parameters resulting from self-driving vehicles. Furthermore the demand choice of the agents is static and according to preset parameters. Finally the fleet control (rebalancing and dispatching) for the (shared) autonomous taxis is implemented based on heuristics and a local gradient based optimization method.

Boesch et al \cite{boesch2016autonomous} investigate a scenario of the greater Zurich region in Switzerland. They use a demand pattern for private vehicles generated with MATSim: $1.3$ million  private vehicle users out of a total of $2.1$ million agents generate $3.6$ million trips. This demand profile generated with the co-evolutionary algorithm inherent to MATSim is then post-processed in a static simulation where $1-10 \%$ of the car trips are served by $10-100 \%$ of the total number of substituted users. The authors conclude that approximately $30 ~ \%$ of the substituted fleet can serve almost $100 \%$ of the substituted requests within less than $10 ~ \textnormal{mins}$ wait time. If this threshold wait time is surpassed, then the request is dropped. The results present the first such study for the city of Zurich in Switzerland. Its limitations are that no fleet management for the autonomous vehicles is considered, i.e. no rebalancing or dispatching is taking place, furthermore network routing is not considered, travel times are based on Euclidean distance and a scaling factor. The demand profile is static and does not vary depending on service times, congestion rates and performance of the modes.

In contrast to the study for Zurich presented above, a case study for Berlin presented in \cite{bischoff2016simulation} takes into account dynamic demand. It considers a city-wide replacement of private vehicles with autonomous taxis. The dispatching of the car works according to a policy that distinguishes between oversupply (more available vehicles than open requests) and undersupply and matches the closest vehicle to an appearing request, the next available vehicle to the closest request respectively. Using this strategy named XYY in our work, the authors are able to serve $4.7$ million requests generated by $1.1$ million car users with a fleet of $100,000$ autonomous vehicles. The recorded average wait time for this case is about $2.5 \%$ minutes and the $95\%$ quantile approximately $8.5$ minutes. The resulting sharing factor is approximately $10$ to $12$. The study is one of the first large-scale dynamic simulations of a shared autonomous taxi system, however it does not consider different rebalancing and dispatching strategies and it does not rigorously evaluate the performance metrics of the autonomous vehicle fleet as we do i
XZY

\section{Related research}
\label{subs:literatureResearch}

Two-way mobility on demand systems, e.g. car sharing schemes like Mobility in
Switzerland \citep{katzev2003car} are well-established transport modes
of many cities. These schemes offer flexibility, competitive prices, and good
service levels. However, their popularity is heavily limited by the fact that
the vehicle has to be dropped off at the origin of the journey. On the contrary,
in one-way mobility on demand systems customers can travel with a vehicle (e.g.
autonomous car or bike) from any origin to any destination in the city wich
dramatically increases the flexiblity of these systems.

The price for the increased flexibilty is system imbalance. Due to the
spacio-temporal and in general unbalanced characteristics of travel demand,
vehicles tend to accumulate at certain locations and get depleted at others.
Furthermore system imbalance is not an exception but occurs for most demand
patterns. This can be seen for instance using queuing-theoretical arguments
as shwon in \citep{zhang2016control}.

System imbalance leads to drastically decreased service levels and must be
countered with the targeted repositioning of vehicles from overfull to empty
areas of the city. This repositioning of vehicles represents a substantial
contribution to the operational cost of operators and therefore various strategies
have been tried to minimize the rebalancing effort. For instance in bike-sharing
schemes, trucks are used to move vehicles from full to empty stations, in
\citep{pfrommer2014dynamic} algorithms have been proposed to route these
trucks at minimal cost. In \citep{ruch2014rule} price incentive controllers
are proposed to encourage customers to travel to depleted stations at the end
of their trip. Rebalancing was also researched for car sharing schemes, e.g.
in \citep{smith2013rebalancing} a scheme is proposed to reposition the
rebalancing drivers for one-way car sharing schemes in an optimal way. The
decisive difference of autonomous mobility on demand systems to the previous two
cases is that the vehicles can reposition themselves without the use of transporting
trucks or auxiliary drivers. Therefore rebalancing can be carried out more efficiently
and with more degrees of freedom.

Rebalancing of autonomous mobility on demand systems was first presented as a research
 problem in \citep{pavone2011load}. Optimal rebalancing flows for the vehicles are
 obtained by solving a linear program. In \citep{zhang2016control} the relation to
 queuing theoretical concepts was established. In \citep{treleaven2011asymptotically}
 the relation of the rebalancing effort to the underlying distributions of origins
 and destinations was established and it was shown that for general distributions
 the total minimal rebalancing distance is strictly more than zero.
 In   \citep{zhang2016model} the rebalancing problem was solved with a model
 predictive control algorithm which performs well but does not scale to large systems.

Most of these algorithms were tested on simplified traffic simulations that capture
the main characteristics but do not allow the same level of detail as agent based
traffic simulations like MATSim. For such simulation platforms various results
exist which are presented in the following paragraphs. Most of them do not
implement and compare the algorithms mentioned above which is an important
contribution of this work.

Spieser et al. \cite{spieser2014toward} present a systematic approach to the
design of an autonomous mobility on demand system that is able to serve the entire
travel demand of Singapore with a fleet of automated shared vehicles. Analytic
results are used to compute both the minimal number of vehicles needed to stabilize
the number of open requests as well as the amount of vehicles that is needed to
provide an acceptable level of service. The authors conclude that a fleet size
of 25\% of today's vehicle fleet would be able to offer average wait times of
around 15 minutes and could half the external and internal costs of mobility.
The study does not compare different fleet control algorithms and does not
elaborate on whether congestion effects have been taken into account.

% Took this out... I don't find it really convincing and comparable to our study
%In \cite{marczuk2015autonomous} a case study is presented where no private cars can enter the central business district of Singapore. A total of $25,525$ trips is served by autonomous taxis which operate either in station-based or free-floating scheme. In the station-based scheme a set of fixed stations exists where the cars return to after completion of a trip. In the free-floating scheme the cars remain parked at the destination of trips. The simulations are based on the SimMobility agent based simulation platform and include twelve different fleet sizes from $2,000$ to $7,500$ vehicles. The authors conclude that the free-floating scheme can serve $90\%$ of the demand at the maximum fleet size whereas the station-based model can only serve $68 \% $ of the demand. Furthermore they observe mean customer wait times that saturate at approximately $2.5 ~ \textnormal{mins}$ and $6,000$ vehicles. While the station-based and free-floating concepts are compared, a detailed comparison of dispatching and rebalancing operation strategies is not presented. Furthermore results on the fleet efficiency and performance are not shown as well as a more detailed analysis of the wait times (e.g. different wait time quantiles). As the CBD of Singapore attracts much more than $25,525$ trips during a day it would also be interesting to see the results with the full number of trips taking into account routing policies and congestion levels in the city.

% ride-sharing ... rather take the older one
Fagnant et al. \cite{fagnant2015dynamic} present a case study for Austin, Texas
which focuses on the use of shared autonomous vehicles with ride-sharing capabilities,
i.e. vehicles that can transport more than one customer under some circumstances.
The scenario presented on vehicles with unit capacity yields that 10\% of today's
vehicle fleet could serve the entire demand with average wait times of 4.49 min.

% Reidesharing
\cite{zachariah2014uncongested} present a case study for New Jersey which also focused
on the potential of ride-sharing. The trips generated by a population of $8,791,894$
individuals in New Jersey are covered by walking and biking if the distance is
less than a mile. All other trips are either served by the New Jersey train system,
by autonomous taxis or both. The study concludes that the ride-sharing potential
is large, especially during rush-hour and autonomous vehicles could significantly
reduce congestion levels in the city. The required fleet size is not commented
as well as the influence of the rebalancing and dispatching strategy for the fleet.

In \cite{martinez2017assessing} the authors present a study on the effects of
introducing autonomous taxis and autonomous shared taxis to the city of Lisbon,
Portugal. The agent-based simulation includes 1.2 million trips and three scenarios:
a baseline scenario showing the current situation and two scenarios where private car,
taxi and bus trips are replaced by autonomous taxis and autonomous taxis and shared
taxis respectively. The fleet size of autonomous (shared) taxis is set at $4.8\%$ of
the baseline vehicle fleet. In these scenarios about 50-70 \% of trips are serviced
by the autonomous (shared) taxis which increases the vehicle occupancy
from $50 ~ \textnormal{mins}$ to $12.87 ~ h$ on average per day. The authors conclude
a decrease in cost by $55 \%$, highly increased transportation
accessibility in the city and carbon emission reductions of almost $40\%$. The
simulation does not consider the changes on traffic density parameters resulting
from self-driving vehicles. Furthermore the demand choice of the agents is static
and according to preset parameters. Finally the fleet control (rebalancing and dispatching)
for the (shared) autonomous taxis is implemented based on heuristics and a local
gradient based optimization method.

Boesch et al \cite{boesch2016autonomous} investigate a scenario of the greater
Zurich region in Switzerland. They use a demand pattern for private vehicles
generated with MATSim: $1.3$ million  private vehicle users out of a total of
$2.1$ million agents generate $3.6$ million trips. This demand profile generated
with the co-evolutionary algorithm inherent to MATSim is then post-processed
in a static simulation where $1-10 \%$ of the car trips are served by $10-100 \%$
of the total number of substituted users. The authors conclude that approximately
$30 ~ \%$ of the substituted fleet can serve almost $100 \%$ of the substituted
requests within less than $10 ~ \textnormal{mins}$ wait time. If this threshold
wait time is surpassed, then the request is dropped. The limitations of the results
are that no rebalancing or dispatching is taking place, furthermore network routing
 is not considered, travel times are based on Euclidean distance and a scaling factor.
  The demand profile is static and does not vary depending on service times,
  congestion rates and performance of the modes.

In contrast to the study for Zurich presented above, a case study for Berlin
presented in \cite{bischoff2016simulation} takes into account dynamic demand.
It considers a city-wide replacement of private vehicles with autonomous taxis.
 The dispatching of the car works according to a policy that distinguishes
 between oversupply (more available vehicles than open requests) and undersupply
 and matches the closest vehicle to an appearing request, the next available
 vehicle to the closest request respectively. Using this strategy called single
 heuristic dispatcher in our work, the authors are able to serve $4.7$ million
 requests generated by $1.1$ million car users with a fleet of $100,000$ autonomous
 vehicles. The recorded average wait time for this case is about $2.5 \%$ minutes
 and the $95\%$ quantile approximately $8.5$ minutes. The resulting sharing factor
 is approximately $10$ to $12$. The study is one of the first large-scale dynamic
 simulations of a shared autonomous taxi system, however it does not consider
 different rebalancing and dispatching strategies and it does not rigorously
 evaluate the performance metrics of the autonomous vehicle fleet as we do in [XYZ].


%======================== END of new part ============================================

%Let's go through this again. We need:

%\begin{itemize}
%\item Literature on rebalancing in general (Claudio bike rebalancing, but there is a lot for car-sharing available), just indicate importance not go into detal on algorithms
%\item Simulation studies on AVs (we have that already! commented out right now), start with the conceptual ones (Spieser) and mention we take algos from there
%\item Literature MATSim
%\end{itemize}

%It goes as follows: First literature on rebalancing is introduced, then the simulation studies on AVs
%are presented with the remark that they do NOT consider rebalancing. We end with Kockelman and Bösch,
%who use MATSim but just as a preparational step and lead to full MATSIm simulations, first Michal & Joschka,
%then ABMTRANS, which is used here.

%This way we have introduced everything we're using here.



%While

\section{Control of an AMoD System}
\label{sec:background}

An AMoD service is can only be realized if it is attractive to customers. More
specifically, it can only be maintained if a sufficient number of customers
wants to use the service and the service is profitable for the operator.

While a multitude of factors influence the attractiveness of the service (perhaps
multimedia offers in the vehicle, the quality of Wifi, ...) the authors assume
two key properties: The time that passes between a customer making a request
and a vehicle arriving (i.e. the wait time) and the price that is charged to
the customer. All else being equal, an operator that can offer the shortest wait
times at the lowest price will attract more customers than his competitors. For
now, it remains unknown how those two factors would be valued against each other
by potential customers.

We focus on two main ways for operators to influence the service level of their system:

\begin{itemize}
\item The \textbf{fleet size} can be increased. In general, this should lead to
a decrease of wait time, because the availability of vehicles improves. However,
having a larger number of vehicles imposes more fixed costs that would need to be
balanced by higher demand. In general, adding more vehicles to the fleet can be
regarded as a long-term investment that cannot be altered on a daily basis.
\item The \textbf{fleet control} can be optimized. Since in an AMoD system it is
asssumed that any vehicle can be tracked and controlled online, intelligent fleet
control algorithms can be used to minimize the wait times, but also minimize the
driven distance in order to reduce operational cost. Applying the proper algorithm is a much
less costly intervention than increasing the fleet size with assumably smaller
effects, but may bring a competitive advantage on the market.
\end{itemize}

In the presented experiments both components are investigated by comparing a number
of control algorithms for fleets of varying sizes.

\subsection{Problem Statement}

For the algorithmic improvement of the fleet management the authors distinguish
between two stages:

\begin{itemize}
\item The \textbf{dispatching strategy} decides how to serve the demand, i.e.
how to match the open customer requests, with the available vehicles. At any time the dispatcher can send tasks to pickup a specific customer to any vehicle that is not currently having a customer onboard (since we do not consider ride-sharing with multiple customers). Also a reassignment of a previously assigned
vehicle to another request is possible.
\item The \textbf{rebalancing strategy} decides where to send vehicles when they
are not in use and the demand allows for supplementary movements of the vehicles.
The task of the rebalancer is to anticipate future requests and position vehicles
such that they are able to optimally react to the upcoming demand.
\end{itemize}

Hence, vehicles will produce three kinds of mileage:

\begin{itemize}
\item \textbf{Empty pickup mileage} is produced when an AV is dispatched
to a request and is driving to the pick-up location. It is the mileage that needs
to be covered in order to serve the customer in any way and may be minimized
by an intelligent dispatching algorithm.
\item \textbf{Empty rebalancing mileage} is produced when an AV is sent
to a different location where demand is expected. An ideal operator would
exchange all the pickup mileage in the system against rebalancing mileage, i.e. the operator would always send empty vehicles before an actual request turns up.
\item \textbf{Customer mileage} is produced with a customer on-board. This mileage does only depend on the routing of the cars. In any combination of fleet size and control algorithm, this mileage stays constant, because it is defined by the origin-destinaton relations of all customer trips.
\end{itemize}

Assuming a common pricing scheme that defines a price per distance, the customer mileage
is the only component that produces a benefit for the operator. All other mileage
can directly be translated into costs and should therefore be minimized. For general
demand patterns, however, it cannot be driven to zero. Treleaven et al. \cite{treleaven2011asymptotically}
show that it is bounded below by the earth mover's distance, which is a measure
of how different the distributions of trip origins and destinations are (see \cite{ruschendorf1985wasserstein}).

The objectives for a fleet management algorithm can therefore be defined as:

\begin{enumerate}
\item Minimize the total pickup distance given the non-optimal locations of the vehicles (dispatcher)
\item Exchange as much pickup distance as possible for rebalancing distance (rebalancer)
\end{enumerate}

\subsection{Selected Algorithms}

In this work we analyze four different operating strategies from literature, which are briefly outlined below:

\begin{enumerate}
\item The single heuristic dispatcher is a strategy presented in \cite{bischoff2016simulation}. In every dispatching time step $\delta t_D$ If there are more available vehicles than requests, it iterates on the list of requests and assigns to each request the closest vehicle. If there are more open requests than available vehicles, the controller iterates on the available vehicles and assigns the closest open request to each vehicle. The assignments are binding, i.e. they are not reopened once concluded.
\item The global Euclidean bipartite matching dispatcher determines an optimal bipartite matching between all open requests and available vehicles in every dispatching time step $\delta t_D$. The used distance function is the Euclidean distance which allows to use fast algorithms, e.g. \cite{agarwal2004near}. In contrast to the previous strategy, the assignments can be changed until a vehicle actually reaches its target. For a given set of open requests and available vehicles, this strategy can be considered as the optimal dispatching strategy based on Euclidean distances.
\item In \cite{pavone2011load} a feedforward strategy is presented on how to rebalance vehicles between different vertices in a directed graph $G = (V,E)$. For each vertex $i$ and time step $\delta_t$, the arrival rates $\lambda_i$ and transition probabilities $p_{ij}$ for any nodes $v_i, v_j \in V$  are computed from historical data. The linear program in equation \ref{eq:linearprogram} computes the optimal rebalancing flows $\alpha _{ij}$ for an equilibrium point of the underlying fludidic model with travel times $T_{i,j} \forall v_i, v_j \in V$.  
\begin{align}
&\textnormal{minimize}& &\sum_{i,j} T_{i,j} \alpha _{ij} & && && \nonumber \\
&\textnormal{subject to}& &\sum_{i \neq j} \alpha _{ij} - \alpha _{ji} =-\lambda_i  + \sum_{i \neq j} \lambda_j p_{ji} & &\forall v__i \in V& \label{eq:linearprogram} \\
&& &\alpha_{ij} \geq 0& & \forall v__i, v_j \in V& \nonumber
\end{align}
To implement this strategy, we divided the city of Zurich into a set of areas. The nodes from \cite{pavone2011load} represent the centroids of these areas on which a complete directed graph called virtual network is placed, see figure \ref{fig:study_area_vnodes}. Available cars are continuously rebalanced between the vertices of the virtual network according to the static rebalancing rates $\alpha_{ij}$. As the work does not detail the proposed dispatching algorithm for this strategy, we match cares using global Euclidean bipartite matching. Rebalancing vehicles cannot be dispatched until they reach their destination virtual node.
\item The last implemented strategy is as well derived from \cite{pavone2011load}. Instead of a pure feedforward solution, here in every rebalancing timestep $\delta t_R$ for every area of the virtual network the avaialble cars and open requests are counted and fed into an integer linear program derived from equation \ref{eq:linearprogram} calculating the number of cars $reb _{ij}$ to be sent from virtual vertex $i$ to virtual vertex $j$. As in the feedforward strategy, the matching of the cars is done via global Euclidean bipartite matching.
\end{enumerate}
\section{Simulation Setup}

In order to assess the performance of the different fleet sizes and control
algorithms a novel scenario for the city of Zurich, Switzerland is set up
for the MATSim transport simulation framework and a theoretical fleet sizing
according to \citep{spieser2014toward} is performed.

% Commented to reduce works /sh
%The section is structured as follows:
%First, we give an overview about the used simulation components, second, we specify
%the scenario and finally, we provide fleet sizing results from the theoretical
%methodology presented in \citep{spieser2014toward}.

\subsection{MATSim and AMoD Simulation}

MATSim \citep{Horni2015} is an agent-based transport simulation framework that makes it possible
to simulate large numbers of agents representing a real population in a traffic environment. Similar to reality, each agent has a daily plan with activities
inteded to be performend for a certain duration and to be finished at a specific time of
the day. Since these activities take place at different locations in the scenario,
agents need to move from activity to activity. By default, MATSim allows the
simulation of car traffic, public transit and slow modes such as going by bike
or walking. Road-based modes, such as private cars are simulated in a time-step
based manner in a network of queues with all participants at the same time. This
way it is possible that congestion emerges and agents arrive late at their
activity locations. While MATSim provides more functionality, e.g. the replanning
of agents plans to adapt to the traffic conditions that they perceive, only the
network simulation is used in this research.

An extension developed in \cite{horl_abmtrans17} is used to add automated taxis to the set
of available travel modes. A virtual dispatcher, for which
different algorithms are used in this study, controls a fleet of AVs.
Whenever an agent wants to depart from his current activity location by
AV, a request is issued to the dispatcher and saved.  The choice which vehicle to send and when is completely defined
by the dispatching algorithm. Once the vehicle arrives at the customer's location,
the pickup is processed, the AV drives to the destination and finally drops off the customer. Then,
the vehicle is available for dispatching again. Alternatively, vehicles can be
rebalanced, which means that the dispatcher gives an AV the instruction
to drive to a different location. All of this is performed in the MATSim traffic
simulation such that AVs suffer from congestion as any other vehicle.

% Commented out to save words
%It should be noted that AVs drive directly to the locations where agents finish and
%start their activities. So far no mechanism is implemented that would allow them
%to meet at optimized locations, e.g. a high-capacity avenue instead of a small
%alley.

\subsection{Scenario Definition}

For Switzerland the Microcensus on mobility and transport \cite{microcensus} is
available, which reports the daily travel patterns of 60,000 survey respondents
resident in the country.
It is the basis for a readily available agent population of
Switzerland, which reproduces the demographic attributes and travel patterns
in the country to great detail \cite{ivtbaseline}.

\begin{figure}[h]
\begin{center}\includegraphics[width=1.0\textwidth]{figures/map.pdf}\end{center}
\caption{The AMoD service area covering the 12 districts of Zurich and the nodes of the
virtual network for the rebalancing algorithms. (Map: OpenStreetMap)}
\label{fig:study_area_vnodes}
\end{figure}

Additional modifications are applied to this population of around 8 million
agents to make it suitable for the study at hand. First, a best-response routing
of the trips of all agents is performed to find all agents that interact
with the AMoD service area, which has been defined to be the 12 districts of Zurich (Figure \ref{fig:study_area_vnodes}).
All agents which do not interact with that region (i.e. do not perform an activity within
the area and do not cross the area) are deleted from the population as they do
not contribute congestion in the area. Finally, a 1\%
sample of the remaining agents is created. The rather extensive downscaling becomes necessary for the computationally
demanding algorithms, given that they need to be performed hundreds of times faster
than reality to allow for multiple runs and iterations.

%In order to define the travel demand for the fleet of automated vehicles, agents
%are tagged as whether they are viable for using an automated vehicle or not. Pedestrians
%and cyclists are not simulated at all in this work since they do not contribute to congestion in the current version of the framework.

An agent that travels at least once by private car during the simulation is tagged
as an AV user \textit{only} if all of the legs in the agent's plan take place
within the AMoD service area. This constraint makes sure that no unrealistic travel
plans are generated, where an agent performs his first leg by AV although his
private car is at home and then wants to depart at the next location with that
car. Finally, the ``car'' legs of all viable agents are converted to the ``av'' mode.
All other legs are kept as before, i.e. short legs that are assigned the ``walk''
mode initially are still performed with this mode. 

For agents that use public transit, the procedure is different. Here, any leg
that is performed by the ``pt'' mode in the original population is converted to ``av''
if it lies within the AMoD service area. As for car users, connecting non-motorized
legs are kept fixed. Proceeding as outlined, a demand for Zurich is generated in which each leg that possibly
\textit{can} be performed using an AV \textit{is} performed by AV. To summarize, all agents whose daily trips are entirely in the service area switch to using the AMoD system. Furthermore all public transit trips within the service area are now served by the new system. The remaining trips are unchanged.

In conclusion, the 8,230,971 agents in the population are reduced to
1,935,400 agents, which touch the AMoD service area. From this set of agents
a 1\% sample  is drawn, leading to 13,141 agents that mainly constitute
background traffic for congestion. Among those are 970 agents are viable for the AV
service. The plans of these agents contain 2,096 trips that are to be served by
AVs. In reality, this scaled service would hence need to serve 209,600 requests by
97,000 persons. 

\subsection{Theoretical Fleet Sizing}

% Commented out to save words. I thin kwe explain this further up already. /sh
%Both the capital cost of an AMoD system and the service rates are highly dependent
%on the fleet size which makes fleet sizing an important aspect of AMoD system design.
%If the fleet size is chosen too small, then the service levels will be unacceptable. If the fleet size is chosen too large, the cost of the system becomes unbearable due to low utilization rates.

Fleet sizes can be estimated using simulations, as for instance done in
\citep{bischoff2016simulation}. Despite the accuracy of these
simulation results, they do not provide insights into the fundamental
properties influencing the relationship between fleet size and performance metrics.

For this reason we implement theoretical results from \citep{spieser2014toward}
for the case of Zurich. The authors present two methods for fleet size evaluation.
The first method estimates the theoretical minimum fleet size to stabilize
the system, i.e. to ensure that the number of open requests stays bounded at
all times. To do so, for every vertex $i$ and timestep $\delta_t$ the added
unserved mileage per timestep is calculated as
$\lambda_i \cdot ( \bar{d}_{OD,i}  + \bar{d}_{EMD,i})$ where $\bar{d}_{OD,i}$
is the average distance per trip and  $\bar{d}_{EMD,i}$ the mover's
distance per vehicle in the timeslice. $\bar{d}_{OD,i}  + \bar{d}_{EMD,i}$
represents the average distance that has to be driven per request. A total of
$m$ vehicles at an average speed of $v$ are collectively able to reduce this
 added mileage at a rate of $m \cdot v$. This quantity has to be larger than the
 added unserved mileage per timestep. For the scenario here the
 minimum fleet size computed with this measure are $1380$ vehicles.

While the knowledge of the minimum fleet size is useful, it does not reveal
the relation between service level and fleet size, especially to what number
the fleet size has to be augmented before further addition of vehicles will
not result in a significant decreaes in wait time. In \citep{zhang2016control}
 a method is presented of how an AMoD system can be cast in a Jackson network.
 For such networks, queuing theoretical results allow for the computation of
 performance measures such as vehicle wait times, queue lengths or
 availabilities at vertices. The quantity of interest is the availability
 of a vehicle at a vertex, which is the probability that at least one idle
 vehicle is at that vertex. Computation of the mean availability of all
 timesteps and vertices as a function of the fleet size for Zurich results
 in the curve shown in Figure \ref{fig:performanceavailability}. Note that these results
 are purely theoretical and can be derived solely from input data without performing simulations.
 Therefore they can serve as a measure of accuracy for the simulation results.

%\input{sections/05_Simulations}
\section{Results}
\label{sec:results}

We test the four proposed dispatching strategies in the Zurich scenario with ten
runs per fleet size and strategy.% Commented out to reduce works / sh
%Since the dispatchers rely on free flow speeds in the network
%for their routing when the simulation starts, we let each run perform 20 iterations
%in which the dispatcher step by step senses the traffic conditions, e.g. how to
%avoid traffic jams at peak hours.
The dispatching stages of all algorithms are called once every 60 seconds in
simulated time, while the rebalancing periods for the feedforward and feedback
dispatcher are five minutes and 20 minutes, respectively. Those values have been
obtained from prior simulation runs.

For Zurich, the times with peak congestion and, hence, longest travel times
are from 6:30pm to 9:00am and from 4:30pm to 6:30pm. Figure \ref{fig:mean_peak_waiting_times}
shows the average customer wait time over the whole day and just for peak hours.
While the simple heuristic approach consistently yields the longest wait times
for any fleet size, the feedback dispatcher performs best. The bipartite matching
performs in between, since it is based on an optimal request assignment, but does
not do any rebalancing. %Surprisingly, the feedforward dispatcher performs similar
%to the bipartite matching algorithm. This means that the expected demand that
%has been generated from the scenario data does not optimally predict the actual
%travel patterns in the simulation.
[TODO: waiting for final simulations for FF, they will look better! /sh]

Assuming that 5 minutes at peak times are an acceptable wait time, that delay is
achieved with a fleet of 10,000 vehicles for the heuristic, but with only 8,700
for the feedback dispatcher.

\captionsetup[subfigure]{width=0.9\textwidth}

\begin{figure}
    \centering
    \begin{subfigure}[t]{0.495\textwidth}
        \includegraphics[width=1.0\textwidth]{figures/mean_peak_waiting_times.pdf}
        \caption{Average waiting time for an AV to arrive at peak times (solid) and over the entire day (dashed)}
        \label{fig:mean_peak_waiting_times}
    \end{subfigure}\hfill
    \begin{subfigure}[t]{0.495\textwidth}
        \includegraphics[width=1.0\textwidth]{figures/availability.pdf}
        \caption{Probability of at least one AV being available whenever a request comes in
        [TODO: Passt das scaling so? /sh]}
        \label{fig:performanceavailability}
    \end{subfigure}
    \caption{Fleet performance metrics for different fleet sizes}
\end{figure}







%\begin{figure}
%\includegraphics[width=1.0\textwidth]{figures/mean_peak_waiting_times.pdf}
%\caption{Average waiting time for an AV to arrive at peak times (solid) and over the entire day (dashed)}
%\label{fig:mean_peak_waiting_times}
%\end{figure}

Figure \ref{fig:distances} shows the distances that different service configurations
produce. On the left side the customer distance is shown, which stays constant
over all runs, while one can see that the pickup distance (middle) is decreasing
with larger fleet sizes and thus higher availability of vehicles. For the dispatchers
with rebalancing once can see that they exchange pickup mileage against rebalancing
mileage (on the right), which makes it possible for them to offer shorter wait
times. This comes with the price of a surplus in overall distance for which the
operators needs to cover the costs. [TODO: Let's see the FF, maybe add a comment
on that /sh]

\begin{figure}
\includegraphics[width=1.0\textwidth]{figures/distances.pdf}
\caption{Distribution of distances for different fleet sizes. From left to right:
Customer distance, empty pickup distance, empty rebalancing distance.}
\label{fig:distances}
\end{figure}

Finally, the occupancy of the fleet is measured. For a fleet size of 6,000
vehicles, they are busy serving a passenger for around 4.8h per day, while
this value drops to 2.16h for the maximum fleet size of 180,000. In both cases,
those numbers exceed the average 1.92h [TODO: Do we have an exact value for Switzerland? /sh] of today's vehicle fleet.

\subsection{Financial Analysis}
\label{sec:cost_analysis}

Based on the cost calculator for fleets of automated vehicles by Bösch et al. \cite{Bosch2016a}
 the costs of operating the AV services are computed from a number of key figures
 such as the occupancy, share of empty rides, among others. In Figure~\ref{fig:passenger_price}
 the resulting price is shown that the operators needs to ask his customers per
 kilometer if a profit margin of at least 3\% is targeted. Unsurprisingly, the value
  increases with larger fleet sizes, but a clear difference
between the algorithms can be observed. The simple heuristic is the most
costly operating scheme, while the feedback dispatcher can be operated with the
lowest passenger prices.


\captionsetup[subfigure]{width=0.9\textwidth}

\begin{figure}
    \centering
    \begin{subfigure}[t]{0.495\textwidth}
        \includegraphics[width=1.0\textwidth]{figures/01_passenger_price.pdf}
        \caption{Minimum customer prices that an AV operator needs to charge with a profit margin of at least 3\%}
        \label{fig:passenger_price}
    \end{subfigure}\hfill
    \begin{subfigure}[t]{0.495\textwidth}
        \includegraphics[width=1.0\textwidth]{figures/time_vs_price.pdf}
        \caption{Comparison plot of offered wait times and minimum service prices for the
        simulated fleet configurations}
        \label{fig:time_vs_price}
    \end{subfigure}
    \caption{Analysis of fleet configurations from the customer perspective}
\end{figure}

Compared to the price of a taxi operator in Zurich (base price 8 CHF plus 5 CHF/km, \cite{StadtZurich2014})
the computed prices are extremely low. Hence, an automated service would clearly
push conventional taxi operators out of the market. The variable costs of a today's private vehicle (0.26 CHF/km, \cite{TCS2016}) are lower than the calculated prices for the AV-services, independent of the algorithm. Considering the full costs of a private vehicle which amount to 0.7 CHF/km \cite{TCS2016} however, it can be concluded that AV-services are only more expensive for large fleet sizes. Nonetheless, compared to
(subsidized) prices for mass transit (0.25 CHF/passenger kilometer, \cite{Bosch2016a}), the services are more
expensive.

Therefore, the proposed AV services are highly attractive to car users, but may
not be able to compete with subsidized mass transit. On the other hand, AVs
allow for more direct trips and thus for savings in travel time. Further studies
may analyse how these affect the attractiveness of the AMoD services.

Looking at different choice alternatives for customers, two components are important
that are weighed against each other: The expected wait time and the price. Figure
\ref{fig:time_vs_price} combines the key results from our simulations. There,
the price that a specific operator configuration (fleet size and dispatcher)
needs to charge is
displayed in comparison to the wait time that this operator is offering.
At a wait time of five minutes an operator would be able to offer a satisfactory service
for around 0.45 CHF with the feedback dispatcher, while he would need to charge
0.50 CHF with the simple load-balancing heuristic.

%The better the level of service of the operator is, the larger this margin becomes.

\section{Conclusion \& Outlook}
\label{sec:Conclusion}

The study shows that the right choice of dispatching algorithm for an AMoD system
does not only have strong impact on the performance in terms of wait time for
the customer, but also that it bears a substantial economic advantage for the
operator. Operators with intelligent redispatching and rebalancing algorithms are able to attract
more customers through quicker pickups and lower prices than a competitor at small additional cost.

In order to assess the significance for real fleets of (not necessarily
automated) taxis it needs to be noted that all of the presented algorithms are
able to process dispatching and rebalancing tasks for fleets of thousands of
vehicles within minutes. It is perfectly feasible to control 100k vehicles in
five minute updates using a standard laptop for the computational tasks.

For the presented simulations, this still poses a burden, because there
a speedup compared to reality of hundreds of times is desired to be able
to run large numbers of simulations. Hence, the algorithms
are only tested on a subsample of 1\% of the agent population that is available.
In future studies, effort will be put into overcoming this restriction, either
by finding approximate formulations for the presented algorithms or pursuing research
on completely new algorithms.

Throughout the paper, a ``100\%'' demand scenario is used, in which all
trips that possibly could be undertaken by AVs are converted to the automated
mode. MATSim offers the possibility to explicitly
simulate attitudes toward new elements in the traffic system by defining utilities
for using specific modes with distinct valuation of travel costs, travel times and
distances. This way, by integrating the presented algorithms into the full
MATSim loop, as shown in \cite{horl_abmtrans17}, the actual attractiveness of an
AV service can be analysed including the tradeoff that people make between
paying for the service, spending time in the vehicle and having to wait for it.
Naturally, not 100\% of possible trips would be performed by AV then, but only
a fraction. Future work will take these considerations into account.

[TODO: Referenzen muessen noch aufgeraeumt werden. /sh]


%\input{Background}
%\input{ScoringMATSim}
%\input{ProposedScoringFunction}
%\input{DefaultProposedUtilityFunction}
%\input{ReschedulingResults}
%\input{Discussion}
%\input{Conclusion}
%%%%%%%%%%%%%%%%%%%%%%%%%%%%%%%%%%%%%%%%%%%%%%%%%%%%%%%%%%%%%%%%%%%%%%
%% Bibliography
%%   Leave this as is, and add you own entries to my.bib
%%   Many references are already defined in _latexfiles/bibs/all-eng.bib
%%   Refer to the BibTeX/LaTeX tutorial for adding new entries
%%   to the IVT BibTeX database
\bibliography{\mypath/bibs/all-eng,my}
%%%%%%%%%%%%%%%%%%%%%%%%%%%%%%%%%%%%%%%%%%%%%%%%%%%%%%%%%%%%%%%%%%%%%%

%%%%%%%%%%%%%%%%%%%%%%%%%%%%%%%%%%%%%%%%%%%%%%%%%%%%%%%%%%%%%%%%%%%%%%
%% Appendices
%%   Usually they would start on a separate page
%%%%%%%%%%%%%%%%%%%%%%%%%%%%%%%%%%%%%%%%%%%%%%%%%%%%%%%%%%%%%%%%%%%%%%


\end{document}

%%%%%%%%%%%%%%%%%%%%%%%%%%%%%%%%%%%%%%%%%%%%%%%%%%%%%%%%%%%%%%%%%%%%%%
%%%%%%%%%%%%%%%%%%%%%%%%%%%%%%%%%%%%%%%%%%%%%%%%%%%%%%%%%%%%%%%%%%%%%%
%%
%% END OF DOCUMENT
%%
%%%%%%%%%%%%%%%%%%%%%%%%%%%%%%%%%%%%%%%%%%%%%%%%%%%%%%%%%%%%%%%%%%%%%%
%%%%%%%%%%%%%%%%%%%%%%%%%%%%%%%%%%%%%%%%%%%%%%%%%%%%%%%%%%%%%%%%%%%%%%

%%%%%%%%%%%%%%%%%%%%%%%%%%%%%%%%%%%%%%%%%%%%%%%%%%%%%%%%%%%%%%%%%%%%%%
%% Editor specific keywords:
%%   This is not part of you paper, but sometimes it is used
%%   for additional features of TeX Editors.
%%
%% WinEdt:
%%   to get the bibliography list
%GATHER{./_bibs/all-eng.bib}
