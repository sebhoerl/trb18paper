% !TeX encoding = usascii
% Prepare to get rid of \mypath one day
\providecommand\mypath[3]{}
%%%%%%%%%%%%%%%%%%%%%%%%%%%%%%%%%%%%%%%%%%%%%%%%%%%%%%%%%%%%%%%%%%%%%%
%% $Id$
%%%%%%%%%%%%%%%%%%%%%%%%%%%%%%%%%%%%%%%%%%%%%%%%%%%%%%%%%%%%%%%%%%%%%%

%%%%%%%%%%%%%%%%%%%%%%%%%%%%%%%%%%%%%%%%%%%%%%%%%%%%%%%%%%%%%%%%%%%%%%
%%
%% TRR PREVIEW PAPER LAYOUT
%% Date: 2014-02-26
%% author:
%%   Kirill M\"uller, kirill.mueller@ivt.baug.ethz.ch
%%
%%%%%%%%%%%%%%%%%%%%%%%%%%%%%%%%%%%%%%%%%%%%%%%%%%%%%%%%%%%%%%%%%%%%%%

%%%%%%%%%%%%%%%%%%%%%%%%%%%%%%%%%%%%%%%%%%%%%%%%%%%%%%%%%%%%%%%%%%%%%%
%%%%%%%%%%%%%%%%%%%%%%%%%%%%%%%%%%%%%%%%%%%%%%%%%%%%%%%%%%%%%%%%%%%%%%
%%
%% Standard latex layout configurations
%%   Here, commands and settings are use which are available
%%   by the latex packages
%%
%%%%%%%%%%%%%%%%%%%%%%%%%%%%%%%%%%%%%%%%%%%%%%%%%%%%%%%%%%%%%%%%%%%%%%
%%%%%%%%%%%%%%%%%%%%%%%%%%%%%%%%%%%%%%%%%%%%%%%%%%%%%%%%%%%%%%%%%%%%%%
%% Type of document:
%% - Paperformat: letterpaper, a4paper, a5paper, b5paper,
%%   executivepaper, legalpaper
%% - Main font size: 10pt, 11pt, 12pt
%% - Formulae setting: - (centred), fleqn (left-aligned)
%% - Numbering of formulae: - (right-aligned), leqno (left-aligned)
%% - New page after title: titlepage, notitlepage
%% - Number of columns per page: onecolumn, twocolumn
%% - Page style: oneside, twoside
%% - Paper rotation: - (protrait), landscape
%% - Chapter start: openright, openany (not required for oneside layout)
%% - Mark overfull boxes: draft, final
\documentclass[9pt,fleqn,twocolumn,twoside,final]{extarticle}
%%%%%%%%%%%%%%%%%%%%%%%%%%%%%%%%%%%%%%%%%%%%%%%%%%%%%%%%%%%%%%%%%%%%%%
%% borders, margins and offset
\usepackage[letterpaper,left=0.75in,right=0.75in,top=0.8in,bottom=0.85in,columnsep=0.5in]{geometry}
%%%%%%%%%%%%%%%%%%%%%%%%%%%%%%%%%%%%%%%%%%%%%%%%%%%%%%%%%%%%%%%%%%%%%%
%% Package options
\PassOptionsToPackage{nooneline,font=bf,labelsep=quad,format=hang}{caption}
\PassOptionsToPackage{font=small,labelsep=space}{subcaption}
\PassOptionsToPackage{fleqn}{amsmath}
\PassOptionsToPackage{compress}{natbib}
\PassOptionsToPackage{noabbrev}{cleveref}
\PassOptionsToPackage{dvipsnames}{xcolor}
%%%%%%%%%%%%%%%%%%%%%%%%%%%%%%%%%%%%%%%%%%%%%%%%%%%%%%%%%%%%%%%%%%%%%%
%% Include default packages and settings
%%%%%%%%%%%%%%%%%%%%%%%%%%%%%%%%%%%%%%%%%%%%%%%%%%%%%%%%%%%%%%%%%%%%%%
%% Bugfixes:
\ifdefined\ivtnofixltx\else\usepackage{fixltx2e}\fi
%%%%%%%%%%%%%%%%%%%%%%%%%%%%%%%%%%%%%%%%%%%%%%%%%%%%%%%%%%%%%%%%%%%%%%
%% Smart space after macro:
\usepackage{xspace}
%%%%%%%%%%%%%%%%%%%%%%%%%%%%%%%%%%%%%%%%%%%%%%%%%%%%%%%%%%%%%%%%%%%%%%
%% string manipulation
\usepackage{stringstrings}
\usepackage{textcase}
%%%%%%%%%%%%%%%%%%%%%%%%%%%%%%%%%%%%%%%%%%%%%%%%%%%%%%%%%%%%%%%%%%%%%%
%% providing if-then-else command:
\usepackage{ifthen}
%%%%%%%%%%%%%%%%%%%%%%%%%%%%%%%%%%%%%%%%%%%%%%%%%%%%%%%%%%%%%%%%%%%%%%
%% providing expandonce command (and others):
\usepackage{etoolbox}
%%%%%%%%%%%%%%%%%%%%%%%%%%%%%%%%%%%%%%%%%%%%%%%%%%%%%%%%%%%%%%%%%%%%%%
%% LaTeX3 document parsing:
\usepackage{xparse}
%%%%%%%%%%%%%%%%%%%%%%%%%%%%%%%%%%%%%%%%%%%%%%%%%%%%%%%%%%%%%%%%%%%%%%
%% helper commands
\newcommand{\ifeqe}[4]{%
  \ifthenelse{\equal{\expandonce{#1}}{\expandonce{#2}}}{#3}{#4}%
}
\newcommand{\ifeq}[3]{%
  \ifeqe{#1}{#2}{#3}{}%
}
\newcommand{\ifneq}[3]{%
  \ifeqe{#1}{#2}{}{#3}%
}
\newcommand{\ifne}[2]{%
  \ifneq{#1}{}{#2}%
}
%%%%%%%%%%%%%%%%%%%%%%%%%%%%%%%%%%%%%%%%%%%%%%%%%%%%%%%%%%%%%%%%%%%%%%
%% number of words used (wordcount)
%% - include number
\AtBeginDocument{%
\providecommand{\mywordcount}{}%
\renewcommand{\mywordcount}{%
  \newcounter{mywordcount}%
  \IfFileExists{mywordcount}{%
    \newcommand{\mytextwordcount}{6453}%
%
    \setcounter{mywordcount}{\mytextwordcount}%
  }{%
    \newcommand{\mytextwordcount}{Use the script wordcount.py to count}%
  }%
  \newcounter{myequivalentcount}%
  \setcounter{myequivalentcount}{\arabic{mywordcount} + \totvalue{figure} * 250 + \totvalue{table} * 250}%
  \ifthenelse{\equal{\totvalue{table}}{0}}{%
    \newcommand{\mytableid}{}%
  }{%
    \ifthenelse{\equal{\totvalue{table}}{1}}{%
      \newcommand{\mytableid}{ + 1 table}%
    }{%
      \newcommand{\mytableid}{ + \total{table}\xspace tables}%
    }%
  }%
  \ifthenelse{\equal{\totvalue{figure}}{0}}{%
    \newcommand{\myfigureid}{}%
  }{%
    \ifthenelse{\equal{\totvalue{figure}}{1}}{%
      \newcommand{\myfigureid}{ + 1 figure}%
    }{%
      \newcommand{\myfigureid}{ + \total{figure}\xspace figures}%
    }%
  }%
  \ifthenelse{\equal{\arabic{mywordcount}}{\arabic{myequivalentcount}}}{%
    \newcommand{\myeqtext}{}
  }{%
    \newcommand{\myeqtext}{ = \arabic{myequivalentcount} word equivalents}
  }%
  \mytextwordcount\xspace words\myfigureid\mytableid\myeqtext%
}%
}
\newcommand{\ifelsewc}[2]{%
  \ifdefined\wcFileName%
    #1%
  \else%
    #2%
  \fi%
}
\newcommand{\ifwc}[1]{%
  \ifelsewc{#1}{}%
}
\newcommand{\ifnwc}[1]{%
  \ifelsewc{}{#1}%
}
%%%%%%%%%%%%%%%%%%%%%%%%%%%%%%%%%%%%%%%%%%%%%%%%%%%%%%%%%%%%%%%%%%%%%%
%% Support for conversion to Word
\providecommand{\AsPicture}[1]{#1}
\makeatletter
\@ifpackageloaded{tex4ht}{%
  \newcommand{\ifelseht}[2]{#1}%
}{%
  \newcommand{\ifelseht}[2]{#2}%
}%
\makeatother
\newcommand{\ifht}[1]{%
  \ifelseht{#1}{}%
}
\newcommand{\ifnht}[1]{%
  \ifelseht{}{#1}%
}
%%%%%%%%%%%%%%%%%%%%%%%%%%%%%%%%%%%%%%%%%%%%%%%%%%%%%%%%%%%%%%%%%%%%%%
%% Support for make parts of the document (such as page headers)
%% non-selectable
%%   (usually, you also want to ignore these parts when counting
%%    words)
\ifelseht{%
  \DeclareRobustCommand\squelch[1]{#1}%
}{%
  \ifelsewc{%
    \DeclareRobustCommand\squelch[1]{}%
  }{%
    \usepackage{accsupp}
    \DeclareRobustCommand\squelch[1]{%
      \BeginAccSupp{method=plain,ActualText={}}#1\EndAccSupp{}}%
  }
}%
%%%%%%%%%%%%%%%%%%%%%%%%%%%%%%%%%%%%%%%%%%%%%%%%%%%%%%%%%%%%%%%%%%%%%%
%% Table captions aligned with table:
\usepackage{varwidth}
%%%%%%%%%%%%%%%%%%%%%%%%%%%%%%%%%%%%%%%%%%%%%%%%%%%%%%%%%%%%%%%%%%%%%%
%% default language:
\ifthenelse{\equal{\myfirstlang}{german}}{%
  \usepackage[english,german]{babel}%
}{%
  \usepackage[german,english]{babel}%
}
%%%%%%%%%%%%%%%%%%%%%%%%%%%%%%%%%%%%%%%%%%%%%%%%%%%%%%%%%%%%%%%%%%%%%%
%% bibliography:
\usepackage{natbib}
%%%%%%%%%%%%%%%%%%%%%%%%%%%%%%%%%%%%%%%%%%%%%%%%%%%%%%%%%%%%%%%%%%%%%%
%% AMS mathematics:
%%   (include before txfonts, and use txfonts's iint)
\usepackage{amsmath}
\ifnwc{%
  \let\iint\relax
}%
%%%%%%%%%%%%%%%%%%%%%%%%%%%%%%%%%%%%%%%%%%%%%%%%%%%%%%%%%%%%%%%%%%%%%%
%% The package microtype adjusts font width for individual words
%%   in order to achieve better line breaking.
%%   Also, margin kerning makes the margin look more even.
%%   This also renders the use of \sloppy unnecessary.
\usepackage{microtype}
\fussy
%%%%%%%%%%%%%%%%%%%%%%%%%%%%%%%%%%%%%%%%%%%%%%%%%%%%%%%%%%%%%%%%%%%%%%
%% providing umlauts:
\usepackage[utf8]{inputenc}
\ifnwc{%
\usepackage[T1]{fontenc}
}
%%%%%%%%%%%%%%%%%%%%%%%%%%%%%%%%%%%%%%%%%%%%%%%%%%%%%%%%%%%%%%%%%%%%%%
%% line spacing
\ifnwc{%
\usepackage{setspace}
}
%%%%%%%%%%%%%%%%%%%%%%%%%%%%%%%%%%%%%%%%%%%%%%%%%%%%%%%%%%%%%%%%%%%%%%
%% Allow rotating single pages
%% The package is orientated correctly when displayed on screen
\usepackage{pdflscape}
%% Extract command from this package, to be used in the
%% "sidewaysfigure" environment...
\makeatletter
\let\AddPageRotate=\PLS@AddRotate
\let\RemovePageRotate=\PLS@RemoveRotate
\makeatother
%% ...but reset the command that un-rotates pages inside the
%% "landscape" environment, so that the page rotation stays in effect
%% there
%%
%% (AtBeginEnvironment
\ifdefined\AtBeginEnvironment
  \AtBeginEnvironment{landscape}{\def\RemovePageRotate{\relax}}
\else
  \message{Use a recent version of etoolbox package to get
  correct page rotation in the landscape environment.}
\fi
%%%%%%%%%%%%%%%%%%%%%%%%%%%%%%%%%%%%%%%%%%%%%%%%%%%%%%%%%%%%%%%%%%%%%%
%% Hack for orientating sideways{tables,figures} correctly
%%
%% By default, every page is not rotated:
\usepackage{everypage}
\AddEverypageHook{\RemovePageRotate}%
%% The package floatpag implements a hook that is executed
%% when the current float is displayed. We use this hook
%% to add an \AddPageRotate command for the sideways floats.
\usepackage{floatpag}
\ifnwc{%
  \AtBeginDocument{%
    \pagestyle{\mypagestyle}%
    \floatpagestyle{\mypagestyle}%
    \rotfloatpagestyle{\mypagestyle}%
  }
}
\makeatletter
\def\thisfloatcommand#1{%
  \expandafter\expandafter\expandafter\gdef\expandafter\csname\number\@currbox @float\endcsname{#1}\relax}
\makeatother
%%%%%%%%%%%%%%%%%%%%%%%%%%%%%%%%%%%%%%%%%%%%%%%%%%%%%%%%%%%%%%%%%%%%%%
%% Captions and subcaptions
\usepackage{caption}
\ifnht{%
  \usepackage[singlelinecheck=on,labelformat=simple]{subcaption}
  \renewcommand\thesubfigure{(\alph{subfigure})}
  \renewcommand\thesubtable{(\alph{subtable})}
}
%%%%%%%%%%%%%%%%%%%%%%%%%%%%%%%%%%%%%%%%%%%%%%%%%%%%%%%%%%%%%%%%%%%%%%
%% providing graphics:
\usepackage{graphics}
\usepackage{graphicx}
%%%%%%%%%%%%%%%%%%%%%%%%%%%%%%%%%%%%%%%%%%%%%%%%%%%%%%%%%%%%%%%%%%%%%%
%% sideways figures and tables:
\usepackage[figuresright]{rotating}
%%%%%%%%%%%%%%%%%%%%%%%%%%%%%%%%%%%%%%%%%%%%%%%%%%%%%%%%%%%%%%%%%%%%%%
%% figures:
%%   The following are sometimes needed to avoid pushing
%%   the figs to the end of the text.
\def\textfraction{0.0}
\def\topfraction{0.9999}
\def\dbltopfraction{0.9999}
\def\floatpagefraction{0.8}
%%%%%%%%%%%%%%%%%%%%%%%%%%%%%%%%%%%%%%%%%%%%%%%%%%%%%%%%%%%%%%%%%%%%%%
%% tables:
\usepackage{multirow}
%%%%%%%%%%%%%%%%%%%%%%%%%%%%%%%%%%%%%%%%%%%%%%%%%%%%%%%%%%%%%%%%%%%%%%
%% pretty printing:
\usepackage{listings}
%%%%%%%%%%%%%%%%%%%%%%%%%%%%%%%%%%%%%%%%%%%%%%%%%%%%%%%%%%%%%%%%%%%%%%
%% XML code setup:
\lstloadlanguages{XML}
%%
\lstset {
  columns=fullflexible,
  showstringspaces=false,
  basicstyle=\ttfamily\footnotesize,
  lineskip=0pt,
  breaklines=true,
  breakatwhitespace=true,
  breakindent=12pt,
  fontadjust=true,
  keywordstyle=\bfseries,
  commentstyle=\itshape,
  stringstyle=\bfseries\itshape,
  xleftmargin=0mm,
  xrightmargin=0mm,
  tabsize=2
}

\lstdefinelanguage{XML}
{
  morestring=[b]",
  moredelim=[s][\bfseries\color{Maroon}]{<}{\ },
  moredelim=[s][\bfseries\color{Maroon}]{</}{>},
  moredelim=[l][\bfseries\color{Maroon}]{/>},
  moredelim=[l][\bfseries\color{Maroon}]{>},
  morecomment=[s]{<?}{?>},
  morecomment=[s]{<!--}{-->},
  commentstyle=\color{DarkOliveGreen},
  stringstyle=\color{blue},
  identifierstyle=\color{red}
}
%%%%%%%%%%%%%%%%%%%%%%%%%%%%%%%%%%%%%%%%%%%%%%%%%%%%%%%%%%%%%%%%%%%%%%
%% Support for figure and table count:
\usepackage{totcount}
\usepackage{calc}
\regtotcounter{figure}
\regtotcounter{table}
%%%%%%%%%%%%%%%%%%%%%%%%%%%%%%%%%%%%%%%%%%%%%%%%%%%%%%%%%%%%%%%%%%%%%%
%% Less space between enumeration lists
\usepackage{paralist}
\renewenvironment{itemize}[1]{\begin{compactitem}#1}{\end{compactitem}}
\renewenvironment{enumerate}[1]{\begin{compactenum}#1}{\end{compactenum}}
\renewenvironment{description}[0]{\begin{compactdesc}}{\end{compactdesc}}
%%%%%%%%%%%%%%%%%%%%%%%%%%%%%%%%%%%%%%%%%%%%%%%%%%%%%%%%%%%%%%%%%%%%%%
%% Typesetting-quality tables
\usepackage{booktabs}
%%%%%%%%%%%%%%%%%%%%%%%%%%%%%%%%%%%%%%%%%%%%%%%%%%%%%%%%%%%%%%%%%%%%%%
%% new verbatim environment
\usepackage{verbatim}
%%%%%%%%%%%%%%%%%%%%%%%%%%%%%%%%%%%%%%%%%%%%%%%%%%%%%%%%%%%%%%%%%%%%%%
%% Extended color definitions
\PassOptionsToPackage{svgnames}{xcolor}
\usepackage{xcolor}
%%%%%%%%%%%%%%%%%%%%%%%%%%%%%%%%%%%%%%%%%%%%%%%%%%%%%%%%%%%%%%%%%%%%%%
%% Just in case (before hyperref):
\usepackage{float}
\usepackage{longtable}
\usepackage{ltabptch}
\usepackage{nameref}
%%%%%%%%%%%%%%%%%%%%%%%%%%%%%%%%%%%%%%%%%%%%%%%%%%%%%%%%%%%%%%%%%%%%%%
%% Use hyper-refs for URLs and citations,
%% allow line breaks for URLs
%%   include after all other packages, especially after titlesec
\PassOptionsToPackage{obeyspaces}{url}
\ifnht{\usepackage{hyperref}}
\usepackage{url}
%%%%%%%%%%%%%%%%%%%%%%%%%%%%%%%%%%%%%%%%%%%%%%%%%%%%%%%%%%%%%%%%%%%%%%
%% convenient referencing (after hyperref):
\usepackage[capitalize]{cleveref}
%%%%%%%%%%%%%%%%%%%%%%%%%%%%%%%%%%%%%%%%%%%%%%%%%%%%%%%%%%%%%%%%%%%%%%
%% tables (after hyperref):
\usepackage{tabularx}
%%%%%%%%%%%%%%%%%%%%%%%%%%%%%%%%%%%%%%%%%%%%%%%%%%%%%%%%%%%%%%%%%%%%%%
%% do not count words in references
%% count only title, subtitle and abstract, no auxiliary information
\ifwc{%
  \renewcommand{\bibliography}[1]{}%
  \AtBeginDocument{%
    \renewcommand{\createtitlepage}{\mytitle \mysubtitle}%
    \renewcommand{\createabstract}[1]{#1}%
  }%
}
%%%%%%%%%%%%%%%%%%%%%%%%%%%%%%%%%%%%%%%%%%%%%%%%%%%%%%%%%%%%%%%%%%%%%%
%% for document classes that do not provide \captionabove and
%% \captionbelow
\providecommand{\captionabove}[2][]{\caption[#1]{#2}}
\providecommand{\captionbelow}[2][]{\caption[#1]{#2}}

%%%%%%%%%%%%%%%%%%%%%%%%%%%%%%%%%%%%%%%%%%%%%%%%%%%%%%%%%%%%%%%%%%%%%%
%% Header and footer definition:
\usepackage{fancyhdr}%
\newcommand{\mypagestyle}{fancy}
\fancyhf{}%
\fancyhead[R]{\footnotesize \squelch{\nouppercase{\thepage}}}%
\fancyhead[L]{\footnotesize \squelch{\nouppercase{\internauthorstring}}}%
\renewcommand{\headrulewidth}{0pt}%
\renewcommand{\footrulewidth}{0pt}%
%%%%%%%%%%%%%%%%%%%%%%%%%%%%%%%%%%%%%%%%%%%%%%%%%%%%%%%%%%%%%%%%%%%%%%
%% paragraph settings:
\setlength{\parindent}{\parindent}%
\setlength{\parskip}{0pt plus 0.001pt}%
\flushbottom%
%%%%%%%%%%%%%%%%%%%%%%%%%%%%%%%%%%%%%%%%%%%%%%%%%%%%%%%%%%%%%%%%%%%%%%
%% caption settings:
\DeclareCaptionLabelFormat{allcaps}{\MakeUppercase{#1\ #2}}
\captionsetup*{labelformat=allcaps}
%%%%%%%%%%%%%%%%%%%%%%%%%%%%%%%%%%%%%%%%%%%%%%%%%%%%%%%%%%%%%%%%%%%%%%
%% Define the depth of numbering parts,chapter,sections and paragraphs:
%%   Numbers representing the depth of sectional units:
%%   -1 = \part    (in book or report document classes)
%%    0 = \chapter (in book or report document classes)
%%    0 = \part    (in article document classes)
%%    1 = \section
%%    2 = \subsection
%%    3 = \subsubsection
%%    4 = \paragraph
%%    5 = \subparagraph
\setcounter{secnumdepth}{3}
%%%%%%%%%%%%%%%%%%%%%%%%%%%%%%%%%%%%%%%%%%%%%%%%%%%%%%%%%%%%%%%%%%%%%%
%% citation style:
\setcitestyle{numbers}
\bibpunct{\textit\bgroup(}{)\egroup}{,}{n}{,}{,}
\providecommand{\citenumfont}[1]{}
\renewcommand{\citenumfont}[1]{\textit{#1}}
\renewcommand{\bibnumfmt}[1]{#1.}
\ifthenelse{\equal{\myfirstlang}{german}}{%
  \bibliographystyle{\mypath../_latexfiles/styles/template_ivt-unsrt-ger}%
}{%
  \bibliographystyle{\mypath../_latexfiles/styles/template_ivt-unsrt-eng}%
}
%%%%%%%%%%%%%%%%%%%%%%%%%%%%%%%%%%%%%%%%%%%%%%%%%%%%%%%%%%%%%%%%%%%%%%
%% no indentation for formulae:
\setlength\mathindent{0pt}
%%%%%%%%%%%%%%%%%%%%%%%%%%%%%%%%%%%%%%%%%%%%%%%%%%%%%%%%%%%%%%%%%%%%%%
%% Times font
% !TeX encoding = usascii
%%%%%%%%%%%%%%%%%%%%%%%%%%%%%%%%%%%%%%%%%%%%%%%%%%%%%%%%%%%%%%%%%%%%%%
%% Font:
%% (unfortunately, mathptmx does not provide bold math fonts,
%%  and times still uses the rather different CM font for maths)
\ifnwc{%
  \ifnht{%
    \usepackage{newtxtext}
    \usepackage{newtxmath}
    \usepackage{courier}
    \undef\oct
  }
}


%%%%%%%%%%%%%%%%%%%%%%%%%%%%%%%%%%%%%%%%%%%%%%%%%%%%%%%%%%%%%%%%%%%%%%
%% single line spacing
\singlespacing
%%%%%%%%%%%%%%%%%%%%%%%%%%%%%%%%%%%%%%%%%%%%%%%%%%%%%%%%%%%%%%%%%%%%%%
%% heading settings:
%%   now using the titlesec package
\usepackage{titlesec}
%%
%% legacy (match appearance of previous version)
\newlength{\sectionbeforedist}
\setlength{\sectionbeforedist}{4ex plus1ex minus0.5ex}
\newlength{\sectionafterdist}
\setlength{\sectionafterdist}{2ex plus0.5ex minus0.25ex}
%%
%% section
\titleformat{\section}{\normalfont\bfseries}{}{0pt}{\MakeUppercase}
\titlespacing*{\section}{0in}{\sectionbeforedist}{\sectionafterdist}
%%
%% subsection
\titleformat{\subsection}{\normalfont\bfseries}{}{0pt}{}
\titlespacing*{\subsection}{0in}{\sectionbeforedist}{\sectionafterdist}
%%
%% subsubsection
\titleformat{\subsubsection}{\normalfont\itshape}{}{0pt}{}
\titlespacing*{\subsubsection}{0in}{\sectionbeforedist}{\sectionafterdist}
%%
%% paragraph
\titleformat{\paragraph}[runin]{\normalfont\bfseries}{}{0pt}{}
\titlespacing*{\paragraph}{0in}{\sectionbeforedist}{5pt}
%%
%% subparagraph
\titleformat{\subparagraph}[runin]{\normalfont\itshape}{}{0pt}{}
\titlespacing*{\subparagraph}{0in}{\sectionbeforedist}{5pt}
%%%%%%%%%%%%%%%%%%%%%%%%%%%%%%%%%%%%%%%%%%%%%%%%%%%%%%%%%%%%%%%%%%%%%%

%%%%%%%%%%%%%%%%%%%%%%%%%%%%%%%%%%%%%%%%%%%%%%%%%%%%%%%%%%%%%%%%%%%%%%
%%%%%%%%%%%%%%%%%%%%%%%%%%%%%%%%%%%%%%%%%%%%%%%%%%%%%%%%%%%%%%%%%%%%%%
%% Language-specific words
% !TeX encoding = usascii
%%%%%%%%%%%%%%%%%%%%%%%%%%%%%%%%%%%%%%%%%%%%%%%%%%%%%%%%%%%%%%%%%%%%%%%%%%%%%%%%%%%%%%%%%%%%%%%%%%%%%%%%%%%%%%%%%%%%%%%%%%%%%%%%%%%%%%%%%%%%
%%
%% The following defines language specific words
%%   These are internal commands. They are not used in the main file.
%%   Langugage specific word commands always starts with '\word'
%%
%%%%%%%%%%%%%%%%%%%%%%%%%%%%%%%%%%%%%%%%%%%%%%%%%%%%%%%%%%%%%%%%%%%%%%
%%%%%%%%%%%%%%%%%%%%%%%%%%%%%%%%%%%%%%%%%%%%%%%%%%%%%%%%%%%%%%%%%%%%%%
%% Error message
\newcommand{\langerrmessage}{\errmessage{Wrong language. Define myfirstlang as either english or german.}}
%%%%%%%%%%%%%%%%%%%%%%%%%%%%%%%%%%%%%%%%%%%%%%%%%%%%%%%%%%%%%%%%%%%%%%
%% and/und
\newcommand{\wordand}{\iflanguage{english}{and}{\iflanguage{german}{und}{\langerrmessage}}}
%%%%%%%%%%%%%%%%%%%%%%%%%%%%%%%%%%%%%%%%%%%%%%%%%%%%%%%%%%%%%%%%%%%%%%
%% phone/Tel
\newcommand{\wordphone}{\iflanguage{english}{phone}{\iflanguage{german}{Tel}{\langerrmessage}}}
%%%%%%%%%%%%%%%%%%%%%%%%%%%%%%%%%%%%%%%%%%%%%%%%%%%%%%%%%%%%%%%%%%%%%%
%% fax/Fax
\newcommand{\wordfax}{\iflanguage{english}{fax}{\iflanguage{german}{Fax}{\langerrmessage}}}
%%%%%%%%%%%%%%%%%%%%%%%%%%%%%%%%%%%%%%%%%%%%%%%%%%%%%%%%%%%%%%%%%%%%%%
%% email/EMail
\newcommand{\wordemail}{\iflanguage{english}{email}{\iflanguage{german}{Mail}{\langerrmessage}}}
%%%%%%%%%%%%%%%%%%%%%%%%%%%%%%%%%%%%%%%%%%%%%%%%%%%%%%%%%%%%%%%%%%%%%%
%% Preferred citation style/Bevorzugter Zitierstil
\newcommand{\wordprefcit}{\iflanguage{english}{Preferred citation style}{\iflanguage{german}{Bevorzugter Zitierstil}{\langerrmessage}}}
%%%%%%%%%%%%%%%%%%%%%%%%%%%%%%%%%%%%%%%%%%%%%%%%%%%%%%%%%%%%%%%%%%%%%%
%% Keywords/Schl\"usselw\"orter
\newcommand{\wordkeywords}{\iflanguage{english}{Keywords}{\iflanguage{german}{Schl\"usselw\"orter}{\langerrmessage}}}
%%%%%%%%%%%%%%%%%%%%%%%%%%%%%%%%%%%%%%%%%%%%%%%%%%%%%%%%%%%%%%%%%%%%%%
%% Quelle/Source
\newcommand{\wordsource}{\iflanguage{english}{Source:\ }{\iflanguage{german}{Quelle:\ }{\langerrmessage}}}
%%%%%%%%%%%%%%%%%%%%%%%%%%%%%%%%%%%%%%%%%%%%%%%%%%%%%%%%%%%%%%%%%%%%%%
%% jan,feb,mar,apr,may,jun,jul,aug,sep,oct,nov,dec -> german/english
\newcommand{\wordmonth}{%
  \ifthenelse{\equal{\mymonth}{jan}}{\iflanguage{english}{January}{\iflanguage{german}{Januar}{\langerrmessage}}}%
  {\ifthenelse{\equal{\mymonth}{feb}}{\iflanguage{english}{February}{\iflanguage{german}{Februar}{\langerrmessage}}}%
   {\ifthenelse{\equal{\mymonth}{mar}}{\iflanguage{english}{March}{\iflanguage{german}{M\"arz}{\langerrmessage}}}%
    {\ifthenelse{\equal{\mymonth}{apr}}{\iflanguage{english}{April}{\iflanguage{german}{April}{\langerrmessage}}}%
     {\ifthenelse{\equal{\mymonth}{may}}{\iflanguage{english}{May}{\iflanguage{german}{Mai}{\langerrmessage}}}%
      {\ifthenelse{\equal{\mymonth}{jun}}{\iflanguage{english}{June}{\iflanguage{german}{Juni}{\langerrmessage}}}%
       {\ifthenelse{\equal{\mymonth}{jul}}{\iflanguage{english}{July}{\iflanguage{german}{Juli}{\langerrmessage}}}%
        {\ifthenelse{\equal{\mymonth}{aug}}{\iflanguage{english}{August}{\iflanguage{german}{August}{\langerrmessage}}}%
         {\ifthenelse{\equal{\mymonth}{sep}}{\iflanguage{english}{September}{\iflanguage{german}{September}{\langerrmessage}}}%
          {\ifthenelse{\equal{\mymonth}{oct}}{\iflanguage{english}{October}{\iflanguage{german}{Oktober}{\langerrmessage}}}%
           {\ifthenelse{\equal{\mymonth}{nov}}{\iflanguage{english}{November}{\iflanguage{german}{November}{\langerrmessage}}}%
            {\ifthenelse{\equal{\mymonth}{dec}}{\iflanguage{english}{December}{\iflanguage{german}{Dezember}{\langerrmessage}}}%
             {}}}}}}}}}}}}}
%%%%%%%%%%%%%%%%%%%%%%%%%%%%%%%%%%%%%%%%%%%%%%%%%%%%%%%%%%%%%%%%%%%%%%
%% jan,feb,mar,apr,may,jun,jul,aug,sep,oct,nov,dec -> num
\newcommand{\nummonth}{%
  \ifthenelse{\equal{\mymonth}{jan}}{01}%
  {\ifthenelse{\equal{\mymonth}{feb}}{02}%
   {\ifthenelse{\equal{\mymonth}{mar}}{03}%
    {\ifthenelse{\equal{\mymonth}{apr}}{04}%
     {\ifthenelse{\equal{\mymonth}{may}}{05}%
      {\ifthenelse{\equal{\mymonth}{jun}}{06}%
       {\ifthenelse{\equal{\mymonth}{jul}}{07}%
        {\ifthenelse{\equal{\mymonth}{aug}}{08}%
         {\ifthenelse{\equal{\mymonth}{sep}}{09}%
          {\ifthenelse{\equal{\mymonth}{oct}}{10}%
           {\ifthenelse{\equal{\mymonth}{nov}}{11}%
            {\ifthenelse{\equal{\mymonth}{dec}}{12}%
             {}}}}}}}}}}}}}
%%%%%%%%%%%%%%%%%%%%%%%%%%%%%%%%%%%%%%%%%%%%%%%%%%%%%%%%%%%%%%%%%%%%%%



%%%%%%%%%%%%%%%%%%%%%%%%%%%%%%%%%%%%%%%%%%%%%%%%%%%%%%%%%%%%%%%%%%%%%%
%%%%%%%%%%%%%%%%%%%%%%%%%%%%%%%%%%%%%%%%%%%%%%%%%%%%%%%%%%%%%%%%%%%%%%
%% Internal commands
% !TeX encoding = usascii
%%%%%%%%%%%%%%%%%%%%%%%%%%%%%%%%%%%%%%%%%%%%%%%%%%%%%%%%%%%%%%%%%%%%%%
%%%%%%%%%%%%%%%%%%%%%%%%%%%%%%%%%%%%%%%%%%%%%%%%%%%%%%%%%%%%%%%%%%%%%%
%%
%% The following defines other internal commands
%%   Internal command are not used by the main file.
%%   Internal commands always starts with '\internal'
%%
%%%%%%%%%%%%%%%%%%%%%%%%%%%%%%%%%%%%%%%%%%%%%%%%%%%%%%%%%%%%%%%%%%%%%%
%%%%%%%%%%%%%%%%%%%%%%%%%%%%%%%%%%%%%%%%%%%%%%%%%%%%%%%%%%%%%%%%%%%%%%
%% user defined german keywords/user defined english keywords
%%   The command only returns a string which are defined by the
%%   main file (\mykeywordsEN or \mykeywordsDE)
\newcommand{\internkeywords}{\iflanguage{english}{\mykeywordsEN}{\iflanguage{german}{\mykeywordsDE}{\langerrmessage}}}
%%%%%%%%%%%%%%%%%%%%%%%%%%%%%%%%%%%%%%%%%%%%%%%%%%%%%%%%%%%%%%%%%%%%%%
%% papertype definition
%%   i.e. working paper/Arbeitsberichte Verkehrs- und Raumplanung
%%   i.e. disseration/Doktorarbeit
\providecommand{\internpapertype}{}
%%%%%%%%%%%%%%%%%%%%%%%%%%%%%%%%%%%%%%%%%%%%%%%%%%%%%%%%%%%%%%%%%%%%%%
%% user defined german institution/user defined english institution
%%   The command only returns a string which are defined by the
%%   main file (\myinstitutionEN or \myinstitutionDE)
\newcommand{\interninstitution}{\iflanguage{english}{\myinstitutionEN}{\iflanguage{german}{\myinstitutionDE}{\langerrmessage}}}
%%%%%%%%%%%%%%%%%%%%%%%%%%%%%%%%%%%%%%%%%%%%%%%%%%%%%%%%%%%%%%%%%%%%%%
%% authorlist
%%   prints out all the given author in a specified way
\newcommand{\internauthorlist}{%
  \providecommand{\myseventhauthor}{}%
  \providecommand{\myeighthauthor}{}%
  \providecommand{\myninthauthor}{}%
  \providecommand{\mytenthauthor}{}%
  \providecommand{\myeleventhauthor}{}%
  \providecommand{\mytwelfthauthor}{}%
  \ifthenelse{\equal{\myseventhauthor}{}}{%
  \ifthenelse{\equal{\myfirstauthor}{}}{}{\myfirstauthor}%
  \ifthenelse{\equal{\mysecondauthor}{}}{}{\newline\mysecondauthor}%
  \ifthenelse{\equal{\mythirdauthor}{}}{}{\newline\mythirdauthor}%
  \ifthenelse{\equal{\myfourthauthor}{}}{}{\newline\myfourthauthor}%
  \ifthenelse{\equal{\myfifthauthor}{}}{}{\newline\myfifthauthor}%
  \ifthenelse{\equal{\mysixthauthor}{}}{}{\newline\mysixthauthor}%
  }%
  {%
  \begin{tabular}{@{}ll@{}}
  \myfirstauthor & \mysecondauthor \\
  \mythirdauthor & \myfourthauthor \\
  \myfifthauthor & \mysixthauthor \\
  \myseventhauthor & \myeighthauthor \\
  \myninthauthor & \mytenthauthor \\
  \myeleventhauthor & \mytwelfthauthor \\
  \end{tabular}%
  }%
}
%%%%%%%%%%%%%%%%%%%%%%%%%%%%%%%%%%%%%%%%%%%%%%%%%%%%%%%%%%%%%%%%%%%%%%
%% author names as a single line
\newcommand{\internauthorstringlong}{%
  \providecommand{\myseventhauthor}{}%
  \providecommand{\myeighthauthor}{}%
  \providecommand{\myninthauthor}{}%
  \providecommand{\mytenthauthor}{}%
  \providecommand{\myeleventhauthor}{}%
  \providecommand{\mytwelfthauthor}{}%
  \ifthenelse{\equal{\myfirstauthor}{}}{}{%
    \myfirstauthor%
    \ifthenelse{\equal{\mysecondauthor}{}}{}{%
      \ifthenelse{\equal{\mythirdauthor}{}}{\ \wordand}{,}
      \mysecondauthor%
      \ifthenelse{\equal{\mythirdauthor}{}}{}{%
        \ifthenelse{\equal{\myfourthauthor}{}}{\ \wordand}{,}
        \mythirdauthor%
        \ifthenelse{\equal{\myfourthauthor}{}}{}{%
          \ifthenelse{\equal{\myfifthauthor}{}}{\ \wordand}{,}
          \myfourthauthor%
          \ifthenelse{\equal{\myfifthauthor}{}}{}{%
            \ifthenelse{\equal{\mysixthauthor}{}}{\ \wordand}{,}
            \myfifthauthor%
            \ifthenelse{\equal{\mysixthauthor}{}}{}{%
              \ifthenelse{\equal{\myseventhauthor}{}}{\ \wordand}{,}
              \mysixthauthor%
              \ifthenelse{\equal{\myseventhauthor}{}}{}{%
                \ifthenelse{\equal{\myeighthauthor}{}}{\ \wordand}{,}
                \myseventhauthor%
                \ifthenelse{\equal{\myeighthauthor}{}}{}{%
                  \ifthenelse{\equal{\myninthauthor}{}}{\ \wordand}{,}
                  \myeighthauthor%
                  \ifthenelse{\equal{\myninthauthor}{}}{}{%
                    \ifthenelse{\equal{\mytenthauthor}{}}{\ \wordand}{,}
                    \myninthauthor%
                    \ifthenelse{\equal{\mytenthauthor}{}}{}{%
                      \ifthenelse{\equal{\myeleventhauthor}{}}{\ \wordand}{,}
                      \mytenthauthor%
                      \ifthenelse{\equal{\myeleventhauthor}{}}{}{%
                        \ifthenelse{\equal{\mytwelfthauthor}{}}{\ \wordand}{,}
                        \myeleventhauthor%
                        \ifthenelse{\equal{\mytwelfthauthor}{}}{}{%
                          \ifthenelse{\equal{}{}}{\ \wordand}{,}
                          \mytwelfthauthor%
                        }%
                      }%
                    }%
                  }%
                }%
              }%
            }%
          }%
        }%
      }%
    }%
  }%
  \ 
}
%%%%%%%%%%%%%%%%%%%%%%%%%%%%%%%%%%%%%%%%%%%%%%%%%%%%%%%%%%%%%%%%%%%%%%
%% author names as a single line, abbreviated
\newcommand{\internauthorstring}{%
  \providecommand{\myseventhauthorREF}{}%
  \providecommand{\myeighthauthorREF}{}%
  \providecommand{\myninthauthorREF}{}%
  \providecommand{\mytenthauthorREF}{}%
  \providecommand{\myeleventhauthorREF}{}%
  \providecommand{\mytwelfthauthorREF}{}%
  \ifthenelse{\equal{\myfirstauthorREF}{}}{}{%
    \myfirstauthorREF%
    \ifthenelse{\equal{\mysecondauthorREF}{}}{}{%
      \ifthenelse{\equal{\mythirdauthorREF}{}}{\ \wordand}{,}
      \mysecondauthorREF%
      \ifthenelse{\equal{\mythirdauthorREF}{}}{}{%
        \ifthenelse{\equal{\myfourthauthorREF}{}}{\ \wordand}{,}
        \mythirdauthorREF%
        \ifthenelse{\equal{\myfourthauthorREF}{}}{}{%
          \ifthenelse{\equal{\myfifthauthorREF}{}}{\ \wordand}{,}
          \myfourthauthorREF%
          \ifthenelse{\equal{\myfifthauthorREF}{}}{}{%
            \ifthenelse{\equal{\mysixthauthorREF}{}}{\ \wordand}{,}
            \myfifthauthorREF%
            \ifthenelse{\equal{\mysixthauthorREF}{}}{}{%
              \ifthenelse{\equal{\myseventhauthorREF}{}}{\ \wordand}{,}
              \mysixthauthorREF%
              \ifthenelse{\equal{\myseventhauthorREF}{}}{}{%
                \ifthenelse{\equal{\myeighthauthorREF}{}}{\ \wordand}{,}
                \myseventhauthorREF%
                \ifthenelse{\equal{\myeighthauthorREF}{}}{}{%
                  \ifthenelse{\equal{\myninthauthorREF}{}}{\ \wordand}{,}
                  \myeighthauthorREF%
                  \ifthenelse{\equal{\myninthauthorREF}{}}{}{%
                    \ifthenelse{\equal{\mytenthauthorREF}{}}{\ \wordand}{,}
                    \myninthauthorREF%
                    \ifthenelse{\equal{\mytenthauthorREF}{}}{}{%
                      \ifthenelse{\equal{\myeleventhauthorREF}{}}{\ \wordand}{,}
                      \mytenthauthorREF%
                      \ifthenelse{\equal{\myeleventhauthorREF}{}}{}{%
                        \ifthenelse{\equal{\mytwelfthauthorREF}{}}{\ \wordand}{,}
                        \myeleventhauthorREF%
                        \ifthenelse{\equal{\mytwelfthauthorREF}{}}{}{%
                          \ifthenelse{\equal{}{}}{\ \wordand}{,}
                          \mytwelfthauthorREF%
                        }%
                      }%
                    }%
                  }%
                }%
              }%
            }%
          }%
        }%
      }%
    }%
  }%
  \ 
}
%%%%%%%%%%%%%%%%%%%%%%%%%%%%%%%%%%%%%%%%%%%%%%%%%%%%%%%%%%%%%%%%%%%%%%
%% author names with address references (used in HKSTS)
\newcommand{\internauthortitlestring}{%
  \ifthenelse{\equal{\mysecondauthor}{}}{%
    \myfirstauthor$^\myfirstauthoraddress$\ %
  }{%
    \ifthenelse{\equal{\mythirdauthor}{}}{%
      \myfirstauthor$^\myfirstauthoraddress$\ \wordand\ \mysecondauthor$^\mysecondauthoraddress$\ %
    }{%
      \ifthenelse{\equal{\myfourthauthor}{}}{%
        \myfirstauthor$^\myfirstauthoraddress$, \mysecondauthor$^\mysecondauthoraddress$\ \wordand\ \mythirdauthor$^\mythirdauthoraddress$\ %
      }{%
        \ifthenelse{\equal{\myfifthauthor}{}}{%
          \myfirstauthor$^\myfirstauthoraddress$, \mysecondauthor$^\mysecondauthoraddress$, \mythirdauthor$^\mythirdauthoraddress$\ \wordand\ \myfourthauthor$^\myfourthauthoraddress$\ %
        }{%
          \ifthenelse{\equal{\mysixthauthor}{}}{%
            \myfirstauthor$^\myfirstauthoraddress$, \mysecondauthor$^\mysecondauthoraddress$, \mythirdauthor$^\mythirdauthoraddress$, \myfourthauthor$^\myfourthauthoraddress$\ \wordand\ \myfifthauthor$^\myfifthauthoraddress$\ %
          }{%
            \myfirstauthor$^\myfirstauthoraddress$, \mysecondauthor$^\mysecondauthoraddress$, \mythirdauthor$^\mythirdauthoraddress$, \myfourthauthor$^\myfourthauthoraddress$, \myfifthauthor$^\myfifthauthoraddress$\ \wordand\ \mysixthauthor$^\mysixthauthoraddress$\ %
          }%
        }%
      }%
    }%
  }%
}
%%%%%%%%%%%%%%%%%%%%%%%%%%%%%%%%%%%%%%%%%%%%%%%%%%%%%%%%%%%%%%%%%%%%%%
%% address list compatible with \internauthortitlestring
\newcommand{\internaddresslist}{%
  \ifne\mysecondaddress{$^a$}%
  \myfirstaddress\\%
  \ifne\mysecondaddress{%
    $^b$\mysecondaddress\\%
      \ifne\mythirdaddress{%
        $^c$\mythirdaddress\\%
          \ifne\myfourthaddress{%
            $^d$\myfourthaddress\\%
              \ifne\myfifthaddress{%
                $^e$\myfifthaddress\\%
                  \ifne\mysixthaddress{%
                    $^f$\mysixthaddress\\%
  }}}}}%
}
%%%%%%%%%%%%%%%%%%%%%%%%%%%%%%%%%%%%%%%%%%%%%%%%%%%%%%%%%%%%%%%%%%%%%%
%% \internmakeonecolumn{entry1}
%%   creates a table with one column containing the given entries
\AtBeginDocument{%
  \newlength{\interncolumnwidth}%
}
\newcommand{\internmakeonecolumn}[1]{{%
  \settowidth{\interncolumnwidth}{#1}%
  \ifthenelse{\lengthtest{\interncolumnwidth=0pt}}{}{%
    #1\\
  }%
}}
%%%%%%%%%%%%%%%%%%%%%%%%%%%%%%%%%%%%%%%%%%%%%%%%%%%%%%%%%%%%%%%%%%%%%%
%% \internmaketwocolumns{entry1}{entry2}
%%   creates a table with two columns containing the given entries
\newcommand{\internmaketwocolumns}[2]{
  \settowidth{\interncolumnwidth}{#1}%
  \ifthenelse{\lengthtest{\interncolumnwidth=0pt}}{}{%
    #1\hfill{}#2\\
  }%
}
%%%%%%%%%%%%%%%%%%%%%%%%%%%%%%%%%%%%%%%%%%%%%%%%%%%%%%%%%%%%%%%%%%%%%%
%% \internmakethreecolumns{entry1}{entry2}{entry3}
%%   creates a table with three columns containing the given entries
\newcommand{\internmakethreecolumns}[3]{
  \settowidth{\interncolumnwidth}{#1}%
  \ifthenelse{\lengthtest{\interncolumnwidth=0pt}}{}{%
    #1\hfill{}#2\hfill{}#3\\
  }%
}
%%%%%%%%%%%%%%%%%%%%%%%%%%%%%%%%%%%%%%%%%%%%%%%%%%%%%%%%%%%%%%%%%%%%%%
%% citation
%%   returns the language specific citation of the paper
%%   default: no citation
\providecommand{\interncitation}{}
%%%%%%%%%%%%%%%%%%%%%%%%%%%%%%%%%%%%%%%%%%%%%%%%%%%%%%%%%%%%%%%%%%%%%%

%%%%%%%%%%%%%%%%%%%%%%%%%%%%%%%%%%%%%%%%%%%%%%%%%%%%%%%%%%%%%%%%%%%%%%
%%%%%%%%%%%%%%%%%%%%%%%%%%%%%%%%%%%%%%%%%%%%%%%%%%%%%%%%%%%%%%%%%%%%%%
%%
%% Standard commands
%%   The following commands >>>HAVE<<< to be defined, because
%%   they are called by the main file.
%%   Standard commands always starts with '\create'. (except
%%   '\switchlanguage' since it would sound stupid ^_^ and
%%   '\ackname' since it is the standard def in babel the package)
%%
%%%%%%%%%%%%%%%%%%%%%%%%%%%%%%%%%%%%%%%%%%%%%%%%%%%%%%%%%%%%%%%%%%%%%%
%% switch language
\newcommand{\switchlanguage}{\iflanguage{english}{\selectlanguage{german}}{\selectlanguage{english}}}
%%%%%%%%%%%%%%%%%%%%%%%%%%%%%%%%%%%%%%%%%%%%%%%%%%%%%%%%%%%%%%%%%%%%%%
%% switch language
\newcommand{\ackname}{\iflanguage{english}{Acknowledgement}{\iflanguage{german}{Danksagung}{\langerrmessage}}}

%%%%%%%%%%%%%%%%%%%%%%%%%%%%%%%%%%%%%%%%%%%%%%%%%%%%%%%%%%%%%%%%%%%%%%
%% PDF keywords
%%%%%%%%%%%%%%%%%%%%%%%%%%%%%%%%%%%%%%%%%%%%%%%%%%%%%%%%%%%%%%%%%%%%%%
\AtBeginDocument{
  \ifthenelse{\equal{\mysecondauthor}{}}{%
    \newcommand{\internpdfauthorstring}{%
      \myfirstauthor%
    }
  }{%
    \ifthenelse{\equal{\mythirdauthor}{}}{%
      \newcommand{\internpdfauthorstring}{%
        \myfirstauthor, \mysecondauthor%
      }
    }{%
      \ifthenelse{\equal{\myfourthauthor}{}}{%
        \newcommand{\internpdfauthorstring}{%
          \myfirstauthor, \mysecondauthor, \mythirdauthor%
        }
      }{%
        \ifthenelse{\equal{\myfifthauthor}{}}{%
          \newcommand{\internpdfauthorstring}{%
            \myfirstauthor, \mysecondauthor, \mythirdauthor, \myfourthauthor%
          }
        }{%
          \ifthenelse{\equal{\mysixthauthor}{}}{%
            \newcommand{\internpdfauthorstring}{%
              \myfirstauthor, \mysecondauthor, \mythirdauthor, \myfourthauthor, \myfifthauthor%
            }
          }{%
            \newcommand{\internpdfauthorstring}{%
              \myfirstauthor, \mysecondauthor, \mythirdauthor, \myfourthauthor, \myfifthauthor, \mysixthauthor%
            }
          }%
        }%
      }%
    }%
  }%
  \providecommand{\mysubject}{}
  \ifnht{%
    \hypersetup{
      pdfauthor={\internpdfauthorstring},
      pdftitle={\mytitle},
      pdfsubject={\mysubject},
      pdfkeywords={\mykeywordsEN}
    }%
  }
}

%%%%%%%%%%%%%%%%%%%%%%%%%%%%%%%%%%%%%%%%%%%%%%%%%%%%%%%%%%%%%%%%%%%%%%
%%%%%%%%%%%%%%%%%%%%%%%%%%%%%%%%%%%%%%%%%%%%%%%%%%%%%%%%%%%%%%%%%%%%%%
%% Hyperlink color
% !TeX encoding = usascii
\ifnht{%
    \hypersetup{
        citebordercolor=ForestGreen,
        linkbordercolor=Crimson,
        urlbordercolor=blue,
        pdfborder={0 0 0.35 [1 2]},
        plainpages=false
    }
}

%%%%%%%%%%%%%%%%%%%%%%%%%%%%%%%%%%%%%%%%%%%%%%%%%%%%%%%%%%%%%%%%%%%%%%
%%%%%%%%%%%%%%%%%%%%%%%%%%%%%%%%%%%%%%%%%%%%%%%%%%%%%%%%%%%%%%%%%%%%%%
%% Figure definitions
% !TeX encoding = usascii
%%%%%%%%%%%%%%%%%%%%%%%%%%%%%%%%%%%%%%%%%%%%%%%%%%%%%%%%%%%%%%%%%%%%%%
%% Caption on top or at bottom?
%%   TRB guidelines: Caption at bottom
\providecommand{\figurecaptionandcontents}[2]{#2#1}
%%%%%%%%%%%%%%%%%%%%%%%%%%%%%%%%%%%%%%%%%%%%%%%%%%%%%%%%%%%%%%%%%%%%%%
%% Caption on top or at bottom for tables?
%%   TRB guidelines: Caption at top
\providecommand{\tablecaptionandcontents}[2]{#1#2}
%%%%%%%%%%%%%%%%%%%%%%%%%%%%%%%%%%%%%%%%%%%%%%%%%%%%%%%%%%%%%%%%%%%%%%
%% Use hrules?
%%   No
\newcommand{\figurerule}{}
\providecommand{\figurecontentsafter}{\end{\figurecontentsenv}\vspace{-2ex}}
%%%%%%%%%%%%%%%%%%%%%%%%%%%%%%%%%%%%%%%%%%%%%%%%%%%%%%%%%%%%%%%%%%%%%%
%% Input definition file
% !TeX encoding = usascii
%%%%%%%%%%%%%%%%%%%%%%%%%%%%%%%%%%%%%%%%%%%%%%%%%%%%%%%%%%%%%%%%%%%%%%
%% Caption on top or at bottom?
%%   Default: on top
\providecommand{\figurecaptionandcontents}[2]{#1#2}%
\ifthenelse{\equal{\figurecaptionandcontents{a}{b}}{ab}}{%
  \captionsetup*[figure]{position=above}%
  \captionsetup*[subfigure]{position=above}%
}{%
  \captionsetup*[figure]{position=below}%
  \captionsetup*[subfigure]{position=below}%
}%
%\captionsetup*[figure]{position=below}%
%%%%%%%%%%%%%%%%%%%%%%%%%%%%%%%%%%%%%%%%%%%%%%%%%%%%%%%%%%%%%%%%%%%%%%
%% Caption on top or at bottom for tables?
%%   Default: same as for figure
\providecommand{\tablecaptionandcontents}[2]{\figurecaptionandcontents{#1}{#2}}%
\ifthenelse{\equal{\tablecaptionandcontents{a}{b}}{ab}}{%
  \captionsetup*[table]{position=above}
  \captionsetup*[subtable]{position=above}%
}{%
  \captionsetup*[table]{position=below}%
  \captionsetup*[subtable]{position=below}%
}%
%%%%%%%%%%%%%%%%%%%%%%%%%%%%%%%%%%%%%%%%%%%%%%%%%%%%%%%%%%%%%%%%%%%%%%
%% Which environment to surround the figure with?
%% (If left empty, do nothing)
%%   Default: center
\providecommand{\figurecontentsenv}{center}
%%%%%%%%%%%%%%%%%%%%%%%%%%%%%%%%%%%%%%%%%%%%%%%%%%%%%%%%%%%%%%%%%%%%%%
%% Which environment to surround the table with?
%% (If left empty, do nothing)
%%   Default: \figurecontentsenv
\providecommand{\tablecontentsenv}{\figurecontentsenv}
%%%%%%%%%%%%%%%%%%%%%%%%%%%%%%%%%%%%%%%%%%%%%%%%%%%%%%%%%%%%%%%%%%%%%%
%% Which command to put before/after figure?
%%   Default: \begin{\figurecontentsenv} and \end{\figurecontentsenv}
%%            or nothing if \figurecontentsenv is empty
\providecommand{\figurecontentsbefore}{\begin{\figurecontentsenv}}
\providecommand{\figurecontentsafter}{\end{\figurecontentsenv}}
%%%%%%%%%%%%%%%%%%%%%%%%%%%%%%%%%%%%%%%%%%%%%%%%%%%%%%%%%%%%%%%%%%%%%%
%% Which command to put before/after table?
%%   Default: \begin{\tablecontentsenv} and \end{\tablecontentsenv}
%%            or nothing if \tablecontentsenv is empty
\providecommand{\tablecontentsbefore}{\begin{\tablecontentsenv}}
\providecommand{\tablecontentsafter}{\end{\tablecontentsenv}}
%%%%%%%%%%%%%%%%%%%%%%%%%%%%%%%%%%%%%%%%%%%%%%%%%%%%%%%%%%%%%%%%%%%%%%
%% Use hrules?
%%   Default: yes
\providecommand{\figurerule}{\hrule}
\providecommand{\tablerule}{\figurerule}
%%%%%%%%%%%%%%%%%%%%%%%%%%%%%%%%%%%%%%%%%%%%%%%%%%%%%%%%%%%%%%%%%%%%%%
%% Surrounds all figures
%%   Default: Just use it
\providecommand{\dofigure}[1]{#1}
\providecommand{\dotable}{\dofigure}
\providecommand{\doxmlfigure}{\dofigure}
%%%%%%%%%%%%%%%%%%%%%%%%%%%%%%%%%%%%%%%%%%%%%%%%%%%%%%%%%%%%%%%%%%%%%%
%% Rotate sideways figures and tables?
%%   Only works reliably for oneside layouts
\makeatletter
\if@twoside
\newcommand{\rotatesideways}{}
\else
\newcommand{\rotatesideways}{\AddThispageHook{\AddPageRotate{90}}}
\fi
\makeatother
%%%%%%%%%%%%%%%%%%%%%%%%%%%%%%%%%%%%%%%%%%%%%%%%%%%%%%%%%%%%%%%%%%%%%%
%% Figure definition:
%%   \createfigure
%%     [placement, default: tbp]
%%     {<short caption>}
%%     {<long caption>}
%%     {<\label{label}>}
%%     {<\includegraphics[...]{figure}>}
%%     {<source>or<>}
%%
%% \createfigure* puts figure across two columns in two-column mode
%%
%% The richfigure environment has been deprecated:  It was not used,
%% and capturing the figure contents did not work well.
\ifnwc{
  \NewDocumentCommand\createfigure{s O{tbp} m m m +m m}{%
    \dofigure{%
      \IfBooleanTF{#1}{\begin{figure*}[#2]}{\begin{figure}[#2]}%
        \ifthenelse{\equal{\figurecaptionandcontents{a}{b}}{ab}}{%
          \captionabove[#3]{#4}#5%
        }{}%
        \figurerule%
        \figurecontentsbefore%
          #6
        \figurecontentsafter%
        \ifthenelse%
          {\equal{#7}{}}%
          {}%
          {\begin{singlespace}\wordsource #7\end{singlespace}\smallskip}%
        \figurerule%
        \ifthenelse{\equal{\figurecaptionandcontents{a}{b}}{ba}}{%
          \captionbelow[#3]{#4}#5%
        }{}%
      \IfBooleanTF{#1}{\end{figure*}}{\end{figure}}%
    }%
  }
}
\ifwc{
  \NewDocumentCommand\createfigure{s O{tbp} m m m +m m}{%
  }
}
%%%%%%%%%%%%%%%%%%%%%%%%%%%%%%%%%%%%%%%%%%%%%%%%%%%%%%%%%%%%%%%%%%%%%%
%% Sideways figure definition
%%   \createsidewaysfigure
%%     {<short caption>}
%%     {<long caption>}
%%     {<\label{label}>}
%%     {<\includegraphics[...]{figure}>}
%%     {<source>or<>}
\newcommand{\createsidewaysfigure}[5]{%
  \ifelseht{%
    \createfigure{#1}{#2}{#3}{#4}{#5}%
  }{%
    \ifnwc{%
      \dofigure{%
        \begin{sidewaysfigure*}%
          \thisfloatcommand{\rotatesideways}%
          \ifthenelse{\equal{\figurecaptionandcontents{a}{b}}{ab}}{%
            \captionabove[#1]{#2}#3%
          }{}%
          \figurerule%
          \begin{center}%
            #4\\%
          \end{center}%
          \ifthenelse
            {\equal{#5}{}}
            {}
            {\begin{singlespace}\wordsource #5\end{singlespace}\smallskip}%
          \figurerule%
          \ifthenelse{\equal{\figurecaptionandcontents{a}{b}}{ba}}{%
            \captionbelow[#1]{#2}#3%
          }{}%
        \end{sidewaysfigure*}%
      }%
    }%
  }%
}
%%%%%%%%%%%%%%%%%%%%%%%%%%%%%%%%%%%%%%%%%%%%%%%%%%%%%%%%%%%%%%%%%%%%%%
%% Sub-figure:
%%   \createsubfigure
%%     {<caption>}
%%     {<\includegraphics[...]{figure}>}
%%     {<\label{label}>}
%%     {<\\>or<>}%
\newcommand{\createsubfigure}[4]{%
  \subcaptionbox{#1#3}{#2}\ #4
}
%%%%%%%%%%%%%%%%%%%%%%%%%%%%%%%%%%%%%%%%%%%%%%%%%%%%%%%%%%%%%%%%%%%%%%
%% Table definition (old style, code in [] can be omitted):
%%   \createtable
%%     [placement, default: tbp]
%%     {<short caption>}
%%     {<long caption>}
%%     {<\label{label}>}
%%     {<\begin{tabular}...\end{tabular}>}
%%     {<source>or<>}
%%
%% \createtable* puts table across two columns in two-column mode
%%
%% The richtable environment has been deprecated:  It was not used,
%% and capturing the contents did not work well.
\ifnwc{
  \NewDocumentCommand\createtable{s O{tbp} m m m +m m}{%
    \dotable{%
      \AsPicture{%
        \IfBooleanTF{#1}{\begin{table*}[#2]}{\begin{table}[#2]}%
          \ifthenelse{\equal{\tablecaptionandcontents{a}{b}}{ab}}{%
            \captionabove[#3]{#4}#5%
          }{}%
          \tablerule%
          \tablecontentsbefore%
            #6
          \tablecontentsafter%
          \ifthenelse%
            {\equal{#7}{}}%
            {}%
            {\begin{singlespace}\wordsource #7\end{singlespace}\smallskip}%
          \tablerule%
          \ifthenelse{\equal{\tablecaptionandcontents{a}{b}}{ba}}{%
            \captionbelow[#3]{#4}#5%
          }{}%
        \IfBooleanTF{#1}{\end{table*}}{\end{table}}%
      }%
    }%
  }
}
\ifwc{
  \NewDocumentCommand\createtable{s O{tbp} m m m +m m}{%
  }
}
%%%%%%%%%%%%%%%%%%%%%%%%%%%%%%%%%%%%%%%%%%%%%%%%%%%%%%%%%%%%%%%%%%%%%%
%% Sideways table definition
%%   \createsidewaystable
%%     {<short caption>}
%%     {<long caption>}
%%     {<\label{label}>}
%%     {<\begin{tabular}...\end{tabular}>}
%%     {<source>or<>}
\newcommand{\createsidewaystable}[5]{%
  \ifelseht{%
    \createtable{#1}{#2}{#3}{#4}{#5}%
  }{%
    \ifnwc{%
      \dotable{%
        \AsPicture{%
          \begin{sidewaystable*}%
          \thisfloatcommand{\rotatesideways}%
            \ifthenelse{\equal{\figurecaptionandcontents{a}{b}}{ab}}{%
              \captionabove[#1]{#2}#3%
            }{}%
            \tablerule%
            \begin{center}%
              #4\\%
            \end{center}%
            \ifthenelse
              {\equal{#5}{}}
              {}
              {\begin{singlespace}\wordsource #5\end{singlespace}\smallskip}%
            \tablerule%
            \ifthenelse{\equal{\tablecaptionandcontents{a}{b}}{ba}}{%
              \captionbelow[#1]{#2}#3%
            }{}%
          \end{sidewaystable*}%
        }%
      }%
    }%
  }%
}
%%%%%%%%%%%%%%%%%%%%%%%%%%%%%%%%%%%%%%%%%%%%%%%%%%%%%%%%%%%%%%%%%%%%%%
%% Sub-table:
%%   \createsubtable
%%     {<caption>}
%%     {<\begin{tabular}...\end{tabular}>}
%%     {<\label{label}>}
%%     {<\\>or<>}%
\newcommand{\createsubtable}[4]{%
  \createsubfigure{#1}{#2}{#3}{#4}%
}
%%%%%%%%%%%%%%%%%%%%%%%%%%%%%%%%%%%%%%%%%%%%%%%%%%%%%%%%%%%%%%%%%%%%%%
%% XML-figure
%%   \createxmlfigure
%%     [placement, default: tbp]
%%     {<short caption>}
%%     {<long caption>}
%%     {<\label{label}>}
%%     {<the/file/with/the/xml/code/to/include>}
%%     {<source>or<>}
%%
%% \createxmlfigure* puts figure across two columns in two-column mode
\ifnwc{
  \NewDocumentCommand\createxmlfigure{s O{tbp} m m m +m m}{%
    \doxmlfigure{%
      \IfBooleanTF{#1}{\begin{figure*}[#2]}{\begin{figure}[#2]}%
        \ifthenelse{\equal{\figurecaptionandcontents{a}{b}}{ab}}{%
          \captionabove[#3]{#4}#5%
        }{}%
        \figurerule%
        \lstinputlisting[language=XML]{#6}%
        \ifthenelse
          {\equal{#7}{}}
          {}
          {\begin{singlespace}\wordsource #7\end{singlespace}\smallskip}%
        \figurerule%
        \ifthenelse{\equal{\figurecaptionandcontents{a}{b}}{ba}}{%
          \captionbelow[#3]{#4}#5%
        }{}%
      \IfBooleanTF{#1}{\end{figure*}}{\end{figure}}%
    }%
  }%
}
\ifwc{
  \NewDocumentCommand\createxmlfigure{s O{tbp} m m m +m m}{%
  }%
}
%%%%%%%%%%%%%%%%%%%%%%%%%%%%%%%%%%%%%%%%%%%%%%%%%%%%%%%%%%%%%%%%%%%%%%
%% Algorithm definition:
%%   \createalgorithm
%%     [placement, default: tbp]
%%     {<short caption>}
%%     {<long caption>}
%%     {<\label{label}>}
%%     {<\includegraphics[...]{figure}>}
%%     {<source>or<>}
%%
%% \createalgorithm* puts figure across two columns in two-column mode
%%
%% The richfigure environment has been deprecated:  It was not used,
%% and capturing the figure contents did not work well.
\ifnwc{
  \NewDocumentCommand\createalgorithm{s O{tbp} m m m +m m}{%
    \dofigure{%
      \IfBooleanTF{#1}{\begin{algorithm*}[#2]}{\begin{algorithm}[#2]}%
        \captionabove[#3]{#4}#5%
          #6
        \ifthenelse%
          {\equal{#7}{}}%
          {}%
          {\begin{singlespace}\wordsource #7\end{singlespace}\smallskip}%
      \IfBooleanTF{#1}{\end{algorithm*}}{\end{algorithm}}%
    }%
  }
}
\ifwc{
  \NewDocumentCommand\createalgorithm{s O{tbp} m m m +m m}{%
  }
}


%%%%%%%%%%%%%%%%%%%%%%%%%%%%%%%%%%%%%%%%%%%%%%%%%%%%%%%%%%%%%%%%%%%%%%

%%%%%%%%%%%%%%%%%%%%%%%%%%%%%%%%%%%%%%%%%%%%%%%%%%%%%%%%%%%%%%%%%%%%%%
%% Titlepage and abstract definition
%% The elsarticle.cls includes the abstract in the title page
%% Hence, we wait until the \createabstract command before displaying
%% the title page
\newcommand{\createtitlepage}{}%
\newcommand{\createcontact}[7]{}%
\newcommand{\extractaddressfromcontact}[1]{%
  \renewcommand{\createcontact}[7]{%
      \ifne{##1}{\textbf{##1}\\##2\\\ifne{##3}{##3, }\removeword{##4}\\}%
  }%
  #1%
}%
\newcommand{\extractorganizationfromcontact}[1]{%
  \renewcommand{\createcontact}[7]{%
    ##2\ifne{##3}{, ##3}%
  }%
  \mdseries\hspace{-.5ex}#1%
}%
\newcommand{\extractemailfromcontact}[1]{%
  \renewcommand{\createcontact}[7]{##7}%
  #1%
}%
\newcommand{\createabstract}[1]{%
%% ===== ARTICLE TITLE (required), subtitle (optional) ==== %%
\title{\mytitle}
\providecommand{\mysubtitle}{}
\newcommand\semibfsf{\relax}

%% ===== SHORT TITLE (required for page headers) ========== %%
\providecommand\myshorttitle{\mytitle}


%% ===== AUTHOR INFORMATION (required) ==================== %%
%% Name, affiliation/institution, and email/contact for each
%% Add as many as necessary, separated by "\and":
\ifeqe{\mysecondauthor}{}{
  \newcommand\myauthors{\myfirstauthor}
  \author{
    {\myfirstauthor}
      \thanks{\protect\extractemailfromcontact{\protect\myfirstaddress}}
      \\\extractorganizationfromcontact{\myfirstaddress}
   % \and {\bfseries } \\
   % \and {\bfseries } \\
  }% End authors
}{
  \ifeqe{\mythirdauthor}{}{
    \newcommand\myauthors{\myfirstauthor{} and \mysecondauthor}
    \author{
      {\semibfsf\myfirstauthor}
        \thanks{\protect\extractemailfromcontact{\protect\myfirstaddress}}
        \\\extractorganizationfromcontact{\myfirstaddress}
      \and
      {\semibfsf\mysecondauthor}
        \\\extractorganizationfromcontact{\mysecondaddress}
     % \and {\bfseries } \\
    }% End authors
  }{
    \ifeqe{\myfourthauthor}{}{
      \newcommand\myauthors{\myfirstauthor{}, \mysecondauthor, and \mythirdauthor}
      \author{
        {\semibfsf\myfirstauthor}
          \thanks{\protect\extractemailfromcontact{\protect\myfirstaddress}}
          \\\extractorganizationfromcontact{\myfirstaddress}
        \and
        {\semibfsf\mysecondauthor}
          \\\extractorganizationfromcontact{\mysecondaddress}
        \and
        {\semibfsf\mythirdauthor}
          \\\extractorganizationfromcontact{\mythirdaddress}
       % \and {\bfseries } \\
      }% End authors
    }{
      \ifeqe{\myfifthauthor}{}{
        \newcommand\myauthors{\myfirstauthor{}, \mysecondauthor, \mythirdauthor, and \myfourthauthor}
        \author{
          {\semibfsf\myfirstauthor}
            \thanks{\protect\extractemailfromcontact{\protect\myfirstaddress}}
            \\\extractorganizationfromcontact{\myfirstaddress}
          \and
          {\semibfsf\mysecondauthor}
            \\\extractorganizationfromcontact{\mysecondaddress}
          \and
          {\semibfsf\mythirdauthor}
            \\\extractorganizationfromcontact{\mythirdaddress}
          \and
          {\semibfsf\myfourthauthor}
            \\\extractorganizationfromcontact{\myfourthaddress}
         % \and {\bfseries } \\
        }% End authors
      }{
        \ifeqe{\mysixthauthor}{}{
          \newcommand\myauthors{\myfirstauthor{}, \mysecondauthor, \mythirdauthor, \myfourthauthor, and \myfifthauthor}
          \author{
            {\semibfsf\myfirstauthor}
              \thanks{\protect\extractemailfromcontact{\protect\myfirstaddress}}
              \\\extractorganizationfromcontact{\myfirstaddress}
            \and
            {\semibfsf\mysecondauthor}
              \\\extractorganizationfromcontact{\mysecondaddress}
            \and
            {\semibfsf\mythirdauthor}
              \\\extractorganizationfromcontact{\mythirdaddress}
            \and
            {\semibfsf\myfourthauthor}
              \\\extractorganizationfromcontact{\myfourthaddress}
            \and
            {\semibfsf\myfifthauthor}
              \\\extractorganizationfromcontact{\myfifthaddress}
           % \and {\bfseries } \\
          }% End authors
        }{
          \newcommand\myauthors{\myfirstauthor{}, \mysecondauthor, \mythirdauthor, \myfourthauthor, \myfifthauthor, and \mysixthauthor}
          \author{
            {\semibfsf\myfirstauthor}
              \thanks{\protect\extractemailfromcontact{\protect\myfirstaddress}}
              \\\extractorganizationfromcontact{\myfirstaddress}
            \and
            {\semibfsf\mysecondauthor}
              \\\extractorganizationfromcontact{\mysecondaddress}
            \and
            {\semibfsf\mythirdauthor}
              \\\extractorganizationfromcontact{\mythirdaddress}
            \and
            {\semibfsf\myfourthauthor}
              \\\extractorganizationfromcontact{\myfourthaddress}
            \and
            {\semibfsf\myfifthauthor}
              \\\extractorganizationfromcontact{\myfifthaddress}
            \and
            {\semibfsf\mysixthauthor}
              \\\extractorganizationfromcontact{\mysixthaddress}
           % \and {\bfseries } \\
          }% End authors
        }
      }
    }
  }
}

\newcommand\mykeywordssep{\convertchar\mykeywordsEN{,}{;}}


\twocolumn[
\begin{@twocolumnfalse}
\maketitle
\end{@twocolumnfalse}
]
{\onehalfspacing\small\bfseries{}\noindent\ignorespaces#1\par}

\hrulefill
}

\hypersetup{%
			pdfproducer={Transportation Research Record},
			pdfstartpage=1,
			colorlinks=true,
			linkcolor=NavyBlue,
			citecolor=PineGreen,
			urlcolor=BrickRed,
}% END HYPERSETUP
%%%%%%%%%%%%%%%%%%%%%%%%%%%%%%%%%%%%%%%%%%%%%%%%%%%%%%%%%%%%%%%%%%%%%%
%% Custom reference command to support reference by name
\newcommand{\ncref}[1]{\namecref{#1}~\uncref{#1}}
\newcommand{\Ncref}[1]{\nameCref{#1}~\uNcref{#1}}
\newcommand{\uncref}[1]{\emph{\nameref{#1}}\xspace}
\newcommand{\uNcref}[1]{\emph{\nameref{#1}}\xspace}
\newcommand{\cnref}[1]{\uncref{#1}~\namecref{#1}\xspace}
\newcommand{\cNref}[1]{\uNcref{#1}~\nameCref{#1}\xspace}
\crefname{section}{section}{sections}%
\crefname{subsection}{section}{sections}%
\crefname{subsubsection}{section}{sections}%
