\section{Results}
\label{sec:results}

We test the four proposed dispatching strategies in the Zurich scenario with ten
runs per fleet size and strategy. Since the dispatchers rely on freeflow speeds in the network
for their routing when the simulation starts, we let each run perform 20 iterations
in which the dispatcher step by step senses the traffic conditions, e.g. how to
avoid traffic jams at peak hours.

[TODO REVIEW FROM HERE - results are NOT updated yet (Feedforward is still running with 10min period.. should be better)]

The dispatching stages of all algorithms are called once every 60 seconds in
simulated time, while the rebalancing periods for the feedforward and feedback
dispatcher are 60 seconds and 20 minutes, respectively.

For Zurich, the times with peak congestion and, hence, longest travel times are
from 6:30am to 9:00am and from 4:30pm to 6:30pm. In figure \ref{fig:mean_peak_waiting_times}
all trips by AV with departure times in these time windows are collected
and the mean waiting time is computed. As expected, the average
waiting time is decreasing with larger fleet sizes and higher availabiltiy of
vehicles. Almost over the whole range of fleet sizes the feedback algorithm
performs best, while the load-balancing heuristic yields the longest waiting
times.

\begin{figure}
\includegraphics[width=1.0\textwidth]{figures/mean_peak_waiting_times.pdf}
\caption{Average waiting time for an AV to arrive at peak times}
\label{fig:mean_peak_waiting_times}
\end{figure}

Figure \ref{fig:empty_rides} shows the percentage of fleet mileage that is driven
without a customer, either for pickup or rebalancing purposes. Clearly, the LP
algorithms, which both use rebalancing, have a higher share of empty mileage
that the non-rebalancing approaches. The heuristic approach manages to keep the
share lowest, since it mainly operates in a best-response state, where only the
shortest pickup trips are chosen. Remarkably, the total driven distance for all
dispatchers is very similar (Figure \ref{fig:total_distance}), which indicates
that the surplus of empty distance for the intelligent dispatchers does not stem
from inefficient movements, but rather effective movements towards the expected
customer demand for shorter waiting times.

[TODO: Do we need two plots here? Also a plot Total Distance <-> Relative Distance
would be possible, where one can traverse the fleet size along the graph]

\begin{figure}
\includegraphics[width=1.0\textwidth]{figures/empty_rides.pdf}
\caption{The fraction of distance that is driven by AVs without a passenger.}
\label{fig:empty_rides}
\end{figure}

\begin{figure}
\includegraphics[width=1.0\textwidth]{figures/total_distance.pdf}
\caption{The total distance that is driven by AVs, with and without passenger on-board.}
\label{fig:total_distance}
\end{figure}

Finally, figure \ref{fig:occupancy} shows the occupancy of the fleet for different
fleet sizes. Since in the 30h MATSim simulation no AV trips are registered in
the hours around midnight, it is possible to correct the resulting 30h occupancy
rate to one that is based on a 24h day. As can be seen, the occupancy of all
fleet dispatchers exceeds the 8\% that is common today. In general, one can say
that the dispatching algorithm has only little influence on fleet occupancy. The
differences lie in the range of 0.5\% between the best and worst performing algorithm,
which are the LP Feedback dispatcher and the load-balancing heuristic, respectively.
Nevertheless, one can see that the occupancy of the latter is systematically lowest.

\begin{figure}
\includegraphics[width=1.0\textwidth]{figures/occupancy.pdf}
\caption{The occupancy of the AV fleet for different flet sizes.}
\label{fig:occupancy}
\end{figure}

\subsection{Cost Analysis}
\label{sec:cost_analysis}

Based on a paper Bösch et al. \cite{cost_paper} the costs of operating the simulated
AV services are computed. Specifically, by providing their calculator with key
figures of the operator (among them the occupancy, the share of empty rides, the
average travel distance) the price that the operator would at least need to ask
a customer per kilometer if a profit margin of at least 3\% is targeted. The calculation
is based on a detailed analysis of running and fixed costs. Figure~\ref{fig:passenger_price}
shows the results from this analysis. Unsurprisingly, the price that needs to be
imposed on the customer increases with larger fleet sizes. However, the increase
is stronger for the load-balancing heuristic than for any other distpaching strategy.
Therefore, with the same fleet being available to an operator, he would be able
to offer the service for almost 0.10 CHF less per kilometer than before or save
this amount of money.

\begin{figure}
\includegraphics[width=1.0\textwidth]{figures/01_passenger_price.pdf}
\caption{The minimum customer prices that an AV operator needs to charge the customer
in order to have a win margin of at least 3\%.}
\label{fig:passenger_price}
\end{figure}

Compared to the average running costs of driving a private car in Switzerland
(around 0.17 CHF/km) or using public transit (around 0.25 CHF/km) [TODO CITATIONS]
the computed prices still seem rather high. Compared to conventional taxi operators,
however, the price is extremely low (around 6 CHF/km). Therefore, it is imaginable
that the AV service would still be attractive for a large group of people, for
which a conventional taxi would be too expensive on a daily basis, but an AV would
make such travels affordable.

However, the attractiveness of an AV service does not only depend the price itself,
but also on the attitudes of the people towards the service. One key component to
the acceptance of an AMoD system is the waiting times that customers need to endure.
Figure \ref{fig:time_vs_price} combines the key results from our simulations. There,
the price that a specific operator configuration (fleet size and dispatcher) is
displayed in comparison to the waiting time that this operator can offer.
Assuming that, for instance, a waiting time of five minutes is tolerable, the
operator could offer a satisfactory service for around 0.45 CHF with the feedback
dispatcher, while he would need to charge 0.50 CHF with the simple load-balancing
heuristic. The better the level of service of the operator is ought to be, the larger
this margin becomes.

\begin{figure}
\includegraphics[width=1.0\textwidth]{figures/time_vs_price.pdf}
\caption{Time vs. Price}
\label{fig:time_vs_price}
\end{figure}
