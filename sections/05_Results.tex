\section{Results}
\label{sec:results}

We test the four proposed dispatching strategies in the Zurich scenario with ten
runs per fleet size and strategy. The dispatching stages of all algorithms are called once every 60 seconds in
simulated time, while the rebalancing periods for the feedforward and feedback
dispatcher are five minutes and 20 minutes, respectively. Those values have been
obtained from prior simulation runs.

For Zurich, the times with peak congestion and, hence, longest travel times
are from 6:30 am to 9:00 am and from 4:30 pm to 6:30 pm. Figure \ref{fig:mean_peak_waiting_times}
shows the average customer wait time over the whole day (dashed) and just for peak hours (solid).
While the simple heuristic approach consistently yields the longest wait times
for any fleet size, the feedback dispatcher performs best. The bipartite matching
performs in between, since it is based on an optimal request assignment, but does
not do any rebalancing. Since both algorithms rely on rebalancing, the two linear
programs have very similar performance. The feedback algorithm seems to have a
slight advantage, especially for average wait time over the whole day, because it
is able to react to the observed demand more precisely. Assuming that 5 minutes at
peak times are an acceptable wait time, that value is
achieved with a fleet of 10,000 vehicles for the heuristic, but with only 8,700
for the feedback dispatcher.

The measured wait times in Figure \ref{fig:mean_peak_waiting_times} can be compared
with the theoretical fleet sizing in Figure \ref{fig:performanceavailability}: For the
displayed fleet sizes a clear correspondance between wait time and the availability
can be observed.

\captionsetup[subfigure]{width=0.9\textwidth}

\begin{figure}
    \centering
    \begin{subfigure}[t]{0.495\textwidth}
        \includegraphics[width=1.0\textwidth]{figures/mean_peak_waiting_times.pdf}
        \caption{Average waiting time for an AV to arrive at peak times (solid) and over the entire day (dashed)}
        \label{fig:mean_peak_waiting_times}
    \end{subfigure}\hfill
    \begin{subfigure}[t]{0.495\textwidth}
        \includegraphics[width=1.0\textwidth]{figures/availability.pdf}
        \caption{Probability of at least one AV being available whenever a request comes in}
        \label{fig:performanceavailability}
    \end{subfigure}
    \caption{Fleet performance metrics for different fleet sizes}
\end{figure}

%\begin{figure}
%\includegraphics[width=1.0\textwidth]{figures/mean_peak_waiting_times.pdf}
%\caption{Average waiting time for an AV to arrive at peak times (solid) and over the entire day (dashed)}
%\label{fig:mean_peak_waiting_times}
%\end{figure}

Figure \ref{fig:distances} shows the distances that different service configurations
produce. On the left side the customer distance is shown, which stays constant
over all runs, while one can see that the pickup distance (middle, light) is decreasing
with larger fleet sizes and thus higher availability of vehicles. For the dispatchers
with rebalancing one can see that they add a surplus of mileage for rebalancing (right, dark)
such that the overall driven distance is rather stabilized over different fleet sizes.
This added mileage is used to provide the shorter travel times as presented above.
One can see that with similar wait times the feedback dispatcher operates more economically
by saving mileage compared to the feedforward algorithm.

\begin{figure}
\includegraphics[width=1.0\textwidth]{figures/distances.pdf}
\caption{Driven accumulated distances for different fleet sizes. From left to right:
Customer distance (dark), empty pickup distance (light), empty rebalancing distance (dark).}
\label{fig:distances}
\end{figure}

Finally, the occupancy of the fleet is measured. For a fleet size of 6,000
vehicles, they are busy serving a passenger for around 4.8h per day, while
this value drops to 2.16h for the maximum simulated fleet size of 18,000. In both cases,
those numbers exceed the average 1.32h in Switzerland \cite{Bosch2016a}.

\subsection{Financial Analysis}
\label{sec:cost_analysis}

Based on the cost calculator for fleets of automated vehicles by Bösch et al. \cite{Bosch2016a},
 the costs of operating the AV services are computed from a number of key figures
 such as the occupancy and share of empty rides. In Figure~\ref{fig:passenger_price}
 the resulting price per (revenue) vehicle kilometer including a profit margin of 3\% is shown. One can observe that the
 price increases with the fleet size, which can be explained by lower occupancy rates.
 It should however also be pointed out that the required price is different among the algorithms.
 The heuristic operates at the lowest costs while LP Feedforward is most expensive throughout all fleet sizes.


\captionsetup[subfigure]{width=0.9\textwidth}

\begin{figure}
    \centering
    \begin{subfigure}[t]{0.495\textwidth}
        \includegraphics[width=1.0\textwidth]{figures/01_passenger_price.pdf}
        \caption{Minimum customer prices that an AV operator needs to charge with a profit margin of 3\%}
        \label{fig:passenger_price}
    \end{subfigure}\hfill
    \begin{subfigure}[t]{0.495\textwidth}
        \includegraphics[width=1.0\textwidth]{figures/time_vs_price.pdf}
        \caption{Comparison plot of offered wait times and minimum service prices for the
        simulated fleet configurations}
        \label{fig:time_vs_price}
    \end{subfigure}
    \caption{Analysis of fleet configurations from the customer perspective}
\end{figure}

Nevertheless, the wait times decrease with an increasing fleet size. The trade-off
between price per vehicle kilometer and wait times in the peak hours is therefore depicted in
Figure \ref{fig:time_vs_price}. For lower wait times LP Feedback requires the
least prices. Above 5 minutes however, Bipartite matching becomes more efficient
in terms of costs per vehicle kilometer.


Compared to the price of a taxi operator in Zurich (base price 8 CHF plus 5 CHF/km, \cite{StadtZurich2014})
the computed prices are extremely low. Hence, an automated service would clearly
push conventional taxi operators out of the market. The variable costs of a today's
private vehicle (0.26 CHF/km, \cite{TCS2016}) are lower than the calculated prices for the AMoD services,
independent of the algorithm. Considering the full costs of a private vehicle which amount to 0.7 CHF/km \cite{TCS2016} however, it can be concluded that AMoD services are only more expensive for fleet sizes of around 11,500-12,000
in Zurich, depending on the algorithm. If customers are further willing to spend on average four minutes
during peak hours for an AV taxi, the prices are similar to the full costs of a standard car.
Nonetheless, compared to (subsidized) prices for mass transit (0.25 CHF/passenger kilometer, \cite{Bosch2016a}), the services are more expensive if they have the same occupation rate as today's private cars (approx. 1.4 passengers \cite{Bosch2016a}).

Therefore, the proposed AV services are cost-wise highly attractive for car (and taxi) users, but may
not be able to compete with subsidized mass transit. On the other hand, AVs
allow for more direct trips and thus for savings in travel time. Ongoing studies analyse
how these affect the attractiveness of AMoD services \cite{ Becker2017}. It is further expected
that lower wait times will have a positive effect on the occupancy rates of a service
and thus reduce the cost per vehicle kilometer.

%The expected wait time and the price of the service are the two decisive factors
%for customers deciding to use one or another operator, therefore it is important
%to compare these variables for different control methods. Figure \ref{fig:time_vs_price}
%shows the price that a specific operator configuration (fleet size and dispatcher)
%%needs to charge to be profitable in comparison to the wait time that this operator is offering.
%At a wait time of five minutes an operator would be able to offer a satisfactory service for
%around 0.45 CHF with the feedback dispatcher, while he would need to charge 0.50 CHF
%with the simple load-balancing heuristic.

%The better the level of service of the operator is, the larger this margin becomes.
