The performance of four different dispatching and rebalancing algorithms for the
control of an automated mobility-on-demand system is evaluated in a simulation environment.
The case study conducted with an agent-based simulation scenario of the city of Zurich
shows that the choice of an intelligent rebalancing algorithm decreases the average
 wait time in the system. For a wait time of four minutes at peak hours the best performing algorithm requires the same price
per vehicle kilometer as a private car today.
The results indicate that shared mobility systems of automated vehicles will reach higher occupancy rates than conventional private cars.
