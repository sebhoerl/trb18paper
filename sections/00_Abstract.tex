The performance of four different dispatching and rebalancing algorithms for the
control of an automated mobility-on-demand system is evaluated in a simulation environment.
The case study conducted with an agent-based simulation scenario of the city of Zurich
shows that the right choice of control algorithm not only reduces customer wait
times, but also offers large economic benefits to the operator. For an average
wait time of four minutes at peak hours the most performant algorithm requires the same price
per vehicle kilometer as a private car today.
The results show that such a automated mobility on demand services can be offered
while maintaining a higher fleet occupancy than with conventional private cars. Simulation
also confirms that the application of intelligent rebalancing algorithms decreases
the average wait time in the system.
