The performance of four different dispatching and rebalancing algorithms for the
control of an autonomous mobility-on-demand system is evaluated in simulation.
The case study conducted on an agent-based simulation scenario of the city of Zurich
shows that the right choice of control algorithm not only minimzes customer waiting
times, but also offers large economical benefits to the operator. For an average
waiting time at peak hours of five minutes the most performant algorithm would allow
the operator to offer his service for around 0.45 CHF per km, which is more expensive
than using a private car today. Yet it is significantly cheaper than a conventional
taxi. The results show that autonomous mobility-on-demand service can be offered
while maintaining a higher fleet occupancy than with private cars today. Simulation
also confirms that the application of intelligent rebalancing algorithms does decrease
the average wait time in the system, but does not necessarily increase the total
amount of miles driven.
