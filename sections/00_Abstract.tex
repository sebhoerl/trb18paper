The performance of four different dispatching and rebalancing algorithms for the control of an Autonomous Mobility On-Demand system is tested. The case study, which is based on an agent-based simulation scenario of the city of Zurich, shows, that the right choice of control algorithm not only minimizes customer waiting times but also offers large economic benefits to the operator. For an average waiting time at peak hours of five minutes, the most performant algorithm would allow the operator to offer his service for around 0.45 CHF, which is more expensive than using a private car today, but substantially cheaper than a conventional taxi. The results show that this service can be offered while maintaining a higher fleet occupancy than can be observed for private cars today and that the application of intelligent rebalancing algorithms is increasing
the share of miles driven without a customer but does not necessarily increase the total amount of miles driven.