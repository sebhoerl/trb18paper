The performance of four different dispatching and rebalancing algorithms for the
<<<<<<< HEAD
control of an autonomous mobility-on-demand system is evaluated in simulation.
=======
control of an Automated Mobility On-Demand system is evaluated in simulation.
>>>>>>> master
The case study conducted on an agent-based simulation scenario of the city of Zurich
shows that the right choice of control algorithm not only minimzes customer waiting
times, but also offers large economic benefits to the operator. For an average
waiting time at peak hours of five minutes the most performant algorithm would allow
<<<<<<< HEAD
the operator to offer his service for around 0.45 CHF per km, which is more expensive
than using a private car today. Yet it is significantly cheaper than a conventional
taxi. The results show that autonomous mobility-on-demand service can be offered
=======
the operator to offer his service for around 0.45 CHF per km, which is cheaper than
the average full costs of a private car and substantially cheaper than a conventional
taxi. The results show that such a automated mobility on demand services can be offered
>>>>>>> master
while maintaining a higher fleet occupancy than with private cars today. Simulation
also confirms that the application of intelligent rebalancing algorithms decreases
the average wait time in the system.

[TODO: REWORK FINANCIAL ANALYSIS AND ABSTRACT!!! -> Felix]
