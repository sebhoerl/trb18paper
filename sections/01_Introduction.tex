\section{Introduction}

With growing populations in cities and increasing urbanization, many of today's transportation systems approach their performance limits. In Switzerland the distance travelled per year in private motorized vehicles rose from $82,014 ~ \textnormal{Mkm}$ in 2005 to $96'467 ~\textnormal{Mkm}$ in 2015. Also the distance traveled in trams, trolley buses and buses rose from $3,858 ~ \textnormal{Mkm}$ to $4,397 ~ \textnormal{Mkm}$ in the same time span \cite{BFSVerkehrsvolumen}. The external costs induced by road and rail traffic are rising as well, for example did the costs by noise increase by almost $20 \%$ from 2005 to 2009 in Switzerland \cite{AREexterneKosten}.

Many technologies and policies are explored in research and tested in practice to mitigate the negative effects of the increasing traffic demand, e.g. road-pricing, incentives for increased bicycle usage, improved combustion engines, dynamic traffic assignment. In this article we focus on one particularly promising approach, that promises to combine the advantages of personal mobility (i.e. flexiblity, availability, comfort and speed) with the advantages of public transport (low external and environmental costs, ability to transport large passenger volumes).

In shared-vehicle mobility on demand systems, passengers share a fleet of shared vehicles, e.g. cars or bicycles which are available on demand. In two-way mobility on demand systems vehicles are taken from the same location as they are returned to. Examples of two-way mobility-on-demand systems are ``Mobility'' in Switzerland or ``zipcar'' in the United States. While these systems are very popular, they can't be used if the origin and destination of a trip are different or rest idle for an extended amount of time between the trip to and from a destination. In two-way mobility on demand systems, origin and destination of a trip do not need to be equal. For instance in bicycle sharing schemes, users can pickup a bicycle at a set of stations distributed in the city, use it for their trip and drop it at another station. However, the asymmetric spatio-temporal behavior of travel demand causes these systems to get imbalanced, i.e. certain areas or stations get depleted of vehicles while other stations do not have enough empty spots to allow clients to return their vehicles. Bicycle sharing schemes use redistribution trucks to transport bicycles from full to empty stations and there exists research on how to perform this redistribution of bicycles in an effective manner or using price incentives, e.g. \cite{ruch2014rule}. However, the main limitation of mobility on demand schemes is their tendence to get imbalanced quickly resulting in largely decreased service level and increased cost.



\subsection{Autonomous Mobility on Demand Systems}

Autonomous two-way mobility on demand sytems (AMoD) are mobility on demand systems with vehicles that can drive autonomously, i.e. without a driver. This property allows to pickup customers at arbitrary locations in the city and imbalance the system automatically. It can be seen as the key feature to allow sharing.

AMoD systems have been studied in literature both from an operational as well as a transportation research perspective. There are studies on the effects of AMoD systems when introduced at large scale in cities, on the fleet sizing for given traffic scenarios and on the fleet and operational management of the fleets.

In this work we present four main contributions. Firstly we provide novel simulation results of unmatched accuracy for the city of Zurich, our simulations are based on the agent-based traffic simulation MATSim and include ..... XYZ Sebastian please include. Secondly we compare the performance of the fleet under different operational principles and fleet management algorithms published in literature and show that the algorithms used for dispatching and rebalancing of vehicles are key factors of performance. We further introduce metrics that allow to assess the performance of an AMoD systems in terms both service level and operational cost. Lastly we verify theoretical results on fleet sizing introduced in \cite{spieser2014toward} for the city of Zurich. In \ref{subs:literatureResearch} we present related research, then we introduce performance metrics and control strategies for AMoD systems in \ref{sec:background}. We present simulation results for the city of Zurich in \ref{sec:staticSimulations} and conclude our findings in \ref{sec:Conclusion}.


\subsection{Literature Research}
\label{subs:literatureResearch}

The work in \cite{spieser2014toward} presents a systematic approach to the design of an autonomous mobility on demand system in Singapore. The authors consider the case where the entire travel demand of the city of Singapore would be covered by autonomous shared vehicles. Analytic results are used to compute both the minimal number of vehicles needed to stabilize the number of open requests as well as the amount of vehicles that is needed to provide an acceptable level of service. The authors conclude that the travel demands of the entire population of Singapore could be served with a fleet of $300,000$ vehicles and maximum mean wait times of approximately $15~ \textnormal{mins}$ which corresponds to a sharing factor of $\approx 4$. The results of simulations confirm the theoretical approximations but the scope and granularity of the simulations is not commented and it is not clear whether traffic is adequately taken into account. In addition to the analysis on wait times the authors present an an analysis of the total cost of mobility of an AMoD system compared to the cost of owning a private vehicle. They conclude that the cost reduction per $km$ gained when switching from private car ownership to using an AMoD system is $47~ \%$, $48~ \%$ in Singapore, the United States respectively. The study does not compare different fleet operation algorithms.

In \cite{marczuk2015autonomous} a case study is presented where no private cars can enter the central business district of Singapore. A total of $25,525$ trips is served by autonomous taxis which operate either in station-based or free-floating scheme. In the station-based scheme a set of fixed stations exists where the cars return to after completion of a trip. In the free-floating scheme the cars remain parked at the destination of trips. The simulations are based on the SimMobility agent based simulation platform and include twelve different fleet sizes from $2,000$ to $7,500$ vehicles. The authors conclude that the free-floating scheme can serve $90\%$ of the demand at the maximum fleet size whereas the station-based model can only serve $68 \% $ of the demand. Furthermore they observe mean customer wait times that saturate at approximately $2.5 ~ \textnormal{mins}$ and $6,000$ vehicles. While the station-based and free-floating concepts are compared, a detailed comparison of dispatching and rebalancing operation strategies is not presented. Furthermore results on the fleet efficiency and performance are not shown as well as a more detailed analysis of the wait times (e.g. different wait time quantiles). As the CBD of Singapore attracts much more than $25,525$ trips during a day it would also be interesting to see the results with the full number of trips taking into account routing policies and congestion levels in the city.

\cite{fagnant2015dynamic} presents a case study for Austin, Texas which focuses on the use of shared autonomous vehicles with ride-sharing capabilities, i.e. vehicles that can transport more than one customer under some circumstances. The scenario presented on vehicles with unit capacity includes a fleet of $1,715$ vehicles that serve $56,324$ person-trips. Assuming $3.02$ trips per person and day this results in a sharing factor of $10.87$ at average wait times of $1.18 ~ \textnormal{mins}$ and $4.49 ~ \textnormal{mins}$ during peak hour. The authors present also a scaled version of the case including ride-sharing where $11.1\%$ of the trips within the central region of Austin (the ``Geofence'') are served by $9,037$ vehicles suggesting that a fleet of $81,414$ vehicles could serve the entire population of that area. While the authors do not compare different algorithms for fleet operation and base their fleet size estimation on simulation results, they provide a financial analysis of the entrepreneurial viability for a potential shared autonomous taxi operator.

\cite{zachariah2014uncongested} present a case study for New Jersey also focused on the potential for ride-sharing. The trips generated by a population of $8,791,894$ individuals in New Jersey are covered by walking and biking if the distance is less than a mile. All other trips are either served by the New Jersey train system, by autonomous taxis or both. The study concludes that the ride-sharing potential is large, especially during rush-hour and autonomous vehicles could significantly reduce congestion levels in the city. The required fleet size is not commented as well as the influence of the rebalancing and dispatching strategy for the fleet.

In \cite{martinez2017assessing} the authors present a study on the effects of introducting autonomous taxis and autonomous shared taxis to the city of Lisbon, Portugal. The agent-based simulation includes 1.2 million trips and three scenarios: a baseline scenario showing the current situation and two scenarios where private car, taxi and bus trips are replaced by autonomous taxis and autonomous taxis and shared taxis respectively. The fleet size of autonomous (shared) taxis is set at $4.8\%$ of the baseline vehicle fleet. In these scenarios about 50-70 \% of trips are serviced by the autonomous (shared) taxis which increases the vehicle occupancy from $50 ~ \textnormal{mins}$ to $12.87 ~ h$ on average per day. The authors conclude a decrease in cost by $55 \%$ per kilometer, highly increased transportation accessibility in the city and carbon emission reductions of almost $40\%$. The simulation does not consider the changes on traffic density parameters resulting from self-driving vehicles. Furthermore the demand choice of the agents is static and according to preset parameters. Finally the fleet control (rebalancing and dispatching) for the (shared) autonomous taxis is implemented based on heuristics and a local gradient based optimization method.

Boesch et al \cite{boesch2016autonomous} investigate a scenario of the greater Zurich region in Switzerland. They use a demand pattern for private vehicles generated with MATSim: $1.3$ million  private vehicle users out of a total of $2.1$ million agents generate $3.6$ million trips. This demand profile generated with the co-evolutionary algorithm inherent to MATSim is then post-processed in a static simulation where $1-10 \%$ of the car trips are served by $10-100 \%$ of the total number of substituted users. The authors conclude that approximately $30 ~ \%$ of the substituted fleet can serve almost $100 \%$ of the substituted requests within less than $10 ~ \textnormal{mins}$ wait time. If this threshold wait time is surpassed, then the request is dropped. The results present the first such study for the city of Zurich in Switzerland. Its limitations are that no fleet management for the autonomous vehicles is considered, i.e. no rebalancing or dispatching is taking place, furthermore network routing is not considered, travel times are based on Euclidean distance and a scaling factor. The demand profile is static and does not vary depending on service times, congestion rates and performance of the modes.

In contrast to the study for Zurich presented above, a case study for Berlin presented in \cite{bischoff2016simulation} takes into account dynamic demand. It considers a city-wide replacement of private vehicles with autonomous taxis. The dispatching of the car works according to a policy that distinguishes between oversupply (more available vehicles than open requests) and undersupply and matches the closest vehicle to an appearing request, the next available vehicle to the closest request respectively. Using this strategy named XYY in our work, the authors are able to serve $4.7$ million requests generated by $1.1$ million car users with a fleet of $100,000$ autonomous vehicles. The recorded average wait time for this case is about $2.5 \%$ minutes and the $95\%$ quantile approximately $8.5$ minutes. The resulting sharing factor is approximately $10$ to $12$. The study is one of the first large-scale dynamic simulations of a shared autonomous taxi system, however it does not consider different rebalancing and dispatching strategies and it does not rigorously evaluate the performance metrics of the autonomous vehicle fleet as we do i
XZY
