\section{Introduction}

The rapid techological development in recent years has led to the point where
automated vehicles are tested in various pilot projects around the world, e.g by nutonomy Inc. in Singapore \citep{ackerman2017hail}. They promise to increase road capacity and speeds
[TODO include this cite tientrakool, friedrich] and would give access to mobility to formerly
inhibited user groups ( TODO cite Victoria). On the flipside an increase of vehicle
miles travelled (VMT) is expected due to empty rides (cite Litman), and the general increase
of users has the potential to clog road in the urban environment even more than today (cite Becker, Meyer).
Hence, the net effects on
the transport system, environment and society are unclear. Simulations, such as
the one presented in the work at hand can help to better understand the impact
of future developments in vehicle automation.

A number of studies in recent years debated the feasibility of an autonomous
mobility on demand (AMoD) system (see Related Research). With such a system travellers would not need to own their own car, but could call an automated vehicle [AV] to pick them up at any location and bring them to their desired destination. For the customer this would offer the comfortability of an individual taxi service
for a fraction of today's cost. It is predicted that the costs of using the
service on a daily basis heavily compete with privately owned cars and even
public transit, depending on the use scenario [TODO cite cost paper].

The success of an AV operator would depend on the pricing of his service
as well as the wait and travel times that he is able to offer. While high prices
may restrict the user group drastically, long wait times may have the same effect
if they make travelling less predictable than before. Both quantities are inherently
linked by the way the fleet is operated: If wait times should be minimized, vehicles
should be at all times present where the demand is expected. This makes it necessary to to relocate them without a passenger on-board, which directly translates to costs for the operator. Furthermore both quantities are also linked to the vehicle fleet size that heavily influences both cost and wait times.

In the present study we contriubte to research on AMoD system as follows: We
(a) present a simulation scenario of a fleet of automated taxis for Zurich, Switzerland,
based on the MATSim framework (cite Horni), we (b) test and compare four different dispatching
and rebalancing algorithms from literature for different fleet sizes, (c) analyse the results
in terms of customer acceptance and (d) compare our results with theoretical
predictions for fleet sizing.

The remainder is structured as follows: First, an overview of related search is
given, then the simulaton scenario and environment are introduced, as well as the
proposed fleet control algorithms. Thereafter, simulaton results are presented and
analysed, followed by a discussion of our findings.
