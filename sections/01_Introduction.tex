\section{Introduction}

The rapid techological development in recent years has led to the point where
automated vehicles are tested in pilot projects around the world \citep{ackerman2017hail}.
They promise to increase road
capacity and speeds \citep{Tientrakool2011,Friedrich2015} and would give access to mobility to formerly
inhibited user groups \citep{Truong2017}. On the flipside an increase of vehicle
miles travelled (VMT) is expected due to empty rides \citep{Litman2014}, and the general increase
of users has the potential to clog road in the urban environment even more than today \citep{Meyer2017}.
Hence, the net effects on
the transport system, environment and society are unclear. Simulations, such as
the one presented in the work at hand can help to better understand the impact
of future developments in vehicle automation.

A number of studies in recent years debated the feasibility of an automated
mobility on demand (AMoD) system (see Related Research). With such a system travelers
would not need to own their own car, but could call an automated vehicle [AV] to pick
them up at any location and bring them to their desired destination. For the customer
this would offer the comfortability of an individual taxi service
for a fraction of today's cost. It is predicted that the costs of using the
service on a daily basis compete with privately owned cars and even
public transit \citep{Bosch2016a}.

The success of an AV operator would depend on the pricing of his service
as well as the wait and travel times that he is able to offer. While high prices
may restrict the user group drastically, long wait times may have the same effect
if they make traveling less predictable than before. Both quantities are inherently
linked by the way the fleet is operated: If wait times should be minimized, vehicles
should be at all times present where the demand is expected. This makes it necessary to
relocate them without a passenger on-board, which directly translates to costs for
the operator. Furthermore both quantities are also linked to the vehicle fleet
size that heavily influences both cost and wait times.

In the present study we contribute to research on AMoD system as follows: We
(a) present a simulation scenario of a fleet of automated taxis for Zurich, Switzerland,
based on the MATSim framework \citep{Horni2015}, we (b) test and compare four different dispatching
and rebalancing algorithms from literature for different fleet sizes, (c) analyse the results
in terms of customer acceptance and (d) compare our results with theoretical
predictions for fleet sizing.

% Commented out to reduce words. Do we need this? /sh

%The remainder is structured as follows: First, an overview of related search is
%given, then the simulaton scenario and environment are introduced, as well as the
%proposed fleet control algorithms. Thereafter, simulaton results are presented and
%analysed, followed by a discussion of our findings.
