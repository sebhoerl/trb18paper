\section{Control of an AMoD System}
\label{sec:background}

An AMoD service can only be maintained if a sufficient number of customers
wants to use the service, such that it is profitable for the operator. While a multitude
of factors influences the attractiveness of the service (perhaps
multimedia offers in the vehicle, the quality of Wifi, ...) the authors assume
two key properties: The time that passes between a customer making a request
and a vehicle arriving (i.e. the wait time) and the price that is charged to
the customer. All else being equal, an operator that can offer the shortest wait
times at the lowest price will attract more customers than his competitors.

We focus on two main ways for operators to influence the service level of their system:

\begin{itemize}
\item The \textbf{fleet size} can be increased. In general, this should lead to
a decrease of wait time, because the availability of vehicles improves. However,
having a larger number of vehicles imposes higher fixed costs that would need to be
balanced by higher demand. In general, adding more vehicles to the fleet can be
regarded as a long-term investment that cannot be altered on a daily basis.
\item The \textbf{fleet control} can be optimized. Since in an AMoD system it is
asssumed that any vehicle can be tracked and controlled online, intelligent fleet
control algorithms can be used to minimize the wait times, but also to minimize the
driven distance in order to reduce operational cost. Applying the proper algorithm is a
less costly intervention than increasing the fleet size.
\end{itemize}

In the presented experiments both components are investigated by comparing a number
of control algorithms for fleets of varying sizes.

\subsection{Problem Statement}

For the algorithmic improvement of the fleet management, the authors distinguish
between two stages:

\begin{itemize}
\item The \textbf{dispatching strategy} decides on how to serve the demand, i.e.
how to match the open customer requests with the available vehicles. At any time the dispatcher can send tasks to pickup a specific customer to any vehicle that is not currently having a customer on board (since we do not consider ride-sharing with multiple customers). Also a reassignment of a previously assigned
vehicle to another request is possible.
\item The \textbf{rebalancing strategy} decides on where to send vehicles when they
are not in use and the low demand allows for supplementary movements of the vehicles.
The task of the rebalancer is to anticipate future requests and position vehicles
such that they are able to optimally react to the expected future demand.
\end{itemize}

Hence, vehicles will produce three kinds of mileage:

\begin{itemize}
\item \textbf{Empty pickup mileage} is produced when an AV is dispatched
to a request and is driving to the pick-up location. It is the mileage that needs
to be covered in order to serve the customer in any way and may be minimized
by an intelligent dispatching algorithm.
\item \textbf{Empty rebalancing mileage} is produced when an AV is sent
to a different location where demand is expected. An ideal operator would
exchange all the pickup mileage in the system against rebalancing mileage, i.e. the operator would always send empty vehicles before an actual request turns up.
\item \textbf{Customer mileage} is produced with a customer on board. This mileage does only depend on the routing of the cars. In any combination of fleet size and
control algorithm, this mileage stays constant, assuming that the origin-destination relations of the customers do not vary.
\end{itemize}

Mileage for maintenance and recharging is not further considered in this paper and subject to future research. 
Assuming a common pricing scheme that defines a price per distance, the customer mileage
is the only component that produces a benefit for the operator. All other mileage
can directly be translated into costs and should therefore be minimized. For general
demand patterns, however, it cannot be driven to zero. Spieser et al. \cite{spieser2014toward}
show that it is bounded below by the earth mover's distance \cite{levina2001earth}, which is a measure
of how different the distributions of trip origins and destinations are \cite{ruschendorf1985wasserstein}.

The objectives for a fleet management algorithm can therefore be defined as:

\begin{enumerate}
\item Minimize the total pickup distance given the non-optimal locations of the vehicles (dispatcher)
\item Exchange as much pickup distance as possible for rebalancing distance (rebalancer)
\end{enumerate}

It is assumed that the operator does not know any demand patterns in advance. Furthermore, the analysis does not include ridesharing and its implications on fleet management algorithms.

\subsection{Selected Algorithms}

In this work we analyze four different operating strategies from the literature,
which are briefly outlined below:

\begin{enumerate}
\item The \textbf{Load-balancing heuristic} is a strategy presented in \cite{bischoff2016simulation}.
For every dispatching time step $\delta t_D$ , it is checked whether there are more available vehicles than requests.
If this is the case, it iterates on the list of requests and assigns to each request the closest vehicle. If there are more open
requests than available vehicles, the controller iterates on the available vehicles and assigns the closest open request to each vehicle.
The assignments are binding, i.e. they are not reopened once established.
\item The \textbf{Global Euclidean Bipartite Matching} (Hungarian algorithm) dispatcher determines an optimal bipartite
matching between all open requests and available vehicles in every dispatching time step
$\delta t_D$. The distance function used is the Euclidean distance which allows to use fast
algorithms, e.g. \cite{agarwal2004near}. In contrast to the previous strategy, the
assignments can be changed until a vehicle actually reaches its target.
% REF? /sh:  For a given set of open requests and available vehicles, this strategy can be considered as the optimal dispatching strategy based on Euclidean distances.
\item In \cite{pavone2011load} a feedforward strategy is presented on how to rebalance vehicles between different vertices
in a directed graph $G = (V,E)$. For each vertex $i$ and time step $\delta_t$, the arrival rates $\lambda_i$ and transition probabilities $p_{ij}$ for
any nodes $v_i, v_j \in V$  are computed from historical data. The linear program in equation \ref{eq:linearprogram}
computes the optimal rebalancing flows $\alpha _{ij}$ for an equilibrium point of the underlying flow
model with travel times $T_{i,j} \forall v_i, v_j \in V$.
\begin{align}
&\textnormal{minimize}& &\sum_{i,j} T_{i,j} \alpha _{ij} & && && \nonumber \\
&\textnormal{subject to}&
&\sum_{i \neq j} \alpha _{ij} - \alpha _{ji} =-\lambda_i  + \sum_{i \neq j} \lambda_j p_{ji}
& &\forall i \in V &\label{eq:linearprogram} \\
&& &\alpha_{ij} \geq 0& & \forall i, j \in V& \nonumber
\end{align}
To implement this strategy, we divided the city of Zurich into a set of areas.
The nodes from \cite{pavone2011load} represent the centroids of these areas on
which a complete directed graph called virtual network is placed, see
figure \ref{fig:study_area_vnodes}. Available cars are continuously rebalanced between
the vertices of the virtual network according to the static rebalancing rates $\alpha_{ij}$.
As \cite{pavone2011load} does not detail the proposed dispatching algorithm for this strategy, we match
cars using global Euclidean bipartite matching. Rebalancing vehicles cannot be dispatched
until they reach their destination virtual node.
\item The last implemented strategy is a novel derivation from \cite{pavone2011load}. Instead of a pure
feedforward solution, here in every rebalancing timestep $\delta t_R$
for every area of the virtual network the avaialble cars and open requests are counted
and fed into a mixed integer linear program derived from equation \ref{eq:linearprogram}
calculating the number of cars to be sent from virtual vertex $i$ to virtual vertex $j$.
\end{enumerate}

While the first two algorithms only perform the dispatching task, the latter two
are designed to rebalance the available fleet of AVs.
