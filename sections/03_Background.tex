\section{Control of an AMoD System}
\label{sec:background}

An AMoD service only makes sense if it is attractive to any customers. More
specifically, it can only be maintained if a sufficient number of customers
wants to use the service such that the financial benefits for the operator
exceed the costs.

While a multitude of factors influence the attractiveness of the service (perhaps
multimedia offers in the vehicle, the quality of Wifi, ...) the authors assume
two key properties: The time that passes between a customer making a request
and a vehicle arriving (i.e. the wait time) and the price that is charged from
the customer. All else being equal, an operator that can offer the shortest wait
times at the lowest price will attract more customers than his competitors. For
now, it remains unknown how those two factors would be valued against each other
by potential customers.

Based on the customer expectation, a number of options are available for the
operator to become profitable. Here, we focus on two strategies:

\begin{itemize}
\item The \textbf{fleet size} can be increased. In general, this should lead to
a decrease of wait time, because the availability of vehicles improves. However,
having a larger number of vehicles imposes more fixed costs that would need to be
balanced by higher demand. In general, adding more vehicles to the fleet can be
regarded as a long-term investment that cannot be altered on a daily basis.
\item The \textbf{fleet control} can be optimized. Since in an AMoD system it is
asssumed that any vehicle can be tracked and controlled online, intelligent fleet
control algorithms can be used to minimize the wait times, but also minimize the
driven distance in order to save money. Applying the proper algorithm is a much
less costly intervention than increasing the fleet size with assumably smaller
effects, but may bring a competitive advantage on the market.
\end{itemize}

In the presented experiments both components are investigated by comparing a number
of control algorithms for fleets of varying sizes.

\subsection{Problem Statement}

For the algorithmic improvement of the fleet management the authors distinguish
between two stages:

\begin{itemize}
\item The \textbf{dispatching strategy} decides how to serve the demand, i.e.
the open customer requests, with the available supply (the available vehicles).
At any time the dispatcher can send tasks to pickup a specific customer to any
vehicle that is not currently having a customer onboard (since we do not consider
ride-sharing with multiple customers). Also a reassignment of a previously assigned
vehicle is possible at any time if a more viable request comes in.
\item The \textbf{rebalancing strategy} decides where to send vehicles when they
are not in use and the demand allows for supplementary movements of the vehicles.
The task of the rebalancer is to anticipate future requests and position vehicles
such that they are able to optimally react to the upcoming demand.
\end{itemize}

Hence, vehicles will produce three kinds of mileage:

\begin{itemize}
\item \textbf{Empty pickup mileage} is produced when an AV is dispatched
to a request and is driving to the pick-up location. It is the mileage that needs
to be covered in order to serve the customer in any way and may be minimized
by an intelligent dispatching algorithm.
\item \textbf{Empty rebalancing mileage} is produced when an AV is sent
to a different location where demand is expected. An ideal operator would
exchange all the pickup mileage in the system against rebalancing mileage, because
then a vehicle would always already be present when a request pops up.
\item \textbf{Customer mileage} is produced with a customer on-board. In any combination
of fleet size and control algorithm, this mileage stays constant, because it is
defined by the origin-destinaton relations of all customer trips.
\end{itemize}

Assuming a common pricing scheme that defines a price per distance, the customer mileage
is the only component that produces a benefit for the operator. All other mileage
can directly be translated into costs and should therefore be minimized. For general
demand patterns, however, it cannot be driven to zero. Treleaven et al. \cite{treleaven2011asymptotically}
show that it is bounded below by the earth mover's distance, which is a measure
of how different the distributions of trip origins and destinations are (see \cite{ruschendorf1985wasserstein}).

The objectives for a fleet management algorithm can therefore be defined as:

\begin{enumerate}
\item Minimize the total pickup distance given the non-optimal locations of the vehicles (dispatcher)
\item Exchange as much pickup distance as possible for rebalancing distance (rebalancer)
\end{enumerate}

\subsection{Selected Algorithms}

In this work we analyze four different operating strategies from literature, which are briefly outlined below:

[TODO: I would do this more in detail! Why not explicitly state the LPs, at least for the last one, which is ``new''? I think if somebody reads this he cannot really figure out whats going on.]

\begin{enumerate}
\item The single heuristic dispatcher is a strategy presented in \cite{bischoff2016simulation}. In every dispatching time step $\delta t_D$ If there are more available vehicles than requests, it iterates on the list of requests and assigns to each request the closest vehicle. If there are more open requests than available vehicles, the controller iterates on the available vehicles and assigns the closest open request to each vehicle. The assignments are binding, i.e. they are not reopened once concluded.
\item The global Euclidean bipartite matching dispatcher determines an optimal bipartite matching between all open requests and available vehicles in every dispatching time step $\delta t_D$. The used distance function is the Euclidean distance which allows to use fast algorithms, e.g. \cite{agarwal2004near}. In contrast to the previous strategy, the assignments can be changed until a vehicle actually reaches its target. If arrival probabilties for future time steps is taken into account, this strategy can be considered as the optimal dispatching strategy based on Euclidean distances.
\item In \cite{pavone2011load} a feedforward strategy is presented on how to rebalance vehicles between different vertices in a directed graph $G = (V,E)$. For each vertex $i$ and time step $\delta_t$, the arrival rates $\lambda_i$ and transition probabilities $p_{ij}$ for any nodes $v_i, v_j \in V$  are used in a linear program to compute the optimal rebalancing flows $\alpha _{ij}$ in that time step assuming that the system is at equilibrium. To implement this strategy, we divided the city of Zurich into a set of areas. The nodes from \cite{pavone2011load} represent the centroids of these areas on which a complete directed graph called virtual network is placed, see figure \ref{fig:virtualNetwork}. Available cars are continuously rebalanced between the vertices of the virtual network according to the static rebalancing rates $\alpha_{ij}$. As the work does not detail the proposed dispatching algorithm for this strategy, we match cares using global Euclidean bipartite matching. Rebalancing vehicles cannot be dispatched until they reach their destination.
\item The last implemented strategy is as well derived from \cite{pavone2011load}. Instead of a pure feedforward solution, here in every rebalancing timestep $\delta t_R$ for every area of the virtual network the avaialble cars and open requests are counted and fed into an integer linear program which calculates the number of cars $reb _{ij}$ to be sent from virtual vertex $i$ to virtual vertex $j$. As in the feedforward strategy, the matching of the cars is done via global Euclidean bipartite matching.
\end{enumerate}