\section{Relared Research}
\label{subs:literatureResearch}

Two-way mobility on demand systems, e.g. car sharing schemes like Mobility, in
Switzerland \citep{katzev2003car} are a well-established part of the modal share
of many cities. These schemes offer flexibility, competitive prices and good
service levels. However, their popularity is heavily limited by the fact that
the vehicle has to be dropped off at the origin of the journey. On the contrary,
in one-way mobility on demand systems customers can travel with a vehicle (e.g.
autonomous car or bike) from any origin to any destination in the city wich
dramatically increases the flexiblity of these systems.

The price for the increased flexibilty is system inbalance. Due to the
spacio-temporal and in general unbalanced characteristics of travel demand,
vehicles tend to accumulate at certain locations and get depleted at others.
Furthermore system inbalance is not an exception but occurs for most demand
patterns. This can be seen for instance using queuing-theoretical arguments
as shwon in \citep{zhang2016control}.

System inbalance leads to drastically decreased service levels and must be
countered with the targeted repositioning of vehicles from overfull to empty
areas of the city. This repositioning of vehicles represents a significant
contribution to the operational cost of operators and therefore various strategies
have been tried to minimize the rebalancing effort. For intance in bike-sharing
schemes trucks are used to move vehicles from full to empty stations, in
\citep{pfrommer2014dynamic} algorithms have been proposed to route these
trucks at minimal cost. In \citep{ruch2014rule} price incentive controllers
are proposed to encourage customers to travel to depleted stations at the end
of their trip. Rebalancing was also researched for car sharing schemes, e.g.
in \citep{smith2013rebalancing} a scheme is proposed to reposition the
rebalancing drivers for one-way car sharing schemes in an optimal way. The
decisive difference of autonomous mobility on demand systems to the previous two
cases is that the vehicles can reposition themselves without the use of transporting
trucks or auxiliary drivers. Therefore rebalancing can be carried out more efficiently
and with more degrees of freedom.

Rebalancing of autonomous mobility on demand systems was first presented as a research
 problem in \citep{pavone2011load}. Optimal rebalancing flows for the vehicles are
 obtained by solving a linear program. In \citep{zhang2016control} the relation to
 queuing theoretical concepts was established. In \citep{treleaven2011asymptotically}
 the relation of the rebalancing effort to the underlying distributions of origins
 and destinations was established and it was shown that for general distributions
 the total minimal rebalancing distance is strictly more than zero.
 In   \citep{zhang2016model} the rebalancing problem was solved with a model
 predictive control algorithm which performs well but does not scale to large systems.

Most of these algorithms were tested on simplified traffic simulations that capture
the main characteristics but do not allow the same level of detail as agent based
traffic simulations like MATSim. For such simulation platforms various results
exist which are presented in the following paragraphs. Most of them do not
implement and compare the algorithms mentioned above which is an important
contribution that we make in this work.

Spieser et al. \cite{spieser2014toward} present a systematic approach to the
design of an autonomous mobility on demand system that is able to serve the entire
travel demand of Singapore with a fleet of automated shared vehicles. Analytic
results are used to compute both the minimal number of vehicles needed to stabilize
the number of open requests as well as the amount of vehicles that is needed to
provide an acceptable level of service. The authors conclude that a fleet size
of 25\% of today's vehicle fleet would be able to offer average wait times of
around 15 minutes and could reduce the external and internal costs of mobility by 50\%.
The study does not compare different fleet control algorithms and does not
elaborate on whether congestion effects have been taken into account.

% Took this out... I don't find it really convincing and comparable to our study
%In \cite{marczuk2015autonomous} a case study is presented where no private cars can enter the central business district of Singapore. A total of $25,525$ trips is served by autonomous taxis which operate either in station-based or free-floating scheme. In the station-based scheme a set of fixed stations exists where the cars return to after completion of a trip. In the free-floating scheme the cars remain parked at the destination of trips. The simulations are based on the SimMobility agent based simulation platform and include twelve different fleet sizes from $2,000$ to $7,500$ vehicles. The authors conclude that the free-floating scheme can serve $90\%$ of the demand at the maximum fleet size whereas the station-based model can only serve $68 \% $ of the demand. Furthermore they observe mean customer wait times that saturate at approximately $2.5 ~ \textnormal{mins}$ and $6,000$ vehicles. While the station-based and free-floating concepts are compared, a detailed comparison of dispatching and rebalancing operation strategies is not presented. Furthermore results on the fleet efficiency and performance are not shown as well as a more detailed analysis of the wait times (e.g. different wait time quantiles). As the CBD of Singapore attracts much more than $25,525$ trips during a day it would also be interesting to see the results with the full number of trips taking into account routing policies and congestion levels in the city.

% ride-sharing ... rather take the older one
Fagnant et al. \cite{fagnant2015dynamic} present a case study for Austin, Texas
which focuses on the use of shared autonomous vehicles with ride-sharing capabilities,
i.e. vehicles that can transport more than one customer under some circumstances.
The scenario presented on vehicles with unit capacity yields that 10\% of today's
vehicle fleet could serve the entire demand with average wait times of 4.49 min.

% Reidesharing
\cite{zachariah2014uncongested} present a case study for New Jersey also focused
on the potential for ride-sharing. The trips generated by a population of $8,791,894$
individuals in New Jersey are covered by walking and biking if the distance is
less than a mile. All other trips are either served by the New Jersey train system,
by autonomous taxis or both. The study concludes that the ride-sharing potential
is large, especially during rush-hour and autonomous vehicles could significantly
reduce congestion levels in the city. The required fleet size is not commented
as well as the influence of the rebalancing and dispatching strategy for the fleet.

In \cite{martinez2017assessing} the authors present a study on the effects of
introducting autonomous taxis and autonomous shared taxis to the city of Lisbon,
Portugal. The agent-based simulation includes 1.2 million trips and three scenarios:
a baseline scenario showing the current situation and two scenarios where private car,
taxi and bus trips are replaced by autonomous taxis and autonomous taxis and shared
taxis respectively. The fleet size of autonomous (shared) taxis is set at $4.8\%$ of
the baseline vehicle fleet. In these scenarios about 50-70 \% of trips are serviced
by the autonomous (shared) taxis which increases the vehicle occupancy
from $50 ~ \textnormal{mins}$ to $12.87 ~ h$ on average per day. The authors conclude
a decrease in cost by $55 \%$ per kilometer, highly increased transportation
accessibility in the city and carbon emission reductions of almost $40\%$. The
simulation does not consider the changes on traffic density parameters resulting
from self-driving vehicles. Furthermore the demand choice of the agents is static
and according to preset parameters. Finally the fleet control (rebalancing and dispatching)
for the (shared) autonomous taxis is implemented based on heuristics and a local
gradient based optimization method.

Boesch et al \cite{boesch2016autonomous} investigate a scenario of the greater
Zurich region in Switzerland. They use a demand pattern for private vehicles
generated with MATSim: $1.3$ million  private vehicle users out of a total of
$2.1$ million agents generate $3.6$ million trips. This demand profile generated
with the co-evolutionary algorithm inherent to MATSim is then post-processed
in a static simulation where $1-10 \%$ of the car trips are served by $10-100 \%$
of the total number of substituted users. The authors conclude that approximately
$30 ~ \%$ of the substituted fleet can serve almost $100 \%$ of the substituted
requests within less than $10 ~ \textnormal{mins}$ wait time. If this threshold
wait time is surpassed, then the request is dropped. The limitations of the results
are that no rebalancing or dispatching is taking place, furthermore network routing
 is not considered, travel times are based on Euclidean distance and a scaling factor.
  The demand profile is static and does not vary depending on service times,
  congestion rates and performance of the modes.

In contrast to the study for Zurich presented above, a case study for Berlin
presented in \cite{bischoff2016simulation} takes into account dynamic demand.
It considers a city-wide replacement of private vehicles with autonomous taxis.
 The dispatching of the car works according to a policy that distinguishes
 between oversupply (more available vehicles than open requests) and undersupply
 and matches the closest vehicle to an appearing request, the next available
 vehicle to the closest request respectively. Using this strategy called single
 heuristic dispatcher in our work, the authors are able to serve $4.7$ million
 requests generated by $1.1$ million car users with a fleet of $100,000$ autonomous
 vehicles. The recorded average wait time for this case is about $2.5 \%$ minutes
 and the $95\%$ quantile approximately $8.5$ minutes. The resulting sharing factor
 is approximately $10$ to $12$. The study is one of the first large-scale dynamic
 simulations of a shared autonomous taxi system, however it does not consider
 different rebalancing and dispatching strategies and it does not rigorously
 evaluate the performance metrics of the autonomous vehicle fleet as we do in [XYZ].


%======================== END of new part ============================================

%Let's go through this again. We need:

%\begin{itemize}
%\item Literature on rebalancing in general (Claudio bike rebalancing, but there is a lot for car-sharing available), just indicate importance not go into detal on algorithms
%\item Simulation studies on AVs (we have that already! commented out right now), start with the conceptual ones (Spieser) and mention we take algos from there
%\item Literature MATSim
%\end{itemize}

%It goes as follows: First literature on rebalancing is introduced, then the simulation studies on AVs
%are presented with the remark that they do NOT consider rebalancing. We end with Kockelman and Bösch,
%who use MATSim but just as a preparational step and lead to full MATSIm simulations, first Michal & Joschka,
%then ABMTRANS, which is used here.

%This way we have introduced everything we're using here.



%While
