\section{Conclusion \& Outlook}
\label{sec:Conclusion}

The study shows that the right choice of dispatching algorithm for an AMoD system
does not only have strong impact on the performance in terms of wait time for
the customer, but also that it bears a substantial economic advantage for the
operator. Operators with intelligent redispatching and rebalancing algorithms are able to attract
more customers through quicker pickups and lower prices than a competitor at small additional cost.

In order to assess the significance for real fleets of (not necessarily
automated) taxis it needs to be noted that all of the presented algorithms are
able to process dispatching and rebalancing tasks for fleets of thousands of
vehicles within minutes. It is perfectly feasible to control 100k vehicles in
five minute updates using a standard laptop for the computational tasks.

For the presented simulations, this still poses a burden, because there
a speedup compared to reality of hundreds of times is desired to be able
to run large numbers of simulations. Hence, the algorithms
are only tested on a subsample of 1\% of the agent population that is available.
In future studies, effort will be put into overcoming this restriction, either
by finding approximate formulations for the presented algorithms or pursuing research
on completely new algorithms.

Throughout the paper, a ``100\%'' demand scenario is used, in which all
trips that possibly could be undertaken by AVs are converted to the automated
mode. MATSim offers the possibility to explicitly
simulate attitudes toward new elements in the traffic system by defining utilities
for using specific modes with distinct valuation of travel costs, travel times and
distances. This way, by integrating the presented algorithms into the full
MATSim loop, as shown in \cite{horl_abmtrans17}, the actual attractiveness of an
AV service can be analysed including the tradeoff that people make between
paying for the service, spending time in the vehicle and having to wait for it.
Naturally, not 100\% of possible trips would be performed by AV then, but only
a fraction. Future work will take these considerations into account.
